\begin{song}{title=\centering Řekni\carka kde ty kytky jsou \\\normalsize Pete Seeger  \vspace*{-0.3cm}}  %% sem se napíše jméno songu a autor
\moveright 4cm \vbox{      %Varianta č. 1  ---> Jeden sloupec zarovnaný na střed	

\sloka 
  ^{C{\color{white}\_\_}}Řekni, kde ty ^{Ami}kytky jsou, ^{F}co se s nima ^{G{\color{white}\_\_\_}}mohlo stát.
  
  ^{C{\color{white}\_\_}}Řekni, kde ty kytky ^{Ami}jsou, kde ^{Dmi{\color{white}\_}}mohou ^{G}být,

  ^{C{\color{white}\_\_}}dívky je tu ^{Ami{\color{white}\_}}během dne ^{D7{\color{white}\_\_}}otrhaly ^{G}do jedné,
  
  ^{F}kdo to kdy ^{C{\color{white}\_\_\_\_}}pochopí, ^{F}kdo to kdy ^{G{\color{white}\_\_\_}C\,\,\,}pochopí?

\sloka
  Řekni, kde ty dívky jsou, co se s nima mohlo stát.
  
  Řekni, kde ty dívky jsou, kde mohou být,
  
  muži si je vyhlédli, s sebou domů odvedli,
  
  kdo to kdy pochopí, kdo to kdy pochopí?

\sloka
  Řekni, kde ti muži jsou, co se s nima mohlo stát.
  
  Řekni, kde ti muži jsou, kde mohou být,
  
  muži v plné polní jdou, do války je zase zvou,
  
  kdo to kdy pochopí, kdo to kdy pochopí?

\sloka
  A kde jsou ti vojáci, co se s nima mohlo stát.
  
  A kde jsou ti vojáci, kde mohou být,
  
  řady hrobů v zákrytu, meluzína kvílí tu,
  
  kdo to kdy pochopí, kdo to kdy pochopí?

\sloka = 1.


}
\setcounter{Slokočet}{0}
\end{song}

