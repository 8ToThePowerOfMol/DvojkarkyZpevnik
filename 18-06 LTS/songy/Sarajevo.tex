\begin{song}{title=\centering Sarajevo \\\normalsize Jaromír Nohavica \vspace*{-0.3cm}}  %% sem se napíše jméno songu a autor
\moveright \stred \vbox{      %Varianta č. 1  ---> Jeden sloupec zarovnaný na střed	

\sloka 
	^{Emi}Přes haličské pláně ^{Ami/F\# }vane vítr zlý,
	
	to ^{H7\,{\color{white}\_}}málo, co jsme měli, nám ^{Emi\,\,}vody sebraly.
	
	Jako tažní ptáci, ^{Ami/F\# }jako rorýsi
	
	^{H7\,{\color{white}\_}}letíme nad zemí, dva ^{Emi{\color{white}\_}}modré dopisy.

\refren
	^{Emi}Ještě hoří oheň a ^{Ami{\color{white}\_}}praská dřevo,
	
	^{D7/F\# }ale už je čas jít ^{G\,\,{\color{white}\_}}spát. ^{H7}

	^{Emi{\color{white}\_}}Tamhle za kopcem je ^{Ami{\color{white}\_\_}}Sarajevo,
	
	tam ^{H7\,\,\,\,\,\,\,\,\,\,}budeme se zítra ráno ^{Emi\,\,}brát.

\sloka
	Farář v kostele nás sváže navěky,
	
	věnec tamaryšku pak hodí do řeky,
	
	voda popluje zpátky do moře,
	
	my dva tady dole a nebe nahoře.

\refren

\sloka
	Postavím ti dům z bílého kamení,
	
	dubovými prkny on bude roubený,

	aby každý věděl, že jsem tě měl rád,
	
	postavím ho pevný, navěky bude stát.

\refren
	Ještě hoří oheň a praská dřevo,
	
	ale už je čas jít spát.
	
	Tamhle za kopcem je Sarajevo,
	
	tam zítra budeme se, lásko, brát\elipsa\dots


}
\setcounter{Slokočet}{0}
\end{song}

