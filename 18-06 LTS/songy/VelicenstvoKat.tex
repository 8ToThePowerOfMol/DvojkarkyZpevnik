\begin{song}{title=\centering Veličenstvo Kat \\\normalsize Karel Kryl  \vspace*{-0.3cm}}  %% sem se napíše jméno songu a autor
\moveright -0.9cm \vbox{      %Varianta č. 1  ---> Jeden sloupec zarovnaný na střed	
\begin{minipage}[t]{0.55\textwidth}\setlength{\parindent}{0.45cm}  %Varianta č. 2 --> Dva sloupce

\sloka
	V ^{Ami{\color{white}\_\_\_}}ponurém osvětlení ^{G{\color{white}\_\_\_\_\_}}gotického ^{Ami\,\,}sálu
 
	^{C{\color{white}\_\_\_\_}}kupčíci vyděšení ^{Emi{\color{white}\_}}hledí do ^{E}misálů

	a ^{Ami{\color{white}\_\_}}houfec ^{F{\color{white}\_\_\_\_}}mordýřů si ^{G{\color{white}\_\_}}žádá ^{{\color{white}\_\_\_\_\_}C\,\,G}požehnání,
         
	/: ^{Dmi{\color{white}\_}}vždyť první z ^{Ami{\color{white}\_\_}}rytířů je ^{E7{\color{white}\_\_\_\_\_\_}}Veličenstvo ^{Ami}Kat. :/

\sloka
	Kněz-ďábel, co mši slouží, z oprátky má štolu,
	
	pod fialovou komží láhev vitriolu,
	
	pach síry z hmoždířů se valí k rudé kápi

	/: prvního z rytířů, hle: Veličenstvo Kat. :/
	
	
\refren
	^{C}Na korouhvi ^{G{\color{white}\_\_\_}}státu je ^{F{\color{white}\_\_\_\_}}emblém s ^{G}gilotinou,

	z ^{C{\color{white}\_\_\_\_\_}}ostnatýho ^{G{\color{white}\_\_\_}}drátu ^{F{\color{white}\_\_\_}}páchne to ^{{\color{white}\_}G}shnilotinou,

	v ^{Dmi\,\,}kraji hnízdí hejno ^{{\color{white}\_\_\_\_\_}Ami}krkavčí,

	^{Dmi\,\,}lidu vládne mistr ^{{\color{white}\_\_\_\_\_}E7}popravčí.

\sloka
	Král klečí před Satanem na žezlo se těší
   
	a lůza pod platanem radu moudrých věší

	a zástup kacířů se raduje a jásá,
 	
 	/: vždyť prvním z rytířů je Veličenstvo Kat. :/

\sloka
	Na rohu ulice vrah o morálce káže,
   
	před vraty věznice se procházejí stráže,
  
	z vojenských pancířů vstříc černý nápis hlásá,
   
   	/: že prvním z rytířů je Veličenstvo Kat. :/
   	
 \refren  	
	Nad palácem vlády ční prapor s gilotinou,
 
 	děti mají rády kornouty se zmrzlinou,
  
  	soudcové se na ně zlobili,
   
   	zmrzlináře dětem zabili.


\end{minipage}\begin{minipage}[t]{0.55\textwidth}\setlength{\parindent}{0.45cm}\vspace*{0.55cm}  % V případě varianty č.2 jde odsud text do pravé části

\sloka
	Byl hrozný tento stát, když musel jsi se dívat,
   
   	jak zakázali psát a zakázali zpívat,
   
   	a bylo jim to málo, poručili dětem
   
   	/: modlit se, jak si přálo Veličenstvo Kat. :/

\sloka
	S úšklebkem Ďábel viděl pro každého podíl,
  
  	syn otce nenáviděl, bratr bratru škodil,
   
   	jen motýl smrtihlav se nad tou zemí vznáší,
   
   	/: kde v kruhu tupých hlav dlí Veličenstvo Kat. :/


\end{minipage}
}
\setcounter{Slokočet}{0}
\end{song}

