\begin{song}{title=\centering Hercegovina \\\normalsize \vspace*{-0.3cm}}  %% sem se napíše jméno songu a autor
\moveright 3.7cm \vbox{      %Varianta č. 1  ---> Jeden sloupec zarovnaný na střed	

\sloka 
	/: ^{G{\color{white}\_\_\_\_\_\_\_\_\_}}Hercegovina, cha-cha-cha-ju-cha-cha,
	
	^{D\,\,}lautr rovina, cha-cha-cha-ju-cha-cha, :/
	
	/: ^{G}tu musela ^{C{\color{white}\_\_\_\_\_\_}}vybojovat ^{D{\color{white}\_\_\_\_\_\_}G}infanteria, cha-cha. :/
	
\sloka
	/: Infanteria, cha-cha-cha-ju-cha-cha,
	
	čestná setnina, cha-cha-cha-ju-cha-cha, :/
	
	/: ta musela bojovati za císaře pána, cha-cha. :/
	
\sloka
	/: Za císaře pána, cha-cha-cha-ju-cha-cha,
	
	a jeho rodinu, cha-cha-cha-ju-cha-cha :/
	
	/: museli jsme vybojovat Hercegovinu, cha-cha. :/
	
\sloka
	/: Vzhůru po stráni, š š š š š,
	
	šnelcuk uhání, š š š š š, :/
	
	/: a pod strání jsou schováni Mohamedáni, cha-cha. :/
	
\sloka
	/: Mohamedáni, cha-cha-cha-ju-cha-cha,
	
	to jsou pohani, cha-cha-cha-ju-cha-cha, :/
	
	/: kalhoty maj roztrhaný a smrkaj do dlaní, cha-cha. :/
	
\sloka
	/: Tyhle Turkyňe, cha-cha-cha-ju-cha-cha,
	
	tlustý jak dýně, cha-cha-cha-ju-cha-cha. :/
	
	/: Císař pan je nerad vidí ve svý rodině, cha-cha. :/
	
\sloka
	/: Tuto píseň skládal, cha-cha-cha-ju-cha-cha,
	
	jeden hoch mladý, cha-cha-cha-ju-cha-cha, :/
	
	/: který se dal skrz svou holku k infanterii, cha-cha. :/
	

}
\setcounter{Slokočet}{0}
\end{song}
