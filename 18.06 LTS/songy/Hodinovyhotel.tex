\begin{song}{title=\centering Hodinový hotel \\\normalsize Mňága \& Žďorp \vspace*{-0.3cm}}  %% sem se napíše jméno songu a autor
\moveright 4cm \vbox{      %Varianta č. 1  ---> Jeden sloupec zarovnaný na střed	

\sloka 
	^{C{\color{white}\_\_\_}}Tlusté koberce plné ^{Emi{\color{white}\_\_}}prachu,

  	^{A{\color{white}\_\_\_\_}}poprvé s holkou, ^{F{\color{white}\_\_\_}}trochu ^{G{\color{white}\_\_\_\_}}strachu

	a ^{C{\color{white}\_\_\_\_}}stará dáma od vedle zas ^{Emi{\color{white}\_\_}}vyvádí,
   
   	^{A{\color{white}\_\_\_}}zbyde tu po ní zvadlé ^{F{\color{white}\_\_\_\_}}kapradí a ^{G}pár prázdných flašek.

\sloka
	Veterán z legií nadává na revma,
   
   	vzpomíná na Emu, jak byla nádherná
   
   	a všechny květináče už tu historku znají,

	ale znova listy nakloní a poslouchají, vzduch voní kouřem.
	
\refren
	A\, /: ^{C\,\,}svět je, ^{Emi}svět je jenom hodinový ^{A{\color{white}\_\_}}hotel

	a můj ^{F{\color{white}\_\_\_}}pokoj je ^{G{\color{white}\_\_\_\_}}studený a ^{C{\color{white}\_\_\_\_}}prázdný. :/

\sloka
	Vezmu si sako a půjdu do baru,
   	
   	absolventi kurzů nudy pořád postaru,
   	
   	kytky v klopě vadnou, dívám se okolo po stínech,
   	
   	kterou? No přeci žádnou!


\refren
	Vracím se pomalu nahoru,
   
   	cestou potkávám ty, co už padají dolů,
   	
	a vedle v pokoji někdo šeptá: \uv{Jak ti je?}
   	
   	Za oknem prší a déšť stejně nic nesmyje.

\refren

\refren \dots\elipsa a fialový.

}
\setcounter{Slokočet}{0}
\end{song}
