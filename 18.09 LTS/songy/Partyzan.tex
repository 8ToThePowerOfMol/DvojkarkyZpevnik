\begin{song}{title=\centering Partyzán  \\\normalsize Jaromír Nohavica  \vspace*{-0.3cm}}  %% sem se napíše jméno songu a autor
\moveright \stred \vbox{      %Varianta č. 1  ---> Jeden sloupec zarovnaný na střed	

\sloka 
	Když ^{C{\color{white}\_\_}}přišli, ^{C/H}bylo ^{Ami\,}léto,
	
	spálená ^{C{\color{white}\_}}pole ^{C/H{\color{white}\_}}páchla ^{Ami\,\,}krví
	
	a kdo se vzdal, byl ^{{\color{white}\_}G}ušetřen
	
	^{Dmi}a já vzal ^{F{\color{white}\_\_}}zbraň do ^{C{\color{white}\_\_\_}}rukou, ^{C/H}hm. ^{Ami}
	
\sloka
	Svoje pravé jméno už neznám,
	
	ženu nemám a syna také ne,
	
	a je nás víc podobných
	
	a naše oči hledí vpřed, hm.

\sloka
	Včera spali jsme v rozbité kůlně,
	
	 stará žena nám vařila polívku,
	
	ráno přišli vojáci,
	
	její dům hoří a ona v něm, hm.

\sloka
	Bylo nás osm, a jsme jen tři,
	
	bůhví, kolik nás bude zítra,
	
	ale jít musíme dál,
	
	hory jsou naše vězení, hm.
	
\sloka
	A vítr fouká, vítr fouká,
	
	nad lesy krouží bílí sokoli,
	
	den svobody se blíží,
	
	pak sejdem z hor do údolí, hm.

\sloka
Na na na\elipsa\dots

\sloka = 5. 


}
\setcounter{Slokočet}{0}
\end{song}


