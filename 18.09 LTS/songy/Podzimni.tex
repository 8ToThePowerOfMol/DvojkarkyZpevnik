%\documentclass[../main.tex]{subfiles}

\begin{song}{title=\centering Podzimní \\\normalsize Karel Plíhal  \vspace*{-0.3cm}}  %% sem se napíše jméno songu a autor
\moveright \stred \vbox{      %Varianta č. 1  ---> Jeden sloupec zarovnaný na střed	

\sloka
^{A{\color{white}\_\_\_\_}}Podzimní obloha dala se ^{D}do gala, 

^{E{\color{white}\_\_\_}}večerní vánek se do vlasů ^{A{\color{white}\_\_}}vplétá, 

a po tý obloze na křídle ^{D{\color{white}\_\_}}rogala 

s ^{E\,\,}tím vánkem ve vlasech Markéta ^{A{\color{white}\_}}létá.

\sloka
Nebe je modrý jako mý džíny, 

tak jsme si zpívali s klukama zamlada, 

zmizely smutky a podzimní splíny, 

prostě to všechno, co Markéta nerada. 

\sloka
Vysoko na nebe, hluboko do polí 

Markéta létá a přitom si zpívá, 

co oči nevidí, to srdce nebolí, 

je totiž podzim a brzo se stmívá. 

\sloka
Zmizely splíny a přívaly pláče 

a s nima ty protivný přízraky z minula,

připravte obvazy, dlahy a fáče, 

kdyby se náhodou se zemí minula. 

\sloka = 1.

\sloka = 2.

\sloka = 3.

\sloka = 4.

\sloka
/: Podzimní obloha dala se do gala, 

večerní vánek se do vlasů vplétá. :/ 

 Podzimní obloha dala se do gala, 
 
večerní vánek se do vlasů vplétá.

}
\setcounter{Slokočet}{0}
\end{song}
