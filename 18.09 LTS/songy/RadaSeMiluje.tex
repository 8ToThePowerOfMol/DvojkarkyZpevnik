%\documentclass[../main.tex]{subfiles}

\begin{song}{title=\centering Ráda se miluje \\\normalsize Karel Plíhal  \vspace*{-0.3cm}}  %% sem se napíše jméno songu a autor
\moveright 4cm \vbox{      %Varianta č. 1  ---> Jeden sloupec zarovnaný na střed	

\refren
^{Hmi}Ráda se miluje, ^{A{\color{white}\_}}ráda ^{D}jí,

^{G{\color{white}\_}}ráda si ^{F#mi\,}jenom tak ^{Hmi{\color{white}\_}}zpívá, 

vrabci se na plotě ^{A\,{\color{white}\_}D\,}hádají, 

^{G{\color{white}\_\_}}kolik že ^{F#mi}času jí ^{Hmi{\color{white}\_}}zbývá.

\sloka
^{G}Než vítr dostrká k ^{D\,{\color{white}\_}}útesu ^{G}tu její legrační ^{{\color{white}\_}D\,\,F#mi}bárku 

a ^{Hmi\,\,}Pámbu si ve svým ^{A{\color{white}\_}D}notesu ^{G{\color{white}\_\_}}udělá ^{F#mi}jen další ^{Hmi\,\,}čárku.

\refren

\sloka
Psáno je v nebeské režii, a to hned na první stránce, 

že naše duše nás přežijí v jinačí tělesný schránce. 

\refren

\sloka
Úplně na konci paseky, tam, kde se ozvěna tříští, 

sedí šnek ve snacku pro šneky -- snad její podoba příští. 


\refren

}
\setcounter{Slokočet}{0}
\end{song}

