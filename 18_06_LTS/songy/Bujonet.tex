\begin{song}{title=\centering Bujonet \\\normalsize Znouzecnost \vspace*{-0.3cm}}  %% sem se napíše jméno songu a autor
\moveright 5.4cm \vbox{      %Varianta č. 1  ---> Jeden sloupec zarovnaný na střed	

{\color{white}\_\_\_}


\sloka
   ^{G}Na frontě západní ^{C{\color{white}\_\_\_}}ticho a ^{Emi\,\,}klid,

   ^{Ami{\color{white}\_}}bláta po kotníky ^{C{\color{white}\_\_\_\_\_}}špinavej ^{D{\color{white}\_\_}}sníh.

   ^{G{\color{white}\_\_\_\_}}Vojáci v kotlíku ^{C{\color{white}\_\_}}oběd ^{Emi\,\,}vaří,

   ^{Ami{\color{white}\_\_}}polívka bublá a ^{C{\color{white}\_\_}}oheň ^{{\color{white}\_\_}G}kouří


\sloka
   Zaduněl za kopcem polední zvon,
   
   v zákopech voní silný bujón.
   
   Polívka železem kořeněná,
   
   slzama hořkýma osolená.


\refren   
   ^{C}Za bujón míchanej ^{F{\color{white}\_\_\_\_\_}C{\color{white}\_\_}}bajonetem

   ^{Dmi{\color{white}\_\_\_}}půjdeme do války s ^{C{\color{white}\_\_\_}}celým ^{\,\,G}světem.

   ^{C{\color{white}\_\_\_\_\_}}Polívka železem ^{F{\color{white}\_\_\_\_\_}C{\color{white}\_}}kořeněná, 

   ^{Ami{\color{white}\_\_}}armáda sytá je ^{F{\color{white}\_\_\_\_\_}C{\color{white}\_}}spokojená.


\sloka
   Polívku míchanou bajonetem 
   
   nazývaj vojáci Bujónetem.
   
   Sedíce na bednách od munice
   
   utíraj do čepic svoje lžíce.


\refren


\sloka
   Na frontě západní ticho a klid,
   
   armáda srká horkou polívku.
   
   Bláta po kotníky a špinavej sníh,
   
   ešusy hřejou a zákopem zní:


\refren

\refren


}
\setcounter{Slokočet}{0}
\end{song}
