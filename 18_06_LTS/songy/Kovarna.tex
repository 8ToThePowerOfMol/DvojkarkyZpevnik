\begin{song}{title=\centering Kovárna \\\normalsize Tři sestry \vspace*{-0.3cm}}  %% sem se napíše jméno songu a autor
\moveright 4cm \vbox{      %Varianta č. 1  ---> Jeden sloupec zarovnaný na střed	

\sloka 
	^{A}Ve středu jsem na Kovárně, ^{C}ve čtvrtek jsem na Kovárně,

	^{F}i v pátek jsem na Kovárně, ^{G}v sobotu zas na Kovárně.
	
	^{\phantom{.}}Ve středu jsem na Kovárně, ve čtvrtek jsem na Kovárně,
	
	^{\phantom{.}}i v pátek jsem na Kovárně, v sobotu zas na Kovárně.

\sloka
	^{A}Můj dědek byl kovář ^{C}a má bába kovářka,

	^{D}muj strejda byl notář, ^{H}moje teta ^{C}notářka.
	
	^{\phantom{.}}Můj táta je kovář a má máma kovářka,
	
	^{\phantom{.}}muj brácha je notář a ségra je notářka.

\refren
	^{A}Středa, ^{C}čtvrtek, ^{D}pátek, ^{H}sobota a ^{C}neděle, 
	
	každej den jsem na Kovárně,	na Kovárně v Bráníku.

\sloka
	Vilu mám jen malou a v ní kupu dětí,
	
	umím jenom kovat, roky rychle letí,
	
	až já z toho umřu, na krchov mě nesou,
	
	na rakev mi dají půllitr mou milóóóůů.

\refren

\sloka
	Rakev mi ozdobí hustou bílou pěnou,
	
	syn se bude sklánět nad mou kovadlinou.
	
	Muj syn bude kovář, jeho žena kovářka
	
	a jejich syn notář, jejich dcera notářka.
	
	Už nebudu kovat, už nebudu kovat,
	
	všichni budou kovat, všichni budou kovat, jen já.

\refren

}
\setcounter{Slokočet}{0}
\end{song}