%\documentclass[../main.tex]{subfiles}

\begin{song}{title=\centering Lano co k nebi nás poutá \\\normalsize Traband  \vspace*{-0.3cm}}  %% sem se napíše jméno songu a autor
\moveright 3.2cm \vbox{      %Varianta č. 1  ---> Jeden sloupec zarovnaný na střed


\sloka
Já ^{F{\color{white}\_\_\_}}sedával v přístavu, ^{Gmi{\color{white}\_\_}}popíjel kořalu, ^{F}s holkama ^{{\color{white}\_\_}C7}laškoval.

A ^{F}bylo mi fuk, co je, ^{Gmi}hlavně když fajfka mi ^{F{\color{white}\_\_}C7}doutná.

Co ^{F\,\,}bylo už není, ^{Gmi{\color{white}\_\_}}všechno mý jmění jsem ^{A7\,\,}dávno ^{{\color{white}\_\_\_}Dmi}rozfofroval.

Jsme ^{F\,\,}silný, jak silný je ^{Gmi}lano, co k nebi nás ^{F C7 F}poutá.

\sloka
Ale najednou zmatek, když vešel ten chlápek, na mou duši! 

Objedná si drink a sedne si vedle do kouta.

Pak se nakloní ke mně a povídá jemně: \uv{Matouši!

Jsme silný, jak silný je lano, co k nebi nás poutá.}

\sloka
Já povídám: \uv{Pane, odkud se známe? Esli se nemýlíte?

A co je vám do mě, starýho mrchožrouta?}

On na to: \uv{Pojď, dej se na moji loď, má jméno \emph{Eternité}.} 

Jsme silný, jak silný je lano, co k nebi nás poutá.\\


^{Dmi D7}

\sloka
^{G}Ty jeho slova se ^{Ami\,\,\,}zařízly do mě ^{G\,\,\,\,}jako bys břitvou ^{D7\,\,\,\,\,\,}šmik,

^{G\,\,\,\,}jako když po ránu ^{Ami\,\,\,\,}vzbudí tě křik ^{G\,\,\,\,\, D7\,\,}kohouta.

^{G}Tak povídám: ,,Jdem!“ A ^{Ami}ještě ten den stal se ^{H7}ze mě ^{\,\,\,\, Emi}námořník. 

^{G}Jsme silný, jak silný je ^{Ami}lano, co k nebi nás ^{G D7 G}poutá.

\sloka
Tak zvedněme kotvy a napněme plachty, vítr začíná vát! 

Černý myšlenky vymeťme někam do kouta.

Hudba ať hraje o dobytí ráje, teď není čeho se bát.

/: Jsme silný, jak silný je lano, co k nebi nás poutá. :/\\


}
\setcounter{Slokočet}{0}
\end{song}
