\begin{song}{title=\centering Lilie \\\normalsize Karel Kryl  \vspace*{-0.3cm}}  %% sem se napíše jméno songu a autor
\moveright 4cm \vbox{      %Varianta č. 1  ---> Jeden sloupec zarovnaný na střed	

\sloka 
	^{Dmi}Než zavřel bránu, oděl se do oceli, ^{A} a zhasil ^{Dmi{\color{white}\_}}svíci,  

	bylo už k ránu, políbil na posteli, ^{A} svou ženu ^{Dmi{\color{white}\_}}spící. 

	/: ^{F{\color{white}\_\_\_}}Spala jak víla, ^{C\,\,}jen vlasy halily ji, 
	
	^{Dmi}jak zlatá žíla, ^{B}jak jitra v ^{A{\color{white}\_\_\_\_}}Kastilii, 
	
	^{Dmi{\color{white}\_\_}}něžná a bílá jak rosa na lilii, ^{B} ^{A}jak luna ^{Dmi{\color{white}\_}}bdící. :/ 

\phantom{tom}

(pískání) \textbf{Dmi\, B\, A\, Dmi\, B\, A\, Dmi}

\sloka
	Jen mraky šedé a ohně na pahorcích -- svědkové němí, 
	
	lilie bledé svítily na praporcích, když táhli zemí.

	/: Polnice břeskné vojácká melodie, 

	potoky teskné -- to koně zkalili je, 
	
	a krev se leskne, když padla na lilie kapkami třemi. :/ 

\sloka
	Dozrály trnky, zvon zvoní na neděli a čas se vleče, 

	rezavé skvrnky zůstaly na čepeli u jílce meče.
	
	/: S rukama v týle jdou vdovy alejemi, 

	za dlouhé chvíle zdobí se liliemi, 
	
	lilie bílé s rudými krůpějemi trhají vkleče. :/  



}
\setcounter{Slokočet}{0}
\end{song}
