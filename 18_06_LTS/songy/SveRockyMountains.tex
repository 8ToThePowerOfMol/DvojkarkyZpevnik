\begin{song}{title=\centering Své Rocky Mountains \\\normalsize Vojta Kidák Tomáško  \vspace*{-0.3cm}}  %% sem se napíše jméno songu a autor
\moveright 4.8cm \vbox{      %Varianta č. 1  ---> Jeden sloupec zarovnaný na střed	

\sloka 
	Je ^{A{\color{white}\_\_\_}}srpnový pátek a na horách sněží,

	^{D{\color{white}\_\_}}někdo jde vzhůru, někdo si věří,
	
	^{A{\color{white}\_\_}}krosna na zádech ^{D}tíží jak žulový ^{A{\color{white}\_}}blok.
	
	^{\phantom{.}}Zatím co dole je léto a stromy
	
	a ^{D{\color{white}\_\_}}silácké řeči těch co se bojí
	
	^{A}do bílé pláně ^{D{\color{white}\_\_}}udělat první ^{A{\color{white}\_}}krok.

\refren
	^{D{\color{white}\_\_\_}}Vždycky se najdou a ^{G{\color{white}\_\_}}znovu to zkouší
	
	^{D{\color{white}\_\_}}slunce jak plamen ^{A{\color{white}\_}}bodá je v očích,
	
	^{E{\color{white}\_}}rvou se jak koně, ^{D}jen aby našli své ^{A}já.

\sloka	
	Sklonil jsem hlavu a ptal se sám sebe,
	
	jestli jak oni mám to svoje nebe
	
	tak blízko na dosah a nejsem na kolenou.
	
	Bez velkých řečí a ohraných frází,
	
	vstával jsem do tmy a nejmíň mi schází
	
	publikum, co chce mít za každou cenu své šou.

\refren

\sloka
	Možná mám poslední, poslední šanci,
	
	vzít svoji víru a v ledovém tanci
	
	na vrchol vynést svou vlajku a vidět ji vlát.
	
	Má každý před sebou své Rocky Mountains,
	
	svůj kopec z dětství, co zdál se být zlatý,
	
	ale kolik jich ztratilo cepín a muselo vzdát. 

\refren

\refren

\refren


}
\setcounter{Slokočet}{0}
\end{song}


