\begin{song}{title=\predtitle\centering Ráda se miluje \\\large Karel Plíhal  \vspace*{-0.3cm}}  %% sem se napíše jméno songu a autor
\begin{centerjustified}
\nejnejvetsi

\refren
^{Hmi}Ráda se miluje, ^{A{\color{white}\_}}ráda ^{D\,}jí,

^{G{\color{white}\_}}ráda si ^{F#mi\,}jenom tak ^*{Hmi}zpívá , 

vrabci se na plotě ^{A\,{\color{white}\_}D\,}hádají, 

^*{G}ko lik že ^{F#mi}času jí ^{Hmi{\color{white}\_}}zbývá.

\sloka
^{G}Než vítr dostrká k ^{D\,{\color{white}\_}}útesu ^{G}tu její legrační ^{D\,\,F#mi}bárku\:\:\: 

a ^{Hmi\,\,}Pámbu si ve svým ^{A{\color{white}\_}D}notesu ^*{G\,}ud ělá ^{F#mi}jen další ^{Hmi\,\,}čárku.

\refren

\sloka
Psáno je v nebeské režii, a to hned na první stránce, 

že naše duše nás přežijí v jinačí tělesný schránce. 

\refren

\sloka
Úplně na konci paseky, tam, kde se ozvěna tříští, 

sedí šnek ve snacku pro šneky -- snad její podoba příští. 


\refren

\end{centerjustified}
\setcounter{Slokočet}{0}
\end{song}
