%%%%%%%%%%%%%%%%%%%%%%%%%%%%%%%%%%%%%%%%%%%%%%%%%%%%%
%			ŠABLONA PÍSNIČEK v. 18.09               %
%%%%%%%%%%%%%%%%%%%%%%%%%%%%%%%%%%%%%%%%%%%%%%%%%%%%%
% Tento soubor slouží jako (naučná) šablona, pomocí 
% které lze vytvářet zdrojové soubory k jednotlivým 
% písním.
%%%%%%%%%%%%%%%%%%%%%%%%%%%%%%%%%%%%%%%%%%%%%%%%%%%%%
%			Jak psát soubory songů?                 %
%%%%%%%%%%%%%%%%%%%%%%%%%%%%%%%%%%%%%%%%%%%%%%%%%%%%%
%	1. Text písně se začíná psát na místě START 
%	   a končí na místě END. Zbylý text ignorujte.
%	2. Jak bude vypadat pdf písně zjistíte po tom, 
%	   co soubor zkompilujete pomocí souboru   
%      ../Generator/generator. 
%	3. Při psaní dodržujte následující TeX pravidla:
%	 a) Nový řádek napíšete pomocí dvou odsazení 
%	    tedy dvou enterů.
%	 b) Nová sloka se píší pomocí \sloka a odsazení.
%		Refrén se píše jako \refren, v případě více 
%		refrénů \refren[č. refrénu].
%	 c) Akordy se píšou tak, že napíšete před slovo,
%	    kde chcete mít akord (bez mezery):
%		^{AKORD1\,AKORD2...}.
%	4. Pokud chcete ušetřit tvůrcům práci, tak 
%	   si přečtěte další poučný soubor o typografii 
%	   ../../Typo_pravidla.txt.
%	5. Akordy stačí psát jen do první sloky, když 
%	   se nezmění -- kytaristé to zvládnou
%	7. Název písně pište na místo [NÁZEV] a autora 
%	   pište na místo [AUTOR] 
%	7. Jak psát věci na české klávesnici:
%	   \ = alt gr + q; [/] = alt gr f/g; 
%      {/} = alt gr + b/n; ^ = alt gr + 3 , cokoliv
%%%%%%%%%%%%%%%%%%%%%%%%%%%%%%%%%%%%%%%%%%%%%%%%%%%%%
%			Jak kompilovat jednotlivé písně?        %
%%%%%%%%%%%%%%%%%%%%%%%%%%%%%%%%%%%%%%%%%%%%%%%%%%%%%
%	1. Více návodu je k tomuto napsáno v souboru 
%      ../Generator/generator. 
%%%%%%%%%%%%%%%%%%%%%%%%%%%%%%%%%%%%%%%%%%%%%%%%%%%%%
%			Jak kompilovat celý zpěvník?			%
%%%%%%%%%%%%%%%%%%%%%%%%%%%%%%%%%%%%%%%%%%%%%%%%%%%%%
%	1. Více návodu je k tomuto napsáno v souboru
%	   ../Cely_zpevnik/zpevnik.tex.
%%%%%%%%%%%%%%%%%%%%%%%%%%%%%%%%%%%%%%%%%%%%%%%%%%%%%
\begin{song}{title=\predtitle \centering Batalion \\\large Spirituál kvintet }  %% sem se napíše jméno songu a autor

\vspace*{.5cm}

\textbf{Capo 3}

\moveright .1cm \vbox{%% ~ na konci řádku rozhazují balanc
\begin{centerjustified}
\vetsi
\refren[1]  %1
^{Ami}Víno ^{C\z}máš a ^{G\z Ami\:\:}markytánku, ^{C\z\z}dlouhá noc ^{G}se ^{Emi Ami}prohýří,~~~~~~~~~~~

   víno máš a chvilku spánku, díky, díky, verbíři.

\sloka
^{Ami\z}Dříve než se rozední, kapitán ^{C\z}k~osedlání ^{G\z}rozkaz ^{Ami Emi\z}dává,~~~~~~~~

^{Ami\z}ostruhami do slabin ^{\z G}koně ^{Ami Emi Ami\z}pohání.~~~~~~~~~~

Tam na straně polední, čekají ženy, zlaťáky a sláva,

do výstřelů z karabin zvon už vyzvání.

\refren[2]
^{Ami\z}Víno na ^{C}kuráž, a ^{G\z}pomilovat ^{\z Ami Emi}markytánku,~~~~

^{Ami\z }zítra do ^{C}Burgund, batalion ^{Ami Emi Ami \,\,\,\,}zamíří.~~~~~~~~~~~~~

Víno na kuráž a k ránu dvě hodiny spánku,

díky, díky vám královští verbíři.

\sloka
Rozprášen je batalion, poslední vojáci se k zemi hroutí,

na polštáři z kopretin budou věčně spát.

Neplač sladká Marion, verbíři nové chlapce přivedou ti,

za královský hermelín, padne každý rád.

\refren[2]

\refren[1]


\end{centerjustified}
}
\setcounter{Slokočet}{0}
\end{song}
