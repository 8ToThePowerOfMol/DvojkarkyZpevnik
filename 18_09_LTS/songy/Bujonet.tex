\begin{song}{title=\predtitle\centering Bujonet \\\large Znouzecnost \vspace*{-0.3cm}}  %% sem se napíše jméno songu a autor
\begin{centerjustified}
\nejvetsi

\sloka
   ^{G}Na frontě západní ^*{C}ti cho a ^{Emi\,\,}klid,

   ^{Ami}bláta po kotníky ^*{C}šp inavej ^*{D}sn íh.

   ^*{G}Vo jáci v kotlíku ^*{C}ob ěd ^{Emi\,\,}vaří,

   ^*{Ami}polív ka bublá a ^{C{\color{white}\_\_}}oheň ^{{\color{white}\_\_}G}kouří


\sloka
   Zaduněl za kopcem polední zvon,
   
   v zákopech voní silný bujón.
   
   Polívka železem kořeněná,
   
   slzama hořkýma osolená.


\refren   
   ^{C}Za bujón míchanej ^{F{\color{white}\_\_\_\_\_}C{\color{white}\_\_}}bajonetem

   ^*{Dmi}půjd eme do války s ^*{C}ce lým ^{\,\,G}světem.

   ^*{C}Po lívka železem ^{F{\color{white}\_\_\_\_\_}C{\color{white}\_}}kořeněná, 

   ^*{Ami}armád a sytá je ^{F{\color{white}\_\_\_\_\_}C{\color{white}\_}}spokojená.


\sloka
   Polívku míchanou bajonetem 
   
   nazývaj vojáci Bujónetem.
   
   Sedíce na bednách od munice
   
   utíraj do čepic svoje lžíce.


\refren


\sloka
   Na frontě západní ticho a klid,
   
   armáda srká horkou polívku.
   
   Bláta po kotníky a špinavej sníh,
   
   ešusy hřejou a zákopem zní:


\refren

\refren

\end{centerjustified}
\setcounter{Slokočet}{0}
\end{song}
