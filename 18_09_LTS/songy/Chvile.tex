%%%%%%%%%%%%%%%%%%%%%%%%%%%%%%%%%%%%%%%%%%%%%%%%%%%%%
%			ŠABLONA PÍSNIČEK v. 18.09               %
%%%%%%%%%%%%%%%%%%%%%%%%%%%%%%%%%%%%%%%%%%%%%%%%%%%%%
% Tento soubor slouží jako (naučná) šablona, pomocí 
% které lze vytvářet zdrojové soubory k jednotlivým 
% písním.
%%%%%%%%%%%%%%%%%%%%%%%%%%%%%%%%%%%%%%%%%%%%%%%%%%%%%
%			Jak psát soubory songů?                 %
%%%%%%%%%%%%%%%%%%%%%%%%%%%%%%%%%%%%%%%%%%%%%%%%%%%%%
%	1. Text písně se začíná psát na místě START 
%	   a končí na místě END. Zbylý text ignorujte.
%	2. Jak bude vypadat pdf písně zjistíte po tom, 
%	   co soubor zkompilujete pomocí souboru   
%      ../Generator/generator. 
%	3. Při psaní dodržujte následující TeX pravidla:
%	 a) Nový řádek napíšete pomocí dvou odsazení 
%	    tedy dvou enterů.
%	 b) Nová sloka se píší pomocí \sloka a odsazení.
%		Refrén se píše jako \refren, v případě více 
%		refrénů \refren[č. refrénu].
%	 c) Akordy se píšou tak, že napíšete před slovo,
%	    kde chcete mít akord (bez mezery):
%		^{AKORD1\,AKORD2...}.
%	4. Pokud chcete ušetřit tvůrcům práci, tak 
%	   si přečtěte další poučný soubor o typografii 
%	   ../../Typo_pravidla.txt.
%	5. Akordy stačí psát jen do první sloky, když 
%	   se nezmění -- kytaristé to zvládnou
%	7. Název písně pište na místo [NÁZEV] a autora 
%	   pište na místo [AUTOR] 
%	7. Jak psát věci na české klávesnici:
%	   \ = alt gr + q; [/] = alt gr f/g; 
%      {/} = alt gr + b/n; ^ = alt gr + 3 , cokoliv
%%%%%%%%%%%%%%%%%%%%%%%%%%%%%%%%%%%%%%%%%%%%%%%%%%%%%
%			Jak kompilovat jednotlivé písně?        %
%%%%%%%%%%%%%%%%%%%%%%%%%%%%%%%%%%%%%%%%%%%%%%%%%%%%%
%	1. Více návodu je k tomuto napsáno v souboru 
%      ../Generator/generator. 
%%%%%%%%%%%%%%%%%%%%%%%%%%%%%%%%%%%%%%%%%%%%%%%%%%%%%
%			Jak kompilovat celý zpěvník?			%
%%%%%%%%%%%%%%%%%%%%%%%%%%%%%%%%%%%%%%%%%%%%%%%%%%%%%
%	1. Více návodu je k tomuto napsáno v souboru
%	   ../Cely_zpevnik/zpevnik.tex.
%%%%%%%%%%%%%%%%%%%%%%%%%%%%%%%%%%%%%%%%%%%%%%%%%%%%%
\begin{song}{title=\predtitle \centering Chvíle \\\large Hop trop }  %% sem se napíše jméno songu a autor

\vspace*{.5cm}

\begin{centerjustified}
\vetsi
\sloka
^{Emi}Mám rád ty chvíle, kdy noc už pomalu ^{Hmi}končí,

^{\z Ami}chvíle, co patřej' jen ^{\z Hmi}těm, co ^{\z Emi}neusínaj',

toulavejm bláznům, když právě s jarem se ^{Hmi}loučí

a ^{Ami\z}za~dobrý slovo ti ^{Hmi}srdce svý na dlani ^{Emi}daj'.

\sloka
Mám rád ty chvíle, kdy holkám ve vočích svítí

slunce, když do korun stromů začlo se drát,

stejně jak v kapičkách rosy na pavoučích sítích,

to najednou chce se mi brečet a zároveň smát.

\refren
Ty rána s ^{A}vůní borový ^{\z Emi}smůly

měly by ^{A\z}zůstat navždycky v ^{\z Emi}nás,

ty rána s ^{C\z}vůní sekaný ^{Hmi}trávy

a ohně, co ^{C}právě ^{Hmi}pomalu ^{\z Emi}zhas'.

\sloka
Mám rád ty chvíle, kdy kluci v duchu si říkaj',

jak je to nádherný, léto před sebou mít,

spolu si přejou, ať moc rychle dny neutíkaj'

a trápení jsou někde v dálce, a kolem je klid.

\sloka = 1.

\refren

\end{centerjustified}

\setcounter{Slokočet}{0}
\end{song}


