%\documentclass[../main.tex]{subfiles}

\begin{song}{title=\centering Hlídač krav \\\normalsize Jaromír Nohavica \vspace*{-0.3cm}}  %% sem se napíše jméno songu a autor
\moveright \stred \vbox{      %Varianta č. 1  ---> Jeden sloupec zarovnaný na střed

\sloka
^{C{\color{white}\_\_}}Když jsem byl malý, říkali mi naši,

dobře se uč a jez vtipnou kaši,

až ^{F{\color{white}\_\_\_\_}}jednou vyrosteš, ^{G7{\color{white}\_\_}}budeš doktorem ^{C{\color{white}\_\_}}práv. 

Takovej doktor si sedí pěkně v suchu,

bere velký peníze a škrábe se v uchu,

ja jim na to řek!: \uv{chci být hlídačem krav}.

\refren
^{G7}Já chci: ^{C}Mít čapku s bambulí nahoře,

jíst kaštany a mýt se v lavoře, 

^{F}od rána po celý ^{G7}den zpívat si ^{C}jen. 

^{G7{\color{white}\_\_}}Zpívat si: ^{C\,\,F\,\,G7\,\,C\,\,}pam\,pam\,pam\elipsa.\elipsa.\elipsa.

\sloka
K vánocům mi kupovali hromadu knih,

co jsem ale vědět chtěl, to nevyčet' jsem z nich,

nikde jsem se nedověděl, jak se hlídají krávy.

Ptal jsem se starších a ptal jsem se všech,

každý na mě koukal jako na pytel blech,

každý se mě opatrně tázal na moje zdraví.

\refren

\sloka
Dnes už jsem starší, vím co vím,

mnohé věci nemůžu a mnohé smím,

a když je mi velmi smutno, lehnu si do mokré trávy.

S nohama křížem, rukama za hlavou,

koukám nahoru, na oblohu modravou,

kde se mezi mraky honí moje strakaté krávy.

\refren

\sloka
^{D\, G\, A7\, D\, E\, A\, H7\, E}Pam\,pam\,pam\elipsa.\elipsa.\elipsa.  

}
\setcounter{Slokočet}{0}
\end{song}
