\begin{song}{title=\predtitle\centering Hotel Hillary \\\large Poutníci \vspace*{-0.3cm}}  %% sem se napíše jméno songu a autor
\begin{centerjustified}
\nejnejvetsi

\sloka 
	Tvař se ^{Ami}trochu nostalgicky, už tě nikdy nepotkám,

	^{F}máš to jistý ^{G}provždycky, ^*{Ami}nastav~uši~vzpom ínkám,

	jak tě znám, i v tuto chvíli měl bys řeči peprný,

	jak ^*{F}te nkrát, když nám ^*{G}tv rdili, že ^*{Ami}je~vítr~st říbrný.


\refren
	A ^{F\z}tváře měli kožený, my jim zdrhli z průvodu,

	^*{Dmi}zahod ili lampióny a ^*{D}na šli hospodu,

	ale ^*{F}ta ky Jacquese Brela a s ním smutek z cizích vin

	a ^*{Dmi}žádos tivost těla a pak ^*{D}ra dost z volovin, a ta nám ^*{Ami}zbejv á.

\sloka
	Po večerech pro diváky dělali jsme kašpary,
	
	pak na zemi dva spacáky -- náš Hotel Hillary,
	
	slavný sliby jsme už znali i to, jak se neplní,
	
	a cenzoři nám kázali vo správným umění.

\refren

\sloka	
	A tak válčím s nostalgií, bují ve mně jako mech,
	
	a pořád všechno slibují starý hesla na domech,
	
	ty jsi splatil všechny dluhy i za Hotel Hillary
	
	a já vyhážu ty černý stuhy funebrákům navzdory.

\refren
	Vždyť mají tváře kožený, my jim zdrhnem z průvodu,
	
	zahodíme lampióny a najdem hospodu,
	
	a tam tvýho Jacquese Brela a s ním smutek z cizích vin
	
	a žádostivost těla a pak radost z volovin, /: a ta nám zbejvá. :/


\end{centerjustified}
\setcounter{Slokočet}{0}
\end{song}
