%%%%%%%%%%%%%%%%%%%%%%%%%%%%%%%%%%%%%%%%%%%%%%%%%%%%%
%			ŠABLONA PÍSNIČEK v. 18.09               %
%%%%%%%%%%%%%%%%%%%%%%%%%%%%%%%%%%%%%%%%%%%%%%%%%%%%%
% Tento soubor slouží jako (naučná) šablona, pomocí 
% které lze vytvářet zdrojové soubory k jednotlivým 
% písním.
%%%%%%%%%%%%%%%%%%%%%%%%%%%%%%%%%%%%%%%%%%%%%%%%%%%%%
%			Jak psát soubory songů?                 %
%%%%%%%%%%%%%%%%%%%%%%%%%%%%%%%%%%%%%%%%%%%%%%%%%%%%%
%	1. Text písně se začíná psát na místě START 
%	   a končí na místě END. Zbylý text ignorujte.
%	2. Jak bude vypadat pdf písně zjistíte po tom, 
%	   co soubor zkompilujete pomocí souboru   
%      ../Generator/generator. 
%	3. Při psaní dodržujte následující TeX pravidla:
%	 a) Nový řádek napíšete pomocí dvou odsazení 
%	    tedy dvou enterů.
%	 b) Nová sloka se píší pomocí \sloka a odsazení.
%		Refrén se píše jako \refren, v případě více 
%		refrénů \refren[č. refrénu].
%	 c) Akordy se píšou tak, že napíšete před slovo,
%	    kde chcete mít akord (bez mezery):
%		^{AKORD1\,AKORD2...}.
%	4. Pokud chcete ušetřit tvůrcům práci, tak 
%	   si přečtěte další poučný soubor o typografii 
%	   ../../Typo_pravidla.txt.
%	5. Akordy stačí psát jen do první sloky, když 
%	   se nezmění -- kytaristé to zvládnou
%	7. Název písně pište na místo [NÁZEV] a autora 
%	   pište na místo [AUTOR] 
%	7. Jak psát věci na české klávesnici:
%	   \ = alt gr + q; [/] = alt gr f/g; 
%      {/} = alt gr + b/n; ^ = alt gr + 3 , cokoliv
%%%%%%%%%%%%%%%%%%%%%%%%%%%%%%%%%%%%%%%%%%%%%%%%%%%%%
%			Jak kompilovat jednotlivé písně?        %
%%%%%%%%%%%%%%%%%%%%%%%%%%%%%%%%%%%%%%%%%%%%%%%%%%%%%
%	1. Více návodu je k tomuto napsáno v souboru 
%      ../Generator/generator. 
%%%%%%%%%%%%%%%%%%%%%%%%%%%%%%%%%%%%%%%%%%%%%%%%%%%%%
%			Jak kompilovat celý zpěvník?			%
%%%%%%%%%%%%%%%%%%%%%%%%%%%%%%%%%%%%%%%%%%%%%%%%%%%%%
%	1. Více návodu je k tomuto napsáno v souboru
%	   ../Cely_zpevnik/zpevnik.tex.
%%%%%%%%%%%%%%%%%%%%%%%%%%%%%%%%%%%%%%%%%%%%%%%%%%%%%
\vspace{-.5cm}
\begin{song}{title=\predtitle \centering Jacek \\\large Jaromír Nohavica }  %% sem se napíše jméno songu a autor

\vspace*{-.5cm}

\begin{centerjustified}
\vetsi
\sloka
^{G}Na druhém břehu řeky ^{D\z}Olše žije Jacek,

^{C}mám k němu stejně blízko ^{G}jak on ke mně,

^{G}máváme na sebe z ^{D}říční navigace,

^{C\z}dva spojenci a dvě ^*{\z G}spřá telené země,

jak malí kluci hážem z ^{D}břehů žabky,

^{C\z}kdo vyhraje, má z ^*{\z G}protější ho srandu,

hlavama kroutí česko-polské babky,

^*{C}dě láme prostě vlastní ^*{\z G}pro pagandu.

\sloka
Na mostě přátelství se tvoří dlouhé fronty

všelikých věcí za všelikou cenu,

já mám však na to velmi úzké horizonty

a Jacek velmi nenáročnou ženu,

týden co týden z břehů navigace

na sebe řveme:\uv{Chlapče, hlavu vzhůru!},

jak je to krásné, moci vykašlat se

na celní předpisy a na cenzuru, na na\elipsa.\elipsa.\elipsa.

\sloka
Z Piastovské věže na nás mává kníže Měšek

a směje se, až třepe se mu brada,

ve zprávách večer běží horký dnešek,

aspoň se máme s Jackem o co hádat,

on tvrdí svoje, já zas tvrdím svoje

a domluvit se někdy bývá marno,

tak spolu vedem pohraniční boje

a v praxi demonstrujem Solidarnosc, na na\elipsa.\elipsa.\elipsa.

\sloka
Na druhém břehu řeky Olše žije Jacek,

mám k němu stejně blízko jak on ke mně,

máváme na sebe z říční navigace,

dva spojenci a dvě spřátelené země

/: a voda plyne, plyne, plyne dlouhé věky,

řeka se kroutí jako modrá šňůrka

a my dva hážem kachnám vprostřed řeky

krajíčky chleba o dvou stejných kůrkách. :/

Na na na\elipsa.\elipsa.\elipsa.



\end{centerjustified}
\setcounter{Slokočet}{0}
\end{song}


