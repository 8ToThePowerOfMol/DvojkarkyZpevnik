%%%%%%%%%%%%%%%%%%%%%%%%%%%%%%%%%%%%%%%%%%%%%%%%%%%%%
%			ŠABLONA PÍSNIČEK v. 18.09               %
%%%%%%%%%%%%%%%%%%%%%%%%%%%%%%%%%%%%%%%%%%%%%%%%%%%%%
% Tento soubor slouží jako (naučná) šablona, pomocí 
% které lze vytvářet zdrojové soubory k jednotlivým 
% písním.
%%%%%%%%%%%%%%%%%%%%%%%%%%%%%%%%%%%%%%%%%%%%%%%%%%%%%
%			Jak psát soubory songů?                 %
%%%%%%%%%%%%%%%%%%%%%%%%%%%%%%%%%%%%%%%%%%%%%%%%%%%%%
%	1. Text písně se začíná psát na místě START 
%	   a končí na místě END. Zbylý text ignorujte.
%	2. Jak bude vypadat pdf písně zjistíte po tom, 
%	   co soubor zkompilujete pomocí souboru   
%      ../Generator/generator. 
%	3. Při psaní dodržujte následující TeX pravidla:
%	 a) Nový řádek napíšete pomocí dvou odsazení 
%	    tedy dvou enterů.
%	 b) Nová sloka se píší pomocí \sloka a odsazení.
%		Refrén se píše jako \refren, v případě více 
%		refrénů \refren[č. refrénu].
%	 c) Akordy se píšou tak, že napíšete před slovo,
%	    kde chcete mít akord (bez mezery):
%		^{AKORD1\,AKORD2...}.
%	4. Pokud chcete ušetřit tvůrcům práci, tak 
%	   si přečtěte další poučný soubor o typografii 
%	   ../../Typo_pravidla.txt.
%	5. Akordy stačí psát jen do první sloky, když 
%	   se nezmění -- kytaristé to zvládnou
%	7. Název písně pište na místo [NÁZEV] a autora 
%	   pište na místo [AUTOR] 
%	7. Jak psát věci na české klávesnici:
%	   \ = alt gr + q; [/] = alt gr f/g; 
%      {/} = alt gr + b/n; ^ = alt gr + 3 , cokoliv
%%%%%%%%%%%%%%%%%%%%%%%%%%%%%%%%%%%%%%%%%%%%%%%%%%%%%
%			Jak kompilovat jednotlivé písně?        %
%%%%%%%%%%%%%%%%%%%%%%%%%%%%%%%%%%%%%%%%%%%%%%%%%%%%%
%	1. Více návodu je k tomuto napsáno v souboru 
%      ../Generator/generator. 
%%%%%%%%%%%%%%%%%%%%%%%%%%%%%%%%%%%%%%%%%%%%%%%%%%%%%
%			Jak kompilovat celý zpěvník?			%
%%%%%%%%%%%%%%%%%%%%%%%%%%%%%%%%%%%%%%%%%%%%%%%%%%%%%
%	1. Více návodu je k tomuto napsáno v souboru
%	   ../Cely_zpevnik/zpevnik.tex.
%%%%%%%%%%%%%%%%%%%%%%%%%%%%%%%%%%%%%%%%%%%%%%%%%%%%%
\begin{song}{title=\predtitle \centering John Brown \\\large  }  %% sem se napíše jméno songu a autor

\vspace*{.5cm}

\nejvetsi
\begin{centerjustified}
\sloka
^{G\z}Černý muž pod bičem otrokáře žil,

^{C\z}černý muž pod bičem ^{G\z}otrokáře žil,

černý muž pod bičem otrokáře ^{Emi\z}žil,~~~

^{\z \z A7}kapitán John ^{D7\z}Brown to ^{G\z}zřel.

\refren
^{G\z}Glory, glory, haleluja,

^{C\z}glory, glory, ^{G}haleluja,

glory, glory, ^{\z Emi}haleluja,

\phantom{.}

^{\z \z \z Ami}kapitán John ^{D7\z}Brown to ^{G\z}zřel.

\sloka
/: Sebral z Virginie černých přátel šik, :/

sebral z Virginie černých přátel šik,

prapor svobody pak zdvih'.

\refren
\dots  prapor svobody pak zdvih'.

\sloka
/: V čele věrných město Harper's Ferry jal, :/

v čele věrných město Harper's Ferry jal,

právo vítězí a čest.

\refren
\dots  právo vítězí a čest.

\sloka
/: Hrstka statečných však udolána jest, :/

hrstka statečných však udolána jest,

kapitán John Brown je jat.

\refren
\dots  kapitán John Brown je jat.

\end{centerjustified}
\newpage
\begin{centerjustified}

\sloka
/: Zvony Charlestonu z dáli temně zní, :/

zvony Charlestonu z dáli temně zní,

Johnův den to poslední.

\refren
\dots  Johnův den to poslední.

\sloka
/: John Brown mrtev jest a jeho tělo tlí, :/

John Brown mrtev jest a jeho tělo tlí,

jeho duch však kráčí dál.

\refren
\dots  jeho duch však kráčí dál.

\end{centerjustified}
\setcounter{Slokočet}{0}
\end{song}
