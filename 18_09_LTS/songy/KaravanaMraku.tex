\begin{song}{title=\predtitle\centering Karavana Mraků \\\large Karel Kryl  \vspace*{-0.3cm}}  %% sem se napíše jméno songu a autor
\begin{centerjustified}

\sloka
	^{D}Slunce je zlatou skobou ^{Hmi\z}na~vobloze přibitý, 

	^{G}pod sluncem ^{A}sedlo kožený, ^{D A7} 
	
	^{D}pod sedlem kůň, pod koněm ^{Hmi}moje boty rozbitý 

	^*{G\z}a starý ^{A}ruce sedřený. ^{D} 

\refren[1]
	^{D7}Dopředu ^{G}jít s tou ^{A}karavanou ^{Hmi}mraků, 

	schovat svou ^{G}pleš pod ^{A}stetson ^{Hmi}děravý,

	/: jen kousek ^{Emi}jít, jen ^{A7}chvíli, ^{Hmi}do ^{Emi}soumraku, 

	až tam, kde ^{Hmi}svítí město, ^{F#}město ^{Hmi}bělavý. :/^{A7}

\sloka
	Vítr si tiše hvízdá po silnici spálený, 

	v tom městě nikdo nezdraví,  

	šerif i soudce -- gangsteři, voba řádně zvolení

	a lidi strachem nezdraví. 

\sloka
	Sto cizejch zabíječů s pistolema skotačí 

	a zákon džungle panuje, 

	provazník plete smyčky, hrobař kopat nestačí 

	a truhlář rakve hobluje. 

\refren[2]
	^*{D7\z}V městě ^{G}je řád a ^{A}pro každého ^{Hmi}práce, 

	 buď ještě ^{G}rád, když ^{A}huba voněmí,^{Hmi}  
 
	/: může tě ^{Emi}hřát, že ^{A7}nejsi ^{Hmi}na voprátce ^{Emi}  

	nebo že ^{Hmi}neležíš pár ^{F#}inchů pod ^{Hmi}zemí. :/^{A7}  

\sloka = 1.

\refren[3]
	Pryč odtud jít s tou karavanou mraků, 

	kde tichej dům a pušky rezaví,  

	/: orat a sít od rána do soumraku  

	a nechat zapomenout srdce bolavý. :/  

\end{centerjustified}
\setcounter{Slokočet}{0}
\end{song}
