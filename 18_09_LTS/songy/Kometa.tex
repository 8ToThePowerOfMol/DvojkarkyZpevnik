\begin{song}{title=\centering Kometa \\\normalsize Jaromír Nohavica \vspace*{-0.3cm}}  %% sem se napíše jméno songu a autor
\moveright 1cm \vbox{      %Varianta č. 1  ---> Jeden sloupec zarovnaný na střed	
\begin{minipage}[t]{0.48\textwidth}\setlength{\parindent}{0.45cm}  %Varianta č. 2 --> Dva sloupce

\sloka 
	^{Ami{\color{white}\_}}Spatřil jsem kometu, oblohou letěla,
	
	chtěl jsem jí zazpívat, ona mi zmizela.
	
	^{Dmi{\color{white}\_}}Zmizela jako laň ^{G7}u lesa v remízku,
	
	^{C}v očích mi zbylo jen ^{E7}pár žlutých penízků.

\sloka
	Penízky ukryl jsem do hlíny pod dubem,
	
	až příště přiletí, my už tu nebudem.
	
	My už tu nebudem, ach, pýcho marnivá,
	
	spatřil jsem kometu, chtěl jsem jí 
	
	zazpívat.
	
\refren
	^{Ami}O vodě, o trávě, ^{Dmi}o lese,
	
	^{G7}o smrti, se kterou smířit ^{C\,\,\,\,\,\,}nejde se,
	
	^{Ami}o lásce, o zradě, ^{Dmi}o světě
	
	^{E}a o všech lidech, co kdy ^{E7}žili na téhle 
	
	 ^{Ami\,\,\,\,\,\,}planetě.
	
\sloka	
	Na hvězdném nádraží cinkají vagóny,
	
	pan Kepler rozepsal nebeské zákony,
	
	hledal, až nalezl v hvězdářských triedrech
	
	tajemství, která teď neseme na bedrech.

\sloka
	Velká a odvěká tajemství přírody,
	
	že jenom z člověka člověk se narodí,
	
	že kořen s větvemi ve strom se spojuje
	
	a krev našich nadějí vesmírem putuje.

\refren
Na na na\elipsa\dots

\sloka
	Spatřil jsem kometu, byla jak reliéf
	
	zpod rukou umělce, který už nežije,
	
	šplhal jsem do nebe, chtěl jsem ji osahat,
	
	marnost mne vysvlékla celého donaha.
	
\end{minipage}\begin{minipage}[t]{0.5\textwidth}\setlength{\parindent}{0.45cm}\vspace*{0.55cm}  % V případě varianty č.2 jde odsud text do pravé části

\sloka
	Jak socha Davida z bílého mramoru
	
	stál jsem a hleděl jsem, hleděl jsem nahoru.
	
	Až příště přiletí, ach, pýcho marnivá,
	
	my už tu nebudem, ale jiný jí zazpívá.


\refren
	O vodě, o trávě, o lese,
	
	o smrti, se kterou smířit nejde se,
	
	o lásce, o zradě, o světě,
	
	bude to písnička o nás a kometě\elipsa\dots

\end{minipage}
}
\setcounter{Slokočet}{0}
\end{song}


