%%%%%%%%%%%%%%%%%%%%%%%%%%%%%%%%%%%%%%%%%%%%%%%%%%%%%
%			ŠABLONA PÍSNIČEK v. 18.09               %
%%%%%%%%%%%%%%%%%%%%%%%%%%%%%%%%%%%%%%%%%%%%%%%%%%%%%
% Tento soubor slouží jako (naučná) šablona, pomocí 
% které lze vytvářet zdrojové soubory k jednotlivým 
% písním.
%%%%%%%%%%%%%%%%%%%%%%%%%%%%%%%%%%%%%%%%%%%%%%%%%%%%%
%			Jak psát soubory songů?                 %
%%%%%%%%%%%%%%%%%%%%%%%%%%%%%%%%%%%%%%%%%%%%%%%%%%%%%
%	1. Text písně se začíná psát na místě START 
%	   a končí na místě END. Zbylý text ignorujte.
%	2. Jak bude vypadat pdf písně zjistíte po tom, 
%	   co soubor zkompilujete pomocí souboru   
%      ../Generator/generator. 
%	3. Při psaní dodržujte následující TeX pravidla:
%	 a) Nový řádek napíšete pomocí dvou odsazení 
%	    tedy dvou enterů.
%	 b) Nová sloka se píší pomocí \sloka a odsazení.
%		Refrén se píše jako \refren, v případě více 
%		refrénů \refren[č. refrénu].
%	 c) Akordy se píšou tak, že napíšete před slovo,
%	    kde chcete mít akord (bez mezery):
%		^{AKORD1\,AKORD2...}.
%	4. Pokud chcete ušetřit tvůrcům práci, tak 
%	   si přečtěte další poučný soubor o typografii 
%	   ../../Typo_pravidla.txt.
%	5. Akordy stačí psát jen do první sloky, když 
%	   se nezmění -- kytaristé to zvládnou
%	7. Název písně pište na místo [NÁZEV] a autora 
%	   pište na místo [AUTOR] 
%	7. Jak psát věci na české klávesnici:
%	   \ = alt gr + q; [/] = alt gr f/g; 
%      {/} = alt gr + b/n; ^ = alt gr + 3 , cokoliv
%%%%%%%%%%%%%%%%%%%%%%%%%%%%%%%%%%%%%%%%%%%%%%%%%%%%%
%			Jak kompilovat jednotlivé písně?        %
%%%%%%%%%%%%%%%%%%%%%%%%%%%%%%%%%%%%%%%%%%%%%%%%%%%%%
%	1. Více návodu je k tomuto napsáno v souboru 
%      ../Generator/generator. 
%%%%%%%%%%%%%%%%%%%%%%%%%%%%%%%%%%%%%%%%%%%%%%%%%%%%%
%			Jak kompilovat celý zpěvník?			%
%%%%%%%%%%%%%%%%%%%%%%%%%%%%%%%%%%%%%%%%%%%%%%%%%%%%%
%	1. Více návodu je k tomuto napsáno v souboru
%	   ../Cely_zpevnik/zpevnik.tex.
%%%%%%%%%%%%%%%%%%%%%%%%%%%%%%%%%%%%%%%%%%%%%%%%%%%%%
\begin{song}{title=\predtitle \centering Lokomotiva \\\large Poletíme? }  %% sem se napíše jméno songu a autor

\vspace*{.5cm}

\begin{centerjustified}
\vetsi
\sloka
^{G\z}Pokaždé když tě vidím, ^{D\z}vím, že by to šlo

a ^{Emi\z}když jsem přemejšlel, co cítím, ^{C\z}tak mě napadlo

jestli ^{G\z}nechceš svýho osla vedle ^{D\z}mýho osla hnát,

jestli ^{Emi\z}nechceš se mnou tahat ze ^{\z C}země rezavej drát.

\refren
^{G\z}Jsi ^{\,\, D}Lokomotiva, ^{\z \z Emi}která se řítí ^{C\z}tmou,

^{G\z}jsi ^*{\z \,\,\,\,\, D}indiá ni, kteří ^{Emi\z}prérií ^{C}jedou,

^{G\z}jsi kulka ^{D\z}vystřelená ^{Emi\z}do~mojí ^{\z C}hlavy,

^{G\z}jsi ^*{\z D}prezide nt a já tvé ^{Emi\z}spojené ^*{\z C}státy .


\sloka
Přines jsem ti kytku, no co koukáš, to se má

je to koruna žvejkačkou ke špejli přilepená,

a dva kelímky vod jogurtu, co je mezi nima niť,

můžeme si takhle volat, když budeme chtít.

\refren

\sloka
Každej příběh má svůj konec, ale né ten náš,

nám to bude navždy dojit, všude kam se podíváš,

naše kachny budou zlato nosit a krmit se popcornem,

já to každej večer spláchnu půlnočním expresem.

\refren

\sloka
Dětem dáme jména Jessie, Jeddej, Jad a John,

ve stopadesáti letech ho budu mít stále jako zvon,

a ty nestratíš svoji krásu, stále štíhlá kolem pasu,

stále dokážeš mě chytit lasem a přitáhnout na terasu.


\refren

\refren

\end{centerjustified}
\setcounter{Slokočet}{0}
\end{song}
