\begin{song}{title=\predtitle\centering Milionář \\\large Jaromír Nohavica  \vspace*{-0.3cm}}  %% sem se napíše jméno songu a autor
\normalsize

\begin{centerjustified}
\begin{varwidth}[t]{0.48\textwidth}\setlength{\parindent}{\pindent}  %Varianta č. 2 --> Dva sloupce
\sloka
U nás ^{D}v domě ^{{\color{white}\_\_}A7}říkají mi Franta ^*{D}Ši ška,

bo už od pohledu ^ {G}ch ytry jsem jak ^{D{\color{white}\_\_}}liška. 

A dyž kery něco ^{Emi}neví 

nebo ^{A7}dyž je na co ^{D{\color{white}\_}}levy,  

tak de za mnu a ja ^{A7{\color{white}\_\_\_}}všecko najdu 

^{D}v knižkach. 

\sloka
Raz mi říkal jeden znamy dole v baře, 

že s tu hlavu moh bych do Milionaře. 

Čemu ne, říkám si brachu, 

šak má Železný dost prachu, 

no a Čechovi se podíváš do tvaře. 

\sloka
Dostal jsem se mezi partu uchazeču, 

nikdo nemá šajnu jak tam nervy teču. 

Všecko viděl jsem do hnědě,

tak jsem zmáčknul AbeCeDe,

no a už mě kruci ke stolečku vleču. 

\sloka
Čech to začal takym malym interviju, 

co pry robim, esi kuřim a co piju. 

Tak jsem řeknul, co jsem řeknul, 

on se evidentně leknul 

a už začly blikat světla ve studiu.

\sloka
To se přiznam nebylo mi vesele,

první otázka pry co je ukulele. 

Tož tak jsem radši hlavu sklonil, 

abych to všecko nezkonil, 

říkám chtěl bych se obratit na přitele.

\sloka
Lojza byl po hlasu silně nevyspaly, 

asi zase celu šichtu prochlastali.

Bylo slyšet jak tam dycha, 

ale třicet vteřin ticha, 

to je tak, dyž se vam kamarad navali.

\sloka
Moju staru zatím doma braly mory,

lidi ohryzavali televizory.



\end{varwidth}\mezisloupci\begin{varwidth}[t]{0.5\textwidth}\setlength{\parindent}{\pindent}
\vspace*{0.60cm}  % V případě varianty č.2 jde odsud text do pravé části

Tož padesat na padesat, 

ať vím esi su to ty bulharské hory.

\sloka
A už jasně na tym komputuře sviti, 

buďto je to za A vzacne lučni kviti, 

nebo za Be nastroj strunny, 

tu de kurňa o koruny 

a ja stejně jak na začatku jsem v řiti.

\sloka
Čech tam zatím maval tymi svymi čisly,

tak si řikam Franta napij sa a mysli. 

Jake tudy sakypaky, 

obratiš se na divaky,

šak tu zatím za ty prachy enem kysli. 

\sloka
Sam jsem byl zvědavy co publikum zvoli, 

bo aj v obecenstvu možu sedět voli. 

Devadesat procent za Be, 

ale to mi přišlo slabe, 

bo co není stopro to mě dycky skoli. 

\sloka
Ještě že jsem chlap co zboja neutika, 

říkam pane Čechu pujdem do rizika. 

Měl jsem v gaťach nadělano, 

ale Čech zakřičel ano, 

mate pravdu, je to nastroj hudebnika.

\sloka 
Lidi tleskali, bo uspech to byl plny, 

radosti zrobili dvě mexicke vlny. 

A ja co mam srdce skromne, 

jako všeci z Dolni Lomne, 

jsem byl spokojeny, bo sem ukol splnil. 

\sloka
Pane Čechu nerad přetahnul bych strunu, 

končim hru a beru tisicikorunu. 

Čech se jenom chytnul stolu, 

obočí mu spadlo dolu, 

no a už se modry ku podlaze sunul. 

\sloka
První třidu do Ostravy Intercity, 

v jidelňaku celu cestu valim kyty.

A ta stovka co mi zbude, 

to je přispěvek na chude, 

bo Ostrava je region razovity.


\end{varwidth}
\end{centerjustified}
\setcounter{Slokočet}{0}
\end{song}
