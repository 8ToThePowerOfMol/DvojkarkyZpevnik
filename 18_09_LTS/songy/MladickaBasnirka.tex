%%%%%%%%%%%%%%%%%%%%%%%%%%%%%%%%%%%%%%%%%%%%%%%%%%%%%
%			ŠABLONA PÍSNIČEK v. 18.09               %
%%%%%%%%%%%%%%%%%%%%%%%%%%%%%%%%%%%%%%%%%%%%%%%%%%%%%
% Tento soubor slouží jako (naučná) šablona, pomocí 
% které lze vytvářet zdrojové soubory k jednotlivým 
% písním.
%%%%%%%%%%%%%%%%%%%%%%%%%%%%%%%%%%%%%%%%%%%%%%%%%%%%%
%			Jak psát soubory songů?                 %
%%%%%%%%%%%%%%%%%%%%%%%%%%%%%%%%%%%%%%%%%%%%%%%%%%%%%
%	1. Text písně se začíná psát na místě START 
%	   a končí na místě END. Zbylý text ignorujte.
%	2. Jak bude vypadat pdf písně zjistíte po tom, 
%	   co soubor zkompilujete pomocí souboru   
%      ../Generator/generator. 
%	3. Při psaní dodržujte následující TeX pravidla:
%	 a) Nový řádek napíšete pomocí dvou odsazení 
%	    tedy dvou enterů.
%	 b) Nová sloka se píší pomocí \sloka a odsazení.
%		Refrén se píše jako \refren, v případě více 
%		refrénů \refren[č. refrénu].
%	 c) Akordy se píšou tak, že napíšete před slovo,
%	    kde chcete mít akord (bez mezery):
%		^{AKORD1\,AKORD2...}.
%	4. Pokud chcete ušetřit tvůrcům práci, tak 
%	   si přečtěte další poučný soubor o typografii 
%	   ../../Typo_pravidla.txt.
%	5. Akordy stačí psát jen do první sloky, když 
%	   se nezmění -- kytaristé to zvládnou
%	7. Název písně pište na místo [NÁZEV] a autora 
%	   pište na místo [AUTOR] 
%	7. Jak psát věci na české klávesnici:
%	   \ = alt gr + q; [/] = alt gr f/g; 
%      {/} = alt gr + b/n; ^ = alt gr + 3 , cokoliv
%%%%%%%%%%%%%%%%%%%%%%%%%%%%%%%%%%%%%%%%%%%%%%%%%%%%%
%			Jak kompilovat jednotlivé písně?        %
%%%%%%%%%%%%%%%%%%%%%%%%%%%%%%%%%%%%%%%%%%%%%%%%%%%%%
%	1. Více návodu je k tomuto napsáno v souboru 
%      ../Generator/generator. 
%%%%%%%%%%%%%%%%%%%%%%%%%%%%%%%%%%%%%%%%%%%%%%%%%%%%%
%			Jak kompilovat celý zpěvník?			%
%%%%%%%%%%%%%%%%%%%%%%%%%%%%%%%%%%%%%%%%%%%%%%%%%%%%%
%	1. Více návodu je k tomuto napsáno v souboru
%	   ../Cely_zpevnik/zpevnik.tex.
%%%%%%%%%%%%%%%%%%%%%%%%%%%%%%%%%%%%%%%%%%%%%%%%%%%%%
\begin{song}{title=\predtitle \centering Mladičká Básnířka \\\large Jaromír Nohavica (feat. Čechomor) }  %% sem se napíše jméno songu a autor

\vspace*{.5cm}

\begin{centerjustified}
\vetsi
\sloka
^*{\z A}Ml adičká ^*{E}bás nířka ^*{D}s~k orálky ^{E}nad kotníky ^{A E D E}

^*{A}bou chala ^{E}na dvířka ^{D\z}paláce ^{\,\,\,\,\,E\z  \,\,\,\,\,\,F#mi}poetiky,

s někým se ^{D\z A}vyspala, někomu ^{D\z A}nedala, láska jako ^{E\z}hobby,

^{A}pak o tom ^*{E}nap sala ^{D}blues ^*{E}na~čt yři doby. ^{A E D E}

\sloka
Svoje srdce skloňovala podle vzoru Ferlinghetti,

ve vzduchu nechávala viset vždy jen půlku věty,

plná tragiky, plná mystiky, plná splínu,

pak jí to otiskli v jednom magazínu.

\sloka
Bývala viděna v malém baru u rozhlasu,

od sebe kolena a cizí ruka kolem pasu,

trochu se napila, trochu se opila na účet redaktora

za týden nato byla hvězdou Mikrofóra.

\sloka
Pod paží nosila rozepsané rukopisy,

ráno se budila vedle záchodové mísy,

můzou políbená, životem potřísněná, plná zázraků

a pak ji vyhodili z gymplu i z baráku.

\sloka
Šly řeči okolím, že měla něco se esenbáky

Ať bylo cokoliv, přestala věřit na zázraky

Cítila u srdce, jak po ní přešla železná bota.

Tak o tom napsala sonet, a ten byl ze života.

\sloka
Pak jednou v pondělí přišla na koncert na koleje,

a hlasem pokorným prosila o text Darmoděje,

Péro vzala, pak se dala

tichounce do pláče,

/:A její slzy kapaly na její mrkváče. :/

\end{centerjustified}
\setcounter{Slokočet}{0}
\end{song}
