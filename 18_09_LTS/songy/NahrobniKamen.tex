%%%%%%%%%%%%%%%%%%%%%%%%%%%%%%%%%%%%%%%%%%%%%%%%%%%%%
%			ŠABLONA PÍSNIČEK v. 18.09               %
%%%%%%%%%%%%%%%%%%%%%%%%%%%%%%%%%%%%%%%%%%%%%%%%%%%%%
% Tento soubor slouží jako (naučná) šablona, pomocí 
% které lze vytvářet zdrojové soubory k jednotlivým 
% písním.
%%%%%%%%%%%%%%%%%%%%%%%%%%%%%%%%%%%%%%%%%%%%%%%%%%%%%
%			Jak psát soubory songů?                 %
%%%%%%%%%%%%%%%%%%%%%%%%%%%%%%%%%%%%%%%%%%%%%%%%%%%%%
%	1. Text písně se začíná psát na místě START 
%	   a končí na místě END. Zbylý text ignorujte.
%	2. Jak bude vypadat pdf písně zjistíte po tom, 
%	   co soubor zkompilujete pomocí souboru   
%      ../Generator/generator. 
%	3. Při psaní dodržujte následující TeX pravidla:
%	 a) Nový řádek napíšete pomocí dvou odsazení 
%	    tedy dvou enterů.
%	 b) Nová sloka se píší pomocí \sloka a odsazení.
%		Refrén se píše jako \refren, v případě více 
%		refrénů \refren[č. refrénu].
%	 c) Akordy se píšou tak, že napíšete před slovo,
%	    kde chcete mít akord (bez mezery):
%		^{AKORD1\,AKORD2...}.
%	4. Pokud chcete ušetřit tvůrcům práci, tak 
%	   si přečtěte další poučný soubor o typografii 
%	   ../../Typo_pravidla.txt.
%	5. Akordy stačí psát jen do první sloky, když 
%	   se nezmění -- kytaristé to zvládnou
%	7. Název písně pište na místo [NÁZEV] a autora 
%	   pište na místo [AUTOR] 
%	7. Jak psát věci na české klávesnici:
%	   \ = alt gr + q; [/] = alt gr f/g; 
%      {/} = alt gr + b/n; ^ = alt gr + 3 , cokoliv
%%%%%%%%%%%%%%%%%%%%%%%%%%%%%%%%%%%%%%%%%%%%%%%%%%%%%
%			Jak kompilovat jednotlivé písně?        %
%%%%%%%%%%%%%%%%%%%%%%%%%%%%%%%%%%%%%%%%%%%%%%%%%%%%%
%	1. Více návodu je k tomuto napsáno v souboru 
%      ../Generator/generator. 
%%%%%%%%%%%%%%%%%%%%%%%%%%%%%%%%%%%%%%%%%%%%%%%%%%%%%
%			Jak kompilovat celý zpěvník?			%
%%%%%%%%%%%%%%%%%%%%%%%%%%%%%%%%%%%%%%%%%%%%%%%%%%%%%
%	1. Více návodu je k tomuto napsáno v souboru
%	   ../Cely_zpevnik/zpevnik.tex.
%%%%%%%%%%%%%%%%%%%%%%%%%%%%%%%%%%%%%%%%%%%%%%%%%%%%%
\begin{song}{title=\predtitle \centering Náhrobní kámen \\\large Petr Novák }  %% sem se napíše jméno songu a autor

\vspace*{.5cm}

\begin{centerjustified}
\vetsi
\sloka
^{G\z}Když půjdeš ^{Emi\z}po~cestě, ^{G}kde růže ^{A\z}vadnou,

kde rostou ^{G\z}stromy ^{C\z}bez ^{\z G}listí,

tak dojdeš ^{Emi\z}na~místo, ^{G\z}kde tvý slzy ^{A\z}spadnou

na ^{\z G}hrob ^{C}co nikdo ^{\z G}nečistí.

\sloka
Jen starej rozbitej náhrobní kámen

řekne ti, kdo nemoh už dál.

Tak sepni ruce svý a zašeptej ámen,

ať jsi tulák nebo král.

\ssloka{Rec.:}
Dřív děvče chodilo s kyticí růží

rozdávat lidem štěstí a svůj smích.

Oči jí maloval sám Bůh černou tuší,

pod jejím krokem hned tál sníh.

\sloka
Všem lidem dávala náručí plnou,

sázela kytky podél cest.

Jednou však zmizela a jako když utne,

přestaly růže náhle kvést.

\sloka
Pak jsem jí uviděl, ubohou vílu,

na zvadlých květech věčně snít.

Všem lidem rozdala svou lásku a sílu,

že sama dál nemohla už žít.

\sloka
Tak jsem jí postavil náhrobní kámen

a čerstvé růže jsem tam dal.

Pak jsem se pomodlil a zašeptal ámen

a svoji píseň jsem jí hrál.

^{G \z C \z G}A-a-a-a-a-a-a.

\end{centerjustified}
\setcounter{Slokočet}{0}
\end{song}
