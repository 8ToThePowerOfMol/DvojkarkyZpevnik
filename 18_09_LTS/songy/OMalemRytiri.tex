%%%%%%%%%%%%%%%%%%%%%%%%%%%%%%%%%%%%%%%%%%%%%%%%%%%%%
%			ŠABLONA PÍSNIČEK v. 18.09               %
%%%%%%%%%%%%%%%%%%%%%%%%%%%%%%%%%%%%%%%%%%%%%%%%%%%%%
% Tento soubor slouží jako (naučná) šablona, pomocí 
% které lze vytvářet zdrojové soubory k jednotlivým 
% písním.
%%%%%%%%%%%%%%%%%%%%%%%%%%%%%%%%%%%%%%%%%%%%%%%%%%%%%
%			Jak psát soubory songů?                 %
%%%%%%%%%%%%%%%%%%%%%%%%%%%%%%%%%%%%%%%%%%%%%%%%%%%%%
%	1. Text písně se začíná psát na místě START 
%	   a končí na místě END. Zbylý text ignorujte.
%	2. Jak bude vypadat pdf písně zjistíte po tom, 
%	   co soubor zkompilujete pomocí souboru   
%      ../Generator/generator. 
%	3. Při psaní dodržujte následující TeX pravidla:
%	 a) Nový řádek napíšete pomocí dvou odsazení 
%	    tedy dvou enterů.
%	 b) Nová sloka se píší pomocí \sloka a odsazení.
%		Refrén se píše jako \refren, v případě více 
%		refrénů \refren[č. refrénu].
%	 c) Akordy se píšou tak, že napíšete před slovo,
%	    kde chcete mít akord (bez mezery):
%		^{AKORD1\,AKORD2...}.
%	4. Pokud chcete ušetřit tvůrcům práci, tak 
%	   si přečtěte další poučný soubor o typografii 
%	   ../../Typo_pravidla.txt.
%	5. Akordy stačí psát jen do první sloky, když 
%	   se nezmění -- kytaristé to zvládnou
%	7. Název písně pište na místo [NÁZEV] a autora 
%	   pište na místo [AUTOR] 
%	7. Jak psát věci na české klávesnici:
%	   \ = alt gr + q; [/] = alt gr f/g; 
%      {/} = alt gr + b/n; ^ = alt gr + 3 , cokoliv
%%%%%%%%%%%%%%%%%%%%%%%%%%%%%%%%%%%%%%%%%%%%%%%%%%%%%
%			Jak kompilovat jednotlivé písně?        %
%%%%%%%%%%%%%%%%%%%%%%%%%%%%%%%%%%%%%%%%%%%%%%%%%%%%%
%	1. Více návodu je k tomuto napsáno v souboru 
%      ../Generator/generator. 
%%%%%%%%%%%%%%%%%%%%%%%%%%%%%%%%%%%%%%%%%%%%%%%%%%%%%
%			Jak kompilovat celý zpěvník?			%
%%%%%%%%%%%%%%%%%%%%%%%%%%%%%%%%%%%%%%%%%%%%%%%%%%%%%
%	1. Více návodu je k tomuto napsáno v souboru
%	   ../Cely_zpevnik/zpevnik.tex.
%%%%%%%%%%%%%%%%%%%%%%%%%%%%%%%%%%%%%%%%%%%%%%%%%%%%%
\begin{song}{title=\predtitle \centering O malém rytíři \\\large Traband }  %% sem se napíše jméno songu a autor

\vspace*{.5cm}

\begin{centerjustified}
\vetsi

\refren
^{Ami}Jede jede rytíř, ^{C}jede do kraje

^{G}Nové dobrodružství ^{Ami\z}v~dálce hledaje.

^{Ami}Neví, co je bázeň, ^{C}neví, co je strach.

^{G}Má jen velké srdce ^{Ami\z}a~na botách prach

\sloka
^{F}Jednou takhle v neděli ^{E}slunce pěkně hřálo,

^{F}bylo kolem poledne, ^{E}když tu se to stalo.

^{F}Panáček uhodí ^{G}pěstičkou do stolu:

^{F},,Dosti bylo pohodlí ^{G}a plnejch kastrólů!

^{Ami}Ještě dneska stůj co stůj ^{G}musím na cestu se dát,

^{F}tak zavolejte sloužící a ^{E}dejte koně osedlat!{``}

\refren

\slok
\uv{Ale milostpane!} spráskne ruce starý čeledín,

ale pán už sedí v sedle a volá s nadšením:

\uv{Má povinnost mi velí pomáhat potřebným,

ochraňovat chudé, slabé, léčit rány nemocným.}

Marně za ním volá stará hospodyně:

\uv{Vraťte se pane, lidi sou svině!}


\refren

\sloka
Ale sotva dojel kousek za městskou bránu,

z lesa na něj vyskočila banda trhanů.

Všichni ti chudí, slabí, potřební -- no chátra špinavá.

Vrhli se na něj a bili ho hlava nehlava.

Než se stačil vzpamatovat, bylo málem po něm,

ukradli mu, co kde měl, a sežrali mu koně.

\end{centerjustified}
\newpage
\begin{centerjustified}

\refren

\sloka
\uv{Vzhůru srdce!} zvolá rytíř, \uv{Nekončí má pouť,

svou čest a slávu dobudu, jen z cesty neuhnout!

Hle, můj meč!} (a zvedl ze země kus drátu)

\uv{A zde můj štít a přílbice!} (plechovka od špenátu)

Pak osedlal si pavouka, sed na něj, řekl: \uv{Hyjé!

Jedem vysvobodit princeznu z letargie.}


\refren

\sloka
A šíleně smutná princezna sotva ho viděla

vyprskla smíchy a plácla se do čela.

Začala se chechtat, až jí z očí tekly slzy

\uv{To je neskutečný,} volala, \uv{jak jsou dneska lidi drzý!

O mou ruku se chce ucházet tahle figůra.

Hej, zbrojnoši, ukažte mu rychle cestu ze dvora!}


\refren

\sloka
Tak jede malý rytíř svojí cestou dál.

Hlavu hrdě vzhůru -- on svou bitvu neprohrál,

i když král ho nechal vypráskat a drak mu sežral boty

a děvka z ulice mu plivla na kalhoty.

Ve světě, kde jsou lidi na lidi jak vlci

zůstává rytířem -- ve svém srdci.


\refren

\end{centerjustified}
\setcounter{Slokočet}{0}
\end{song}
