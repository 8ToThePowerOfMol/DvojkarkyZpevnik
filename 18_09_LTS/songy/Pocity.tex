%%%%%%%%%%%%%%%%%%%%%%%%%%%%%%%%%%%%%%%%%%%%%%%%%%%%%
%			ŠABLONA PÍSNIČEK v. 18.09               %
%%%%%%%%%%%%%%%%%%%%%%%%%%%%%%%%%%%%%%%%%%%%%%%%%%%%%
% Tento soubor slouží jako (naučná) šablona, pomocí 
% které lze vytvářet zdrojové soubory k jednotlivým 
% písním.
%%%%%%%%%%%%%%%%%%%%%%%%%%%%%%%%%%%%%%%%%%%%%%%%%%%%%
%			Jak psát soubory songů?                 %
%%%%%%%%%%%%%%%%%%%%%%%%%%%%%%%%%%%%%%%%%%%%%%%%%%%%%
%	1. Text písně se začíná psát na místě START 
%	   a končí na místě END. Zbylý text ignorujte.
%	2. Jak bude vypadat pdf písně zjistíte po tom, 
%	   co soubor zkompilujete pomocí souboru   
%      ../Generator/generator. 
%	3. Při psaní dodržujte následující TeX pravidla:
%	 a) Nový řádek napíšete pomocí dvou odsazení 
%	    tedy dvou enterů.
%	 b) Nová sloka se píší pomocí \sloka a odsazení.
%		Refrén se píše jako \refren, v případě více 
%		refrénů \refren[č. refrénu].
%	 c) Akordy se píšou tak, že napíšete před slovo,
%	    kde chcete mít akord (bez mezery):
%		^{AKORD1\,AKORD2...}.
%	4. Pokud chcete ušetřit tvůrcům práci, tak 
%	   si přečtěte další poučný soubor o typografii 
%	   ../../Typo_pravidla.txt.
%	5. Akordy stačí psát jen do první sloky, když 
%	   se nezmění -- kytaristé to zvládnou
%	7. Název písně pište na místo [NÁZEV] a autora 
%	   pište na místo [AUTOR] 
%	7. Jak psát věci na české klávesnici:
%	   \ = alt gr + q; [/] = alt gr f/g; 
%      {/} = alt gr + b/n; ^ = alt gr + 3 , cokoliv
%%%%%%%%%%%%%%%%%%%%%%%%%%%%%%%%%%%%%%%%%%%%%%%%%%%%%
%			Jak kompilovat jednotlivé písně?        %
%%%%%%%%%%%%%%%%%%%%%%%%%%%%%%%%%%%%%%%%%%%%%%%%%%%%%
%	1. Více návodu je k tomuto napsáno v souboru 
%      ../Generator/generator. 
%%%%%%%%%%%%%%%%%%%%%%%%%%%%%%%%%%%%%%%%%%%%%%%%%%%%%
%			Jak kompilovat celý zpěvník?			%
%%%%%%%%%%%%%%%%%%%%%%%%%%%%%%%%%%%%%%%%%%%%%%%%%%%%%
%	1. Více návodu je k tomuto napsáno v souboru
%	   ../Cely_zpevnik/zpevnik.tex.
%%%%%%%%%%%%%%%%%%%%%%%%%%%%%%%%%%%%%%%%%%%%%%%%%%%%%
\begin{song}{title=\predtitle \centering Pocity \\\large Tomáš Klus }  %% sem se napíše jméno songu a autor

\vspace*{.5cm}

\begin{centerjustified}
\vetsi
\sloka
^{G\z}Z~posledních ^{\z D}pocitů

^{Emi\z}poskládám ještě jednou ^{C\z}úžasnou chvíli.

Je to tím, že jsi tu,

možná tím, že kdysi jsme byly Ty a

já, my dva, dvě nahý těla.

Tak neříkej, že jinak si to chtěla,

tak neříkej, neříkej,

neříkej mi nic.

\sloka
Stala ses do noci

z ničeho nic moje platonická láska.

Unaven, bezmocný,

usínám vedle Tebe, něco ve mně praská.

\sloka
A ranní probuzení

a slova o štěstí, neboj se to nic není.

Pohled na okouzlení,

a prázdný náměstí na znamení.

\refren
^{G\z}Jenže Ty ^{\z D}neslyšíš,

jenže Ty ^{Emi \z C}neposloucháš,~~

snad ani ^{G\z}nevidíš,

^{D\z}nebo spíš ^{Emi \z}nechceš ^{\z C}vidět~~

a druhejm ^{G\z }závidíš

^{D\z}a~v očích ^{Emi\z}kapky slaný ^{\z C}vody.~~

Zkus změnu, ^{G\z D}uvidíš

a ^{Emi\z}vítej do ^{C\z}svobody.

\end{centerjustified}
\newpage
\begin{centerjustified}

\sloka
Jsi anděl, netušíš,

anděl co ze strachu mu utrhly křídla

a až to ucítíš, zkus kašlat na pravidla.

\sloka
Říkej si vo mě co chceš, já jsem byl vodjakživa blázen.

Nevim co nechápeš, ale vrať se na zem.

\refren

\sloka
Jsi anděl, netušíš,

anděl co ze strachu mu utrhly křídla

a až to ucítíš, zkus kašlat na pravidla.

\refren

\end{centerjustified}
\setcounter{Slokočet}{0}
\end{song}
