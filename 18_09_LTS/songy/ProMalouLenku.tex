\begin{song}{title=\predtitle\centering Pro Malou Lenku \\\large Jaromír Nohavica  \vspace*{-0.3cm}}  %% sem se napíše jméno songu a autor
\begin{centerjustified}

\sloka
^*{Ami\,\,\,}Jak\:m i tak docházejí ^{D7\z}síly,

já ^{G\,\,}pod jazykem ^{G/F#{\color{white}}}cítím ^{Emi{\color{white}\_}}síru,

^{Ami}víru, ztrácím ^*{D7}vír u a to mě ^*{G}mí chá,

^{Ami{\color{white}\_\_\_}}minomety reflektorů ^*{D7}stř ílí,

jsem malým ^{G{\color{white}\_\_\_\_}}terčem na bitevním ^{Emi{\color{white}\_}}poli,

kdekdo mě ^{Ami{\color{white}\_}}skolí a to mě ^{D7{\color{white}\_}}bolí, u srdce ^{G{\color{white}\_\_\_}}píchá.


\refren
^*{Ami}Rána\: jsou smutnější než ^*{D7}več er,

z rozbitého ^*{G}no su ^{G/F#{\color{white}}}krev mi ^{Emi{\color{white}\_}}teče,

na čísle ^{Ami}56 ^{D7\,}109 nikdo to ^*{G}ne bere,

^*{Ami}rána\: jsou smutnější než ^{D7{\color{white}\_\_}}večer,

na hrachu ^*{G}kl ečet, ^{G/F#{\color{white}}}klečet, ^{Emi{\color{white}\_}}klečet,

^*{Ami}opust il mě můj děd ^{{\color{white}\_\_}D7}Vševěd a zas je ^*{G}út erek.

^*{Ami}Jak\:s e ti ^{D7{\color{white}\_}}vede? No ^*{G}ně kdy fajn a ^{G/F}někdy je to v ^{Emi{\color{white}\_}}háji,

dva ^*{Ami}pozou nisti z vesnické ^*{D7}kut álky pod okny mi ^*{G}hr ají:

tú tú ^{Ami\phantom{d}}tú\elipsa.\elipsa.\elipsa. ^{D7\,\,G\,\,Emi\,\,Ami\,\,D7\,\,G\,\,Ami\,\,D7\,\,G\,\,Emi\,\,Ami\,\,D7\,\,G}


\sloka
Jak říká kamarád Pepa:

,,Co po mně chcete, slečno z první řady?

Vaše vnady mě nebaví a trošku baví,

sudička moje byla slepá,

když mi řekla to, co mi řekla,

píšou mi z pekla, že prý mě zdraví, že prý mě zdraví.``


\refren

\end{centerjustified}
\newpage
\begin{centerjustified}


\sloka
Má malá Lenko, co teď děláš,

chápej, že čtyři roky, to jsou čtyři roky

a čas pádí a já jsem tady a ty zase jinde,

až umřem, říkej, žes' nás měla,

to pro tebe skládáme tyhle sloky,

na hrachu klečíme a hloupě brečíme a světu dáváme kvinde.

\end{centerjustified}
\setcounter{Slokočet}{0}
\end{song}
