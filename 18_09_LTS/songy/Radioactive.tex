%%%%%%%%%%%%%%%%%%%%%%%%%%%%%%%%%%%%%%%%%%%%%%%%%%%%%
%			ŠABLONA PÍSNIČEK v. 18.09               %
%%%%%%%%%%%%%%%%%%%%%%%%%%%%%%%%%%%%%%%%%%%%%%%%%%%%%
% Tento soubor slouží jako (naučná) šablona, pomocí 
% které lze vytvářet zdrojové soubory k jednotlivým 
% písním.
%%%%%%%%%%%%%%%%%%%%%%%%%%%%%%%%%%%%%%%%%%%%%%%%%%%%%
%			Jak psát soubory songů?                 %
%%%%%%%%%%%%%%%%%%%%%%%%%%%%%%%%%%%%%%%%%%%%%%%%%%%%%
%	1. Text písně se začíná psát na místě START 
%	   a končí na místě END. Zbylý text ignorujte.
%	2. Jak bude vypadat pdf písně zjistíte po tom, 
%	   co soubor zkompilujete pomocí souboru   
%      ../Generator/generator. 
%	3. Při psaní dodržujte následující TeX pravidla:
%	 a) Nový řádek napíšete pomocí dvou odsazení 
%	    tedy dvou enterů.
%	 b) Nová sloka se píší pomocí \sloka a odsazení.
%		Refrén se píše jako \refren, v případě více 
%		refrénů \refren[č. refrénu].
%	 c) Akordy se píšou tak, že napíšete před slovo,
%	    kde chcete mít akord (bez mezery):
%		^{AKORD1\,AKORD2...}.
%	4. Pokud chcete ušetřit tvůrcům práci, tak 
%	   si přečtěte další poučný soubor o typografii 
%	   ../../Typo_pravidla.txt.
%	5. Akordy stačí psát jen do první sloky, když 
%	   se nezmění -- kytaristé to zvládnou
%	7. Název písně pište na místo [NÁZEV] a autora 
%	   pište na místo [AUTOR] 
%	7. Jak psát věci na české klávesnici:
%	   \ = alt gr + q; [/] = alt gr f/g; 
%      {/} = alt gr + b/n; ^ = alt gr + 3 , cokoliv
%%%%%%%%%%%%%%%%%%%%%%%%%%%%%%%%%%%%%%%%%%%%%%%%%%%%%
%			Jak kompilovat jednotlivé písně?        %
%%%%%%%%%%%%%%%%%%%%%%%%%%%%%%%%%%%%%%%%%%%%%%%%%%%%%
%	1. Více návodu je k tomuto napsáno v souboru 
%      ../Generator/generator. 
%%%%%%%%%%%%%%%%%%%%%%%%%%%%%%%%%%%%%%%%%%%%%%%%%%%%%
%			Jak kompilovat celý zpěvník?			%
%%%%%%%%%%%%%%%%%%%%%%%%%%%%%%%%%%%%%%%%%%%%%%%%%%%%%
%	1. Více návodu je k tomuto napsáno v souboru
%	   ../Cely_zpevnik/zpevnik.tex.
%%%%%%%%%%%%%%%%%%%%%%%%%%%%%%%%%%%%%%%%%%%%%%%%%%%%%
\begin{song}{title=\predtitle \centering Radioactive \\\large Imagine Dragons }  %% sem se napíše jméno songu a autor



\vspace*{.5cm}

\begin{centerjustified}

\capo{2}

\vetsi
\sloka
^{Ami\z}I'm~waking ^{\z C}up to ash and ^{G\z}dust

I wipe my ^{D\z}brow and I sweat my ^{Ami}rust

I'm ^*{\z C}breathing~i n the ^*{\z G}chemical s.  ^{D}

^{Ami}I'm breaking ^{C}in, shaping ^{G}up, then checking ^{D\z}out on the prison bus.

^{Ami\z}This is ^{C \,}it,~ the ^*{\z G}apocalyps e. \dots ^{D\z}Whoa.

\refren[1]
I'm waking up, ^{Ami \z}I~feel it ^{C}in my bones

^{G\z}Enough to make my ^{D\z}systems blow

/: Welcome to the new age, to the new age. :/

/: ^{Ami\z}Whoa, ^{C\z}whoa, I'm ^{G\z}radioactive, ^{D\z}radioactive :/

\sloka
^{Ami\z}I~raise my ^{C\z}flags, don my ^{G\z}clothes

It's a ^*{\z D}revolu tion, I suppose.   ^{Ami}

We're painted ^{C\z}red to fit right ^{G \,}in. \dots ^{D\z}Whoa.

I'm breaking in, shaping up, then checking out on the prison bus

This is it, the apocalypse. \dots Whoa.

\refren

^{Ami\z Cmaj7}All~systems~go, ^{G}sun hasn't ^*{\z D}die d

Deep in my bones, straight from inside.

\refren

\end{centerjustified}
\setcounter{Slokočet}{0}
\end{song}
