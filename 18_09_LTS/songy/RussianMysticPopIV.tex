\begin{song}{title=\predtitle\centering Russian mystic pop IV. \\\large Psí vojáci  \vspace*{-0.3cm}}  %% sem se napíše jméno songu a autor
\begin{centerjustified}
\nejnejvetsi

\sloka 
	^{Ami{\color{white}\_}}Chodím po tom městě a ^*{F\:}ne mám ani na tramvaj.
	
	^{G{\color{white}\_}}Není jaksi na místě, že ^{Ami}bych se z toho vyhrabal.
	
	^{Ami}Půjčil jsem si pětistovku a ^{F}od výčepu k báru
	
	^{G}jsem ji mezi prstama ^{{\color{white}\_}Ami}proměnil v mlžnou páru.
	
\refren
	/: Hej ^{Ami\phantom{dddd}}hej\elipsa\dots\,hej ^{F\phantom{dddd}}hej\elipsa.\elipsa.\elipsa.\,hej ^{G\phantom{dddddd}}hej\elipsa\dots\,jsem ^{{\color{white}\_\_}Ami\:\:}mladej.~:/
%Ok, ti phantomové jsou celkem prasácký, ale fungujou xd -- Jindra	
	
	 /: ^{Gmi{\color{white}\_}}Připadá mi to děsný, ale ^{D{\color{white}\_\_}}začíná mi bejt ^{F{\color{white}\_}}tohle město ^{C{\color{white}\_}}těsný. :/

	 Připadá mi to děsný, ale začíná mi bejt tohle město těsný.

\sloka
	Ani šanci nemám, že bych se ráno nasnídal,
	
	příteli se ozvu, na oběd se pozvu.
	
	Z hlediska věčnosti jsem plnej blbostí,
	
	subspecie ternitatis, holky, holky -- Dakar Paris.

\refren

\sloka
	Až večer budu usínat, schoulím se do sebe
	
	a budu vzpomínat, že měl jsem kdysi tebe.
	
	Nikdo už mi nezavolá, nikdo už mě nepohladí,
	
	všem lidem totiž moje bytost vadí.

\refren

\end{centerjustified}
\setcounter{Slokočet}{0}
\end{song}
