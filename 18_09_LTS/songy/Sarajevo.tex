\begin{song}{title=\predtitle\centering Sarajevo \\\large Jaromír Nohavica \vspace*{-0.3cm}}  %% sem se napíše jméno songu a autor
\begin{centerjustified}
\nejvetsi

\sloka 
	^{Emi}Přes haličské ^*{Ami/F#}pláně~vane~vítr~z lý,
	
	to ^{H7\z}málo, co jsme měli, nám ^{Emi\z}vody sebraly.
	
	Jako tažní ^{Ami/F# }ptáci,~jako~rorýsi
	
	^{H7\z}letíme nad zemí, dva ^{Emi\z}modré dopisy.

\refren
	^{Emi\z}Ještě hoří oheň a ^{Ami\z}praská dřevo,
	
	^{D7/F#\z}ale~už je čas jít ^{G\z}spát. ^{H7}

	^{Emi\z}Tamhle za kopcem je ^{Ami\z}Sarajevo,
	
	tam ^{H7\,\,\,\,\,\,\,\,\,\,}budeme se zítra ráno ^{Emi\,\,}brát.

\sloka
	Farář v kostele nás sváže navěky,
	
	věnec tamaryšku pak hodí do řeky,
	
	voda popluje zpátky do moře,
	
	my dva tady dole a nebe nahoře.

\refren

\sloka
	Postavím ti dům z bílého kamení,
	
	dubovými prkny on bude roubený,

	aby každý věděl, že jsem tě měl rád,
	
	postavím ho pevný, navěky bude stát.

\refren
	Ještě hoří oheň a praská dřevo,
	
	ale už je čas jít spát.
	
	Tamhle za kopcem je Sarajevo,
	
	tam zítra budeme se, lásko, brát\elipsa\dots

\end{centerjustified}
\setcounter{Slokočet}{0}
\end{song}
