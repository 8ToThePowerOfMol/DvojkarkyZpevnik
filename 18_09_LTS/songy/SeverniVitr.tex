%%%%%%%%%%%%%%%%%%%%%%%%%%%%%%%%%%%%%%%%%%%%%%%%%%%%%
%			ŠABLONA PÍSNIČEK v. 18.09               %
%%%%%%%%%%%%%%%%%%%%%%%%%%%%%%%%%%%%%%%%%%%%%%%%%%%%%
% Tento soubor slouží jako (naučná) šablona, pomocí 
% které lze vytvářet zdrojové soubory k jednotlivým 
% písním.
%%%%%%%%%%%%%%%%%%%%%%%%%%%%%%%%%%%%%%%%%%%%%%%%%%%%%
%			Jak psát soubory songů?                 %
%%%%%%%%%%%%%%%%%%%%%%%%%%%%%%%%%%%%%%%%%%%%%%%%%%%%%
%	1. Text písně se začíná psát na místě START 
%	   a končí na místě END. Zbylý text ignorujte.
%	2. Jak bude vypadat pdf písně zjistíte po tom, 
%	   co soubor zkompilujete pomocí souboru   
%      ../Generator/generator. 
%	3. Při psaní dodržujte následující TeX pravidla:
%	 a) Nový řádek napíšete pomocí dvou odsazení 
%	    tedy dvou enterů.
%	 b) Nová sloka se píší pomocí \sloka a odsazení.
%		Refrén se píše jako \refren, v případě více 
%		refrénů \refren[č. refrénu].
%	 c) Akordy se píšou tak, že napíšete před slovo,
%	    kde chcete mít akord (bez mezery):
%		^{AKORD1\,AKORD2...}.
%	4. Pokud chcete ušetřit tvůrcům práci, tak 
%	   si přečtěte další poučný soubor o typografii 
%	   ../../Typo_pravidla.txt.
%	5. Akordy stačí psát jen do první sloky, když 
%	   se nezmění -- kytaristé to zvládnou
%	7. Název písně pište na místo [NÁZEV] a autora 
%	   pište na místo [AUTOR] 
%	7. Jak psát věci na české klávesnici:
%	   \ = alt gr + q; [/] = alt gr f/g; 
%      {/} = alt gr + b/n; ^ = alt gr + 3 , cokoliv
%%%%%%%%%%%%%%%%%%%%%%%%%%%%%%%%%%%%%%%%%%%%%%%%%%%%%
%			Jak kompilovat jednotlivé písně?        %
%%%%%%%%%%%%%%%%%%%%%%%%%%%%%%%%%%%%%%%%%%%%%%%%%%%%%
%	1. Více návodu je k tomuto napsáno v souboru 
%      ../Generator/generator. 
%%%%%%%%%%%%%%%%%%%%%%%%%%%%%%%%%%%%%%%%%%%%%%%%%%%%%
%			Jak kompilovat celý zpěvník?			%
%%%%%%%%%%%%%%%%%%%%%%%%%%%%%%%%%%%%%%%%%%%%%%%%%%%%%
%	1. Více návodu je k tomuto napsáno v souboru
%	   ../Cely_zpevnik/zpevnik.tex.
%%%%%%%%%%%%%%%%%%%%%%%%%%%%%%%%%%%%%%%%%%%%%%%%%%%%%
\begin{song}{title=\predtitle \centering Severní vítr \\\large Zdeněk Svěrák \& Jaroslav Uhlíř }  %% sem se napíše jméno songu a autor

\vspace*{.5cm}

\begin{centerjustified}
\vetsi
\sloka
Jdu ^{C\z}s~děravou patou, mám ^{Ami\z }horečku zlatou,

jsem ^{F\z}chudý, jsem sláb, ^*{\z C}nemoc en.

Hlava mě pálí a ^{Ami\z}v~modravé dáli se ^{F\z}leskne

a ^{G7\z}třpytí můj ^{C\z}sen.

\sloka
Kraj pod sněhem mlčí, tam stopy jsou vlčí,

tam zbytečně budeš mi psát,

sám v dřevěné boudě sen o zlaté hroudě

já nechám si tisíckrát zdát.

\refren
^{C\z}Severní ^{C7\z}vítr je ^{F}krutý,

^{C\z}počítej, lásko má, ^{G7\z}s~tím,

^{C\z}k~nohám Ti ^{C7\z}dám zlaté ^{F\z}pruty

nebo se ^{C\z}vůbec ^{G7 \, C\z}nevrátím.

\sloka
Tak zarůstám vousem a vlci už jdou sem,

už slyším je výt blíž a blíž.

Už mají mou stopu, už větří, že kopu

svůj hrob a že stloukám si kříž.

\sloka
Zde leží ten blázen, chtěl dům a chtěl bazén

a opustil tvou krásnou tvář.

Má plechovej hrnek a pár zlatejch zrnek

a nad hrobem polární zář.

\refren

\end{centerjustified}
\setcounter{Slokočet}{0}
\end{song}
