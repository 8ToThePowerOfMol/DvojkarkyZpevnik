\begin{song}{title=\predtitle\centering Strom kýve pahýly \\\large Divadlo Sklep \vspace*{-0.3cm}}  %% sem se napíše jméno songu a autor
\begin{centerjustified}

\sloka
	^{A\z}Když slunce ^{Hmi7\z}zapadá, tak ^{A\z}moje ^{\z D}nálada ^{E7} ^*{\z A}klesá , ^{Hmi7\,\,A\,\,D\,\,E7}

	^{A\z}strom kýve ^{Hmi7\z}větvemi, ^{A}přítelem on je ^{D}mi, ^{Emi7} ^{\,\,\,\,\,A}plesá, ^{Hmi7\,\,A\,\,D\,\,E7}

	^{A}já však mám v duši žal, čert ^{Hmi7}ví, kde se tam vzal,

	^{\,\,A}tepe, ^{\z Hmi7}tepe, ^{\,\,A}tepe, ^{\,\,D}tepe. ^{E7}

\refren
	^{A\z}Strom kýve ^{Hmi7\,\,\,}pahýly, chtěl ^{A\z}bych jen na ^{\z D}chvíli tebe,

	^{A\z}strom kýve ^{Hmi7\,\,\,}pahýly, chtěl ^{A\z}bych jen na ^{\z D}chvíli tebe,

	^{A\z}rosu mám v ^{Hmi7\z}kanadách, v mých ^{A\z}černých ^{\z D}kanadách zebe,

	^{A\z}rosu mám v ^{Hmi7\z}kanadách, v mých ^{A\z}černých ^{\z D}kanadách zebe.


\sloka
	Znám dobře kůru lip, té dal jsem kdysi slib mlčení,

	znám řeč, jíž mluví hřib, sám jako jedna z ryb jsem němý,

	však lesy, ty mám rád, tam cítím se vždycky mlád,

	vždycky líp, vždycky líp, vždycky líp, vždycky líp.



\refren

\end{centerjustified}
\setcounter{Slokočet}{0}
\end{song}
