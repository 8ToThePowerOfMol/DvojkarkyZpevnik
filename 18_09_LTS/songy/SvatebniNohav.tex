\begin{song}{title=\predtitle\centering Svatební \\\large Jaromír Nohavica \vspace*{-0.3cm}}  %% sem se napíše jméno songu a autor
\begin{centerjustified}

\begin{varwidth}[t]{0.48\textwidth}\setlength{\parindent}{\pindent}  %Varianta č. 2 --> Dva sloupce

\sloka	
^{D \z}Barokní ^{A \z}varhaník ^{G \z}navlík si ^{A \z}paruku

 ^{D}a pudrem ^{A \z}přemázl ^{G}tvář ^{A}

^{D \z}Magda a ^{A}Jan se drží ^{G \z}za~ruku ^{A}

^{D}a ^{\z A}kráčejí před ^{G \z}oltář. ^{A}


\refren
^{D}Hou, ^{A}hou, ^{G}hou ^{A \z}zvony bijí

^{D}hou, ^{A}hou, ^{G}hou a já v ^{A \z}sakristii

^{D}hou, ^{A}hou, ^{G}hou ^{A \z}tajně schován

^{D}hou, ^{A}hou, ^{G}hou ^{A \z D}zamilován.


\sloka
Tři krásní velbloudi - dar krále Hasana

frkají před kostelem svatého Matěje

bílý je pro Magdu hnědý je pro Jana

ten třetí černý vzadu pro mě je


\refren
Hou, hou, hou už jsou svoji

hou, hou, hou a já v černém chvojí

hou, hou, hou tajně schován

hou, hou, hou zamilován


\sloka
Na staré pramici po řece Moravě

připlouvá kmotr Jura

fidlá na housle klobouk má na hlavě

a všichni křičí Hurá! Hurá!

\end{varwidth}\mezisloupci\begin{varwidth}[t]{0.48\textwidth}\setlength{\parindent}{\pindent}
\vspace*{0.405cm}  % V případě varianty č.2 jde odsud text do pravé části

\refren
Hou, hou, hou už jdou spolem

hou, hou, hou a já za topolem

hou, hou, hou tajně schován

hou, hou, hou zamilován


\sloka
Ech lásko bože lásko zanechala si mě

a to sa nedělá

srdce ti vyryju na futra předsíně

abys nezapomněla


\refren
Hou, hou, hou že byl jsem tady

hou, hou, hou umřel hlady

hou, hou, hou tajně schován

hou, hou, hou zamilován,

tvůj zamilován, tvůj zamilován.

\end{varwidth}

\end{centerjustified}
\setcounter{Slokočet}{0}
\end{song}
