\begin{song}{title=\predtitle\centering Vlajky vlají \\\large Mig 21  \vspace*{-0.3cm}}  %% sem se napíše jméno songu a autor
\begin{centerjustified}
\nejvetsi

\begin{varwidth}[t]{0.48\textwidth}\setlength{\parindent}{\pindent}  %Varianta č. 2 --> Dva sloupce

\sloka 
  Dobíhám ^{Dmi\z}tramvaj
  
  \phantom{.}

  z kopce na Petřín

  a ty jedeš ^{Ami\:\z}v~ní,

  v ^{\z B}tramvaji.

  Koukneš se ^{E}ven,

  ^{C\z }trolej ^{\z Dmi}zajiskří.

\sloka
  Dobíhám tramvaj

  z kopce na Petřín

  a ty jedeš v ní,

  v tramvaji.

  Ta mění směr,

  kvér vystřelí.

\refren
  /: ^{Dmi\z }Na~tramvaji vlajky vlají.


  ^{F}Mír volají, mír volají. :/

\sloka
  Zaklínám tramvaj,

  ať ještě zastaví

  a ty vystoupíš

  na chodník

  a pro šeřík 

  si chtít budeš jít.

\sloka
  Zaklínám tramvaj,

  ať ještě zastaví

  a ty vystoupíš

  na chodník.

  Ozve se křik, 

  jiskry padají.

\end{varwidth}\mezisloupci\begin{varwidth}[t]{0.48\textwidth}\setlength{\parindent}{\pindent}
\vspace*{0.405cm}  % V případě varianty č.2 jde odsud text do pravé části
\refren  

  /: ^{Dmi}Mír ^{C\,\,F}volají. :/


\sloka
  Začíná jaro

  nad Prahou, nad Plzní.

  Válka odezní,

  vyšumí.

  Nastane mír,

  osvobození.

\sloka
  Chtěl jsem jet s tebou

  noční tramvají -- ale 

  padám do křoví

  na šeřík.

  A v tom ten keř 

  tebou zavoní.

\refren 

  /: ^{Dmi}Mír ^{C\,\,F}volají. :/
  
  /: ^{Dmi}Mír ^{C\,\,F}volají. :/


\end{varwidth}

\end{centerjustified}
\setcounter{Slokočet}{0}
\end{song}
