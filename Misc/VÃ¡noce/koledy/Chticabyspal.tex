\begin{song}{title=\predtitle\centering Chtíc\carka aby spal \\\large Adam Michna z Otradovic   \vspace*{-0.3cm}}  %% sem se napíše jméno songu a autor
\begin{centerjustified}
\nejnejvetsi

\sloka 
	^{D{\color{white}aa}}Chtíc, ^{D7}aby ^{G}spal, tak ^{D7\,G{\color{white}aa}}zpívala, ^{Ami\,\,\,D7\,G}synáčkovi,

	^{D{\color{white}aa}}matka, ^{D7}jež ^{G{\color{white}aa}D7\,\,G}ponocovala, ^{Ami\,\,\,D7\,G}miláčkovi:

	Spi, ^{D{\color{white}aa}}nebes ^{G\,D\,}dítě ^{G\,Emi\,D\,\,}milostné, ^{Emi}Pán jsi ^{A7}a ^{D\,\,}Bůh,

	^{G{\color{white}aa}}přeje ^{D}ti v ^{G\,D}lásce ^{C{\color{white}a}}celý ^{D}ráj, ^{Ami\,\,\,D7\,}pozemský ^{G}luh.

\sloka
	Dřímej, to matky žádost je, holubičko,
   
   	v tobě se duše raduje, ó, perličko!
   	
   	Nebesa chválu pějí tvou, slávu a čest,
	
	velebí tebe každý tvor, tisíce hvěz.

\sloka	
	Ó lilie, ó fialko, ó růže má,
	
	dřímej, má sladká útěcho, zahrádko má!
	
	Labuti má a loutno má, slavíčku můj,

	dřímej, má harfo líbezná, synáčku můj!

\sloka	
	Miláčku, spi a zmlkněte andělové,
   	
   	před Bohem se mnou klekněte, národové!
	
	Sestoupil v pravdě boží syn na naši zem,
	
	přinesl spásu, pokoj svůj národům všem.




\end{centerjustified}
\setcounter{Slokočet}{0}
\end{song}

