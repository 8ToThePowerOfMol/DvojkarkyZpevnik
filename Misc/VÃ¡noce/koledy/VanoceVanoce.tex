\begin{song}{title=\predtitle\centering Vánoce\carka Vánoce přicházejí \\\large   \vspace*{-0.3cm}}  %% sem se napíše jméno songu a autor
\begin{centerjustified}
\nejvetsi

\refren
	^{C{\color{white}aaaa}}Vánoce, vánoce ^{G7{\color{white}aaaaa}}přicházejí, 

	zpívejme ^{{\color{white}aaa}C}přátelé,

	po roce vánoce, vánoce ^{G7{\color{white}aaaaa}}přicházejí,

	šťastné a ^{{\color{white}aaa}C}veselé.

\sloka
	^{G7}Proč jen děda říct si nedá, ^{D7{\color{white}a}}tluče o stůl v ^{G{\color{white}aaaa}}předsíni

	a pak, běda, marně hledá ^{D7{\color{white}a}}kapra pod ^{{\color{white}aa}G}skříní.

	Naše teta peče léta ^{D7}na vánoce ^{G{\color{white}aaaaa}}vánočku,

	nereptáme, aspoň máme ^{D7{\color{white}a}}něco pro ^{{\color{white}aa}G}kočku.  ^{G7}Jó!

\refren

\sloka
	Bez prskavek, tvrdil Slávek, na Štědrý den nelze být

	a pak táta s minimaxem zavlažoval byt.

	Tyhle ryby neměly by maso míti samou kost,

	říká táta vždy, když chvátá na pohotovost.


\refren

\sloka
	Jednou v roce na vánoce strejda housle popadne,

	jeho vinou se z nich linou tóny záhadné.

	Strejdu vida děda přidá \uv{Neseme vám noviny}

	čímž prakticky zničí vždycky večer rodinný.


\refren

\sloka
	A když sní se, co je v míse, televizor pustíme,

	v jizbě dusné všechno usne v blaženosti své.

	Mně se taky klíží zraky, bylo toho trochu moc,

	máme na rok na klid nárok, zas až do Vánoc.

\refren



\end{centerjustified}
\setcounter{Slokočet}{0}
\end{song}

