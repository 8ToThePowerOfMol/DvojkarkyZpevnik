\begin{song}{title=\centering Nesem vám noviny \\\normalsize   \vspace*{-0.3cm}}  %% sem se napíše jméno songu a autor
\moveright 4.5cm \vbox{      %Varianta č. 1  ---> Jeden sloupec zarovnaný na střed	

\sloka 
	^{D}Nesem ^{G}vám ^{D\,G}noviny, ^{D\,A\,D}poslouchejte,

	^{D}z ^{G}betlémské ^{D\,G}krajiny, ^{D}pozor ^{A\,D}dejte.

	/: ^{D}Slyšte je pilně ^{E7}a ^{A7}neomylně :/, ^{D\,7\,d}rozjímejte. 

\sloka
	Syna porodila čistá Panna,
	
	v jesličky vložila Krista Pána.
	
	/: Jej ovinula a zavinula :/ plenčičkama. 

\sloka
	K němužto andělé z nebe přišli,
   
   	i také pastýři jsou se sešli.
  	
  	/: Jeho vítali, jeho chválili, :/ dary nesli. 

\sloka
	Anděl Páně jim to sám přikázal
   	
   	když se jim na poušti všem ukázal,
	
	/: k Betlému jíti, neprodlévati :/ hned rozkázal. 

\sloka	
	Ejhle, při Kristovu narození
	
	stal se div veliký v okamžení,
  	
  	/: neboť noc tmavá se proměnila :/ v světlo denní. 

\sloka
	Andělé v oblacích prozpěvují,
   	
   	narození Páně ohlašují,
  	
  	/: že jest narozen, v jeslích položen, :/ oznamují. 

\sloka
	My také, křesťané, nemeškejme,

	k těm svatým jesličkám pospíchejme,
  	
  	/: Ježíše svého, Pána našeho :/ přivítejme. 

\sloka
	Vítej nám, Ježíšku, z nebe daný,
   	
   	který se narodil z čisté Panny,
  	
  	/: pohlédni na nás a přijmi od nás :/ tyto dary. 

\sloka
	Z nebe jsi sestoupil z pouhé lásky,
   	
   	krásné Jezulátko, kvítku rajský,
  	
  	/: jak jsi spanilý a ušlechtilý, :/ celý krásný.

}
\setcounter{Slokočet}{0}
\end{song}
