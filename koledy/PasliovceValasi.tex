\begin{song}{title=\predtitle\centering Pásli ovce Valaši \\\large   \vspace*{-0.3cm}}  %% sem se napíše jméno songu a autor
\begin{centerjustified}
\nejvetsi

\begin{varwidth}[t]{0.57\textwidth}\setlength{\parindent}{0.45cm}  %Varianta č. 2 --> Dva sloupce

\sloka
	^{C{\color{white}aa}}Pásli ^{G7}ovce ^{C{\color{white}aa}}Valaši,

	při ^{G7}betlémském ^{C{\color{white}a}}salaši.

\refren
	^{C{\color{white}aaa}}Hajdom hajdom ^{G7{\color{white}aa}\,C\,{\color{white}a}}tydlidom,

	hajdom hajdom ^{G7{\color{white}aa}\,C\,{\color{white}a}}tydlidom.

\sloka
	Anděl se jim ukázal,

	do Betléma jít kázal.

\refren

\sloka
	Jděte, jděte, pospěšte,

	Ježíška tam najdete.

\refren

\sloka
	On tam leží v jesličkách,

	zavinutý v pleničkách.

\refren

\sloka
	Anděl zpívá písničku,

	pozdravuje matičku.

\refren


\sloka
	Zdrávas, Panno Maria,

	Matko Boží spanilá!.

\end{varwidth}\mezisloupci\begin{varwidth}[t]{0.55\textwidth}\setlength{\parindent}{0.45cm}
\vspace*{0.4525cm}  % V případě varianty č.2 jde odsud text do pravé části

\refren

\sloka
	Maria se starala,

	kde by plének nabrala.

\refren

\sloka
	Utrhneme z růže květ,

	obvineme celý svět.

\refren

\end{varwidth}

\end{centerjustified}

\setcounter{Slokočet}{0}
\end{song}

