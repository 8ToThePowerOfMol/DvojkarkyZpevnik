\begin{song}{title=\centering Rolničky \\\normalsize   \vspace*{-0.3cm}}  %% sem se napíše jméno songu a autor
\moveright \stred \vbox{      %Varianta č. 1  ---> Jeden sloupec zarovnaný na střed	

\sloka 
	^{D}Sláva, už je sníh, jedem ^{F\# 7}na ^{G\,H7}saních,

	^{Emi}kluci ^{Gmi}křičí, ^{D}zvonek ^{Ami}zní, ^{Emi}jenom ^{A7}táta ^{D}ztich.

	^{D}Kouká na syna, uši ^{F\# 7\,G}napíná ^{H7}...

	^{Emi}Co to ^{Gmi}slyší ^{D}v ^{Ami}rolničkách? ^{Emi}Na co ^{A7\,\,D}vzpomíná?
    
\refren
	Rolničky, rolničky, kdopak ^{G}vám dal ^{D}hlas?

	^{G}Kašpárek ^{D}maličký ^{E}nebo děda ^{A7}Mráz?

	^{D}Rolničky, rolničky, co to ^{G}zvoní v ^{D}nich?

	^{G}Maminčiny ^{D}písničky, ^{Emi\,A7}vánoce a ^{D}sníh.

\sloka
	Zvonky dětských let, rozezvoňte svět!

	Těm, co už jsou dospělí, ať je znova pět!
	
	Zvoňte zlehýnka, stačí chvilinka:
	
	Vzpomínka jak rolničky v srdci zacinká.
	
\sloka
	Jingle bells, jingle bells,
	
	Jingle all the way.
	
	Oh! what fun it is to ride
	
	in a one-horse open sleigh.
	

}
\setcounter{Slokočet}{0}
\end{song}

