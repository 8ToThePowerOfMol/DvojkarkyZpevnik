\begin{song}{title=\centering Tichá noc \\\normalsize Franz Xaver Gruber \vspace*{-0.3cm}}  %% sem se napíše jméno songu a autor
\moveright 6cm \vbox{      %Varianta č. 1  ---> Jeden sloupec zarovnaný na střed	

\sloka 
	^{A{\color{white}aaa}}Tichá noc, svatá noc
	
	^{E{\color{white}a}}jala ^{E7}lid v ^{A{\color{white}aa}}blahý ^{A7}klid.
	
	^{D\,}Dvé jen srdcí tu ^{A}v Betlémě bdí,

	^{D{\color{white}aaa}}hvězdy při svitu ^{A}u jeslí dlí,
	
	^{E}v nichž malé ^{E7{\color{white}a}}děťátko ^{A\,}spí, ^{A7}
	
	^{A}v nichž malé ^{E{\color{white}aaaa}}děťátko ^{A}spí.

\sloka
	Tichá noc, svatá noc!
	
	Co anděl vyprávěl,
	
	prišed s jasností v pastýřův stan,
	
	zní již z výsosti, z všech země stran:
	
	\uv{Vám je dnes spasitel dán;
	
	přišel Kristus pán!}


\sloka	
	Tichá noc, svatá noc!
	
	Ježíšku na líčku
	
	boží láska si s usměvem hrá,
	
	zpod zlaté řasy k nám vyzírá,
	
	že nám až srdečko plá,
	
	vstříc mu vděčně plá.

}
\setcounter{Slokočet}{0}
\end{song}

