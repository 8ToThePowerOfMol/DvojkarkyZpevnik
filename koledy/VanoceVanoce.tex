\begin{song}{title=\centering Vánoce\carka Vánoce přicházejí \\\normalsize   \vspace*{-0.3cm}}  %% sem se napíše jméno songu a autor
\moveright 4cm \vbox{      %Varianta č. 1  ---> Jeden sloupec zarovnaný na střed	

\refren
	^{C}Vánoce, vánoce ^{G7}přicházejí, 

	zpívejme ^{C}přátelé,

	po roce vánoce, vánoce ^{G7}přicházejí,

	šťastné a ^{C}veselé.

\sloka
	^{G7}Proč jen děda říct si nedá, ^{D7}tluče o stůl v ^{G}předsíni

	a pak, běda, marně hledá ^{D7}kapra pod ^{G}skříní.

	Naše teta peče léta ^{D7}na vánoce ^{G}vánočku,

	nereptáme, aspoň máme ^{D7}něco pro ^{G}kočku.  ^{G7}Jó!
      
\refren
   
\sloka
	Bez prskavek, tvrdil Slávek, na Štědrý den nelze být

	a pak táta s minimaxem zavlažoval byt.

	Tyhle ryby neměly by maso míti samou kost,
	
	říká táta vždy, když chvátá na pohotovost.


\refren

\sloka
	Jednou v roce na vánoce strejda housle popadne,

	jeho vinou se z nich linou tóny záhadné.

	Strejdu vida děda přidá \uv{Neseme vám noviny}

	čímž prakticky zničí vždycky večer rodinný.


\refren

\sloka
	A když sní se, co je v míse, televizor pustíme,
	
	v jizbě dusné všechno usne v blaženosti své.
	
	Mně se taky klíží zraky, bylo toho trochu moc,
	
	máme na rok na klid nárok, zas až do Vánoc.

\refren



}
\setcounter{Slokočet}{0}
\end{song}

