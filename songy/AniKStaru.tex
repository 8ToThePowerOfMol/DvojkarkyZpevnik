\begin{song}{title=\predtitle \centering Ani k stáru \\\large Svěrák \& Uhlíř   \vspace*{-0.3cm}}  %% sem se napíše jméno songu a autor
\begin{centerjustified}
\velky

\begin{varwidth}[t]{0.48\textwidth}\setlength{\parindent}{0.15cm}  %Varianta č. 2 --> Dva sloupce

\sloka
    Mám ploché ^{C \z}nohy po tetě

    a ^{\z \, F}fantazii po svém strýci,

    už dlouho ^{Ami \z}šlapu po světě

    a nevím, ^{G}co mám o něm říci.

\sloka
    Kdysi jsem ^{C \z}sníval o Nilu

    a pak se ^{F \z}plavil po Vltavě,

    já věřil ^{Ami \z}spoustě omylů

    a dodnes ^{G \z}nemám jasno v hlavě,

    nemám jasno ^{C}v hlavě.

\refren
    Ani ^{C}k stáru, ani k stáru, ani k stáru

    nemám ^{Gmi \z}o~životě páru, nemám páru,

    ^{F \z}třebaže jsem ^{Emi \z}dosti ^{\z C}sečtělý, ^{\z G}sečtělý.

    Až mi ^{C \z}tváře zcela blednou, zcela blednou,

    dal bych ^{Gmi \z}si~ho ještě jednou, ještě jednou,

    třeba ^{F}s vámi, třeba ^{Emi \z}s~vámi, ^{\z C}chcete-li, ^{C7}

    třeba ^{F}s vámi, třeba ^{Emi \z}s~vámi, ^{\z C}chcete-li.

\end{varwidth}\mezisloupci \begin{varwidth}[t]{0.48\textwidth}\setlength{\parindent}{0.15cm}
\vspace*{0.165cm}  % V případě varianty č.2 jde odsud text do pravé části

\sloka
    Už dlouho šlapu po světě

    a čekám, co se ještě stane.

    Mám pořád duši dítěte

    a říkají mi starý pane.

\sloka
    Já citově jsem založen,

    při smutných filmech slzy roním

    a měl jsem taky málo žen,

    teď už to asi nedohoním,

    už to nedohoním.

\refren
\dots

	třeba ^{F}s vámi, třeba ^{Emi \z}s~vámi, ^{\z C}chcete-li, ^{C7}

	třeba ^{F}s vámi, milé ^{Emi \z}dámy, ^{\z C}chcete-li.

\end{varwidth}

\end{centerjustified}
\setcounter{Slokočet}{0}
\end{song}

