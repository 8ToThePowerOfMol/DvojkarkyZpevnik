\begin{song}{title=\predtitle \centering Bastila \\\large Znouzecnost \vspace*{-0.3cm}}  %% sem se napíše jméno songu a autor
\begin{centerjustified}
\vetsi

\sloka
^*{Dmi}Lásk a je nemoc ^{A#}a pravda přítěž,

^*{C}id eje maska ^{A}za kterou leccos zmizí.

Slova jsou náboje pro ústa velké ráže.

Peníze náboženství -- jak cesta tak i cíl.


\sloka
Srdce z olova, ruce od špíny,

hovna ze zlata zaplatí všechny viny.

Na cestě po žebříku vedoucím do nebe

dobré je dolů kopat a smát se nad sebe.

\refren
^*{Dmi}Bast ila ^*{A#}pad á {--} ^{C}s ní hlavy ^*{A}kr álů,

^{A#\z }z~chaosu  ^*{C}vz ejde nový ^{A\,\,}řád.

^{Dmi}Lůza se ^{A#\z }baví ^{C}a ^{A}nevědomky

^*{A#}zač íná ^*{C}st avět jiný ^*{Dmi}krim inál. ^{A#\,\,C\,\,A}


\sloka
Bůh zemřel mlčky první den války,

zbyli je šamani a dobře ví jak na to.

Nejvyšší umění -- stvoření krávy z davů

a ta se krmí gesty a dojí moc a slávu.


\sloka
\uv{Všichni jsme bratři -- ať žije svoboda!}

Jen jedna otázka -- co komu kolik za to?

\uv{Všichni jsme bratři -- ať žije svoboda!}

Občane okamžik -- co ty mně -- co já tobě?


\refren


\sloka
Zrozeni v kleci, v kleci i umřem

a klec všech klecí nosíme ve své hlavě.

Nejvyšší spravedlnost -- smutný pán na orloji.

Všichni jsme vyšli z prachu,

tam si budem zas rovni\elipsa\ldots


\refren


\end{centerjustified}
\setcounter{Slokočet}{0}
\end{song}
