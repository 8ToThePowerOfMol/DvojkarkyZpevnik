%\documentclass[../main.tex]{subfiles}

\begin{song}{title=\centering Černej Pasažér \\\normalsize Traband \vspace*{-0.3cm}}  %% sem se napíše jméno songu a autor
\moveright 1cm \vbox{      %Varianta č. 1  ---> Jeden sloupec zarovnaný na střed
\begin{minipage}[t]{0.48\textwidth}\setlength{\parindent}{0.25cm}  %Varianta č. 2 --> Dva sloupce
\setcounter{Slokočet}{0}
\sloka
Mám ^{Dmi}kufr plnej přebytečnejch ^{A}krámů 
                        
a mapu zabalenou do ^{Dmi}plátna 
                              
Můj vlak však jede na opačnou ^{A}stranu 
                               
a moje jízdenka je dávno ^{Dmi}neplatná 

\mezera
\textbf{F  Dmi F  Dmi }

\sloka
Někde ve vzpomínkách stojí dům 

Ještě vidím, jak se kouří z komína 

V tom domě prostřený stůl 

Tam já a moje rodina 

\sloka
Moje minulost se na mě šklebí 

a srdce bolí, když si vzpomenu 

že stromy, který měly dorůst k nebi 

teď leží vyvrácený z kořenů 

\mezera
\textbf{F  Dmi F  Dmi }

\refren
Jsem černej^{B} pasažér 

^{C}nemám ^{F}cíl ani směr 

Vezu se ^{B}načerno ^{C}životem a ^{F}nevím 
             
Jsem černej^{B} pasažér 

^{C}Nemám ^{F}cíl ani směr 

Vezu se ^{B}odnikud ^{C}nikam a ^{A7}nevím, kde skončím

\end{minipage}\begin{minipage}[t]{0.48\textwidth}\setlength{\parindent}{0.45cm}  % V případě varianty č.2 jde odsud text do pravé části
\vspace*{0.6cm}

\sloka
Mám to všechno na barevný fotce  

někdy z minulýho století 

Tu jedinou a pocit bezdomovce 

si nesu s sebou jako prokletí

\mezera
\textbf{F  Dmi F  Dmi }

\refren

\sloka
Mám kufr plnej přebytečnejch krámů 

a mapu zabalenou do plátna 

Můj vlak však jede na opačnou stranu 

a moje jízdenka je dávno neplatná

\refren

\end{minipage}   %Součást druhé varianty
}
\end{song}
\setcounter{Slokočet}{0}
