\begin{song}{title=\centering Cestou od hřbitova \\\normalsize Znouzecnost  \vspace*{-0.3cm}}  %% sem se napíše jméno songu a autor
\moveright 4.2cm \vbox{      %Varianta č. 1  ---> Jeden sloupec zarovnaný na střed	

\sloka
	^{G{\color{white}aaa}}Cestou od hřbitova ^{D{\color{white}aaa}}potkal jsem tu madam,
	
	^{F}slzy se jí pod závojem ^{G{\color{white}aa}}derou do očí,
	
	ňáká mladá vdova -- ^{D}tak upřímnou soustrast
	
	^{F{\color{white}aaaa}}popřeju jí na začátek ^{G}a už se to točí.
	
\refren 
	Laj laj \dots \textbf{G, D, F, G, D, F, G}

	/: ^{G{\color{white}aa}}Svět, svět se ^{D{\color{white}aaaa}}zatočí a ^{F{\color{white}aa}}život jde ^{G}dál. :/ %Zatočí nebo točí? Já bych řekl, že zatočí, Ano, ztoci

\sloka
	Proč ty slzy, madam, vždyť žijem tak krátce,
	
	žal ve vaší tváři k vašemu já neladí.
	
	Nabízím Vám rámě -- služebník a rádce
	
	a kašlete na ty řeči, že se to nehodí.

\refren

\sloka
	Budu Vám číst z dlaní, co vás ještě čeká,
	
	něžně a s pietou vaše slzy osuším.
	
	Cestou od hřbitova jen chviličku váhá
	
	a ten pod tou hlínou už stejně nic netuší.

\refren

\sloka
	Balím holky na hřbitově přes duchovní útěchu,
	
	no a potom cestou domů skončí u mě v pelechu.
	
	Jsem dle řečí na pavlači amorální dobytek.
	
	A můj názor na celou věc: Oboustrannej užitek!

\refren


}
\setcounter{Slokočet}{0}
\end{song}


