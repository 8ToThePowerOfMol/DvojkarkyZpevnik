\begin{song}{title=\centering Cestou od hřbitova \\\normalsize Znouzecnost  \vspace*{-0.3cm}}  %% sem se napíše jméno songu a autor
\moveright 4.8cm \vbox{      %Varianta č. 1  ---> Jeden sloupec zarovnaný na střed	

\sloka
	^{C}Cestou od hřbitova ^{G}potkal jsem tu madam,
	
	^{B}slzy se jí pod závojem ^{C}derou do očí,
	
	ňáká mladá vdova -- ^{G}tak upřímnou soustrast
	
	^{B}popřeju jí na začátek ^{C}a už se to točí.
	
\refren 
	Laj laj \dots \textbf{C, G, B, C, G, B, C}

	/: ^{C}Svět, svět ^{G}se zatočí a ^{B}život jde ^{C}dál. :/ %Zatočí nebo točí? Já bych řekl, že zatočí

\sloka
	Proč ty slzy, madam, vždyť žijem tak krátce,
	
	žal ve vaší tváři k vašemu já neladí.
	
	Nabízím vám rámě - služebník a rádce
	
	a kašlete na ty řeči, že se to nehodí.

\refren

\sloka
	Budu vám číst z dlaní, co vás ještě čeká,
	
	něžně a s pietou vaše slzy osuším.
	
	Cestou od hřbitova jen chviličku váhá
	
	a ten pod tou hlínou už stejně nic netuší.

\refren

\sloka
	Balím holky na hřbitově přes duchovní útěchu,
	
	no, a potom cestou domů skončí u mě v pelechu,
	
	jsem dle řečí na pavlači amorální dobytek
	
	a můj názor na celou věc: Oboustrannej užitek!

\refren


}
\setcounter{Slokočet}{0}
\end{song}


