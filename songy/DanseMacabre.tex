\begin{song}{title=\centering Danse Macabre \\\normalsize Jaromír Nohavica \vspace*{-0.3cm}}  %% sem se napíše jméno songu a autor
\moveright 4cm \vbox{      %Varianta č. 1  ---> Jeden sloupec zarovnaný na střed	


\refren 
	^{Dmi B Dmi B Dmi B Dmi B Dmi F A Dmi B F C F A}{Naj, naj naj naj, naj naj naj \dots \textcolor{white}{\hrulefill }} 
	%Geniální typografické řešení! % LIKE! To vypada elegantne. A.

\sloka
	^{Dmi}Šest milionů srdcí vyletělo ^{{\color{white}\_\_\_}B}komínem,

	^{Dmi}svoje malé lži si, lásko, dnes ^{{\color{white}\_\_\_\_}B}prominem,

	^{F{\color{white}\_\_\_\_}}budeme tančit s ^{A{\color{white}\_\_\_\_\_\_\_}}venkovany,

	na návsi ^{Dmi{\color{white}\_\_\_}}vesnice budeme se ^{B{\color{white}\_\_}}smát, ^{F\,\,C}
	
	mám tě ^{F\,\,\,\,}rád. ^{A}

\refren

\sloka
	Láska je nenávist a nenávist je láska,
	
	jedeme na veselku, kočí bičem práská,
	
	v červené sametové halence

	podobáš se Evě i Marii,
	
	dneska mě zabijí.
	
\refren

\sloka
	Mě děti pochopily, hledí na mě úkosem,

	třetí oko je prázdný prostor nad nosem,
	
	Pánbůh se klidně opil levným balkánským likérem

	a teď vyspává,
	
	jinak to smysl nedává.


}
\setcounter{Slokočet}{0}
\end{song}



