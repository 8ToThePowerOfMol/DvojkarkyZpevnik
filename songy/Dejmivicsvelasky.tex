\begin{song}{title=\centering Dej mi víc své lásky \\\normalsize Olympic  \vspace*{-0.3cm}}  %% sem se napíše jméno songu a autor
\moveright \stred \vbox{      %Varianta č. 1  ---> Jeden sloupec zarovnaný na střed	

\sloka
	^{Emi{\color{white}\_\_}}Vymyslel jsem spoustu napadů, ^{G}aů,
   
	co ^{Emi{\color{white}\_\_}}podporujou hloupou ^{{\color{white}\_}D}náladu, ^{H7}aů,

	^{Emi}hodit klíče do kanálu, ^{A}sjet po zadku ^{Ami}holou skálu,

	v ^{Emi}noci chodit ^{H7\,\,\,\,}strašit do ^{{\color{white}\_}Emi}hradu, aů.

\sloka
	Dám si dvoje housle pod bradu, aů,

	v bílé plachtě chodím pozadu, aů,
	
	úplně melancholicky, s citem pro věc jako vždycky

	vyrábím tu hradní záhadu, aů.

\refren
	^{G}Má drahá, dej mi víc, ^{H7}má drahá, dej mi víc,

	^{Emi}má drahá, ^{C}dej mi víc své ^{G{\color{white}\_\_}}lásky, ^{D7}aů,

	^{G}já nechci skoro nic, ^{H7}já nechci skoro nic,

	^{Emi}já chci jen ^{C{\color{white}\_\_\_}}pohladit tvé ^{G{\color{white}\_\_}}vlásky, ^{H7}aů.

\sloka
	Nejlepší z těch divnejch nápadů, aů,
	
	mi dokonale zvednul náladu, aů,

	natrhám ti sedmikrásky, tebe celou s tvými vlásky

	zamknu si na sedm západů, aů.

\refren

\sloka = 3.


}
\setcounter{Slokočet}{0}
\end{song}



