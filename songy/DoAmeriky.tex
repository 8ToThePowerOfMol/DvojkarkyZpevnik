%\documentclass[../main.tex]{subfiles}

\begin{song}{title=\centering Do Ameriky \vspace*{-0.3cm}}  %% sem se napíše jméno songu a autor
\moveright \stred \vbox{      %Varianta č. 1  ---> Jeden sloupec zarovnaný na střed
\setcounter{Slokočet}{0}
\sloka
^{C}Do Ameriky jezděj Parníky,

když tam ^{Ami{\color{white}\_}}přijdeš, ^{C{\color{white}\_}}zdá se ^{Ami}ti to ^{Dmi{\color{white}\_}}všecko velik^{G}ý,

^{\phantom{S}}je to fakticky hodně praktický  

^{G7}přijít tam a umět ^{{\color{white}\_}C}anglicky. 

\refren
^{C{\color{white}\_\_\_}}Dobrou noc -- good night, ^{D7{\color{white}\_\_\_}}výborně -- all right,

^{G{\color{white}\_\_}}Conrad Weidt ^{G7}-- už je off ^{C}side.

^{C7}His Master's ^{F{\color{white}\_\_}}Voice, Yankee Doo^{A7}dle,

máš ape^{D7}tit? Mám! Vem si ^{G7}štrůdl -- do pusy.

^{C}Sto kilo je cent, ^{D7}patent je patent,

^{G}Husa v troubě ^{G7}--  happy ^{C}end.

\refren

\sloka
Mám na západě chajdu v Nevadě,

zlatou žílu v Arizoně, doly v Kanadě,

babička na mě šetří v Panamě,

ačkoli tu žiju náramě!

\refren

\sloka
Co Buffalo Bill vrazil do kobyl,

za to by si bejval byl koupil automobil, 

cowboy na koni už se nehoní,

přestože mu Fordka nevoní.


}
\end{song}
\setcounter{Slokočet}{0}

