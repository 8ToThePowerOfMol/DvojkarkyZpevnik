%\documentclass[../main.tex]{subfiles}


\begin{song}{title=\centering Do dne a do roka \\ \normalsize Jaromír Nohavica \vspace*{-0.3cm}}  %% sem se napíše jméno songu a autor
\moveright 2cm \vbox{      %Varianta č. 1  ---> Jeden sloupec zarovnaný na střed
\begin{minipage}[t]{0.48\textwidth}\setlength{\parindent}{0.45cm}  %Varianta č. 2 --> Dva sloupce
\setcounter{Slokočet}{0}
%\centering
\sloka
^{Hmi C#7 F#m } 

Byla ^{Hmi}hluboká noc^{C#7} 

Venku ^{F#m}cizí pes vil 

A ja ^{Hmi}u okna stál^{C#7}  

A ^{F#m}pil 

\sloka
Zřel jsem ^{Hmi}jen jeho stín^{C#7}

Měl ^{F#m}rozplizlý tvar 

A ^{Hmi}vypadal jak^{C#7} 

Lomi^{F#m}kar 

\refren
Do dne a ^{Hmi}do roka, za zvuku ^{C#7}baroka, se rodí ^{F#m}rokoko 

Do noci ^{Hmi}hledíme, a vlastně ^{C#7}nevíme, 

zda je to ^{F#m}opravdu anebo jenom tak, naoko 

Do dne a ^{Hmi}do roka, za zvuku ^{C#7}rokoka, se rodí ^{F#m}secese 

Do noci ^{Hmi}hledíme, všichni tam ^{C#7}musíme, ale ^{F#m}nechce se 

\sloka
Chtěl jsem okřiknout jej, 

myslím psa v oné tmě, 

ale neměl jsem slov, jimiž to lze.

Vzal jsem do ruky kolt, 

jenž v mé komodě byl 

a na černý stín jsem namířil. 
\end{minipage}\begin{minipage}[t]{0.48\textwidth}\setlength{\parindent}{0.45cm}  % V případě varianty č.2 jde odsud text do pravé části
\sloka
Ruka chvěla se mi, 

neboť z krbu šel mráz, 

pak se na vteřinu zastavil čas. 

Tmě se zježila srst, 

já ucítil strach, 

kdo má na spoušti prst je vrah.

\refren
%\newpage
\sloka
Výstřel protrhl tmu, 

jako rybářům síť, 

jako sudičce řeč a niť. 

Té noci špatně jsem spal 

v záři voskových svic,

ráno tam, co byl plot, nebylo nic. 

\refren
\end{minipage}
}
\end{song}

