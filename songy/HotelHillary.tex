\begin{song}{title=\centering Hotel Hillary \\\normalsize Poutníci \vspace*{-0.3cm}}  %% sem se napíše jméno songu a autor
\moveright 3.8cm \vbox{      %Varianta č. 1  ---> Jeden sloupec zarovnaný na střed	

\sloka 
	Tvař se ^{Ami}trochu nostalgicky, už tě nikdy nepotkám,

	^{F}máš to jistý ^{G}provždycky, nastav ^{Ami}uši vzpomínkám,

	jak tě znám, i v tuhle chvíli měl bys řeči peprný,
    
	jak ^{F}tenkrát, když nám ^{G}tvrdili, že je ^{Ami}vítr stříbrný.


\refren
	A ^{F}tváře měli kožený, my jim zdrhli z průvodu,

	^{Dmi}zahodili lampióny a ^{D}našli hospodu,

	ale ^{F}taky Jacquese Brela a s ním smutek z cizích vin

	a ^{Dmi}žádostivost těla a pak ^{D}radost z volovin, a ta nám ^{Ami}zbejvá.

\sloka
	Po večerech pro diváky dělali jsme kašpary,
	
	pak na zemi dva spacáky - náš Hotel Hillary,
	
	slavný sliby jsme už znali, i to, jak se neplní,
	
	a cenzoři nám kázali vo správným umění.

\refren

\sloka	
	A tak válčím s nostalgií, bují ve mně jako mech,
	
	a pořád všechno slibují starý hesla na domech,
	
	ty jsi splatil všechny dluhy, i za Hotel Hillary,
	
	a já vyhážu ty černý stuhy funebrákům navzdory.

\refren
	Vždyť mají tváře kožený, my jim zdrhnem z průvodu,
	
	zahodíme lampióny a najdem hospodu,
	
	a tam tvýho Jacquese Brela a s ním smutek z cizích vin
	
	a žádostivost těla a pak radost z volovin, /: a ta nám zbejvá. :/



}
\setcounter{Slokočet}{0}
\end{song}
