\begin{song}{title=\predtitle \centering Hudsonský Šífy \large Wabi Daněk \vspace*{.5cm}}  %% sem se napíše jméno songu a autor



\begin{centerjustified}
\vetsi
\sloka  %1
Ten, kdo ^{Ami \z}nezná hukot vody lopat^{C \z}kama

vířený

jako ^{G}já, jó, ^{\z Ami}jako~já,

kdo hud^{Ami \z}sonský slapy nezná sírou ^{G \z}pekla

sířený,

ať se ^{Ami}na hudsonský ^{G \z}šífy najmout ^{Ami}dá,

^{G \z \z Ami}jo-ho-ho.

\sloka
Ten, kdo nepřekládál uhlí, šíf když

na mělčinu vjel,

málo zná, málo zná.

Ten, kdo neměl tělo ztuhlý, až se

nočním chladem chvěl,

ať se na hudsonský šífy najmout dá,

jo-ho-ho.

\refren
A^{F \z}hoj, páru tam ^{Ami \z}hoď,

ať ^{G}do pekla se dříve dohra^{Ami \z}bem,

^{G \z \z Ami}Jo-ho-ho, ^{G \z \z Ami}jo-ho-ho.

\sloka Ten, kdo nezná noční zpěvy

zarostenejch lodníků

jako já, jó, jako já,

ten, kdo cejtí se bejt chlapem,

umí dělat rotyku,

ať se na hudsonský šífy najmout dá,

jo-ho-ho.

\end{varwidth}\mezisloupci\begin{varwidth}[t]{0.5\textwidth}\setlength{\parindent}{0.45cm}  % V případě varianty č.2 jde odsud text do pravé části
\vspace*{0.375cm}

\sloka
Ten, kdo má na bradě mlíko, kdo

se rumem neopil,

málo zná, málo zná,

kdo necejtil hrůzu z vody, kde se

málem utopil,

ať se na hudsonský šífy najmout dá,

jo-ho-ho.

\refren

\sloka Kdo má roztrhaný boty, kdo má

pořád jenom hlad

jako já, jó, jako já,

kdo chce celý noci čuchat

pekelnýho vohně smrad,

ať se na hudsonský šífy najmout dá,

jo-ho-ho.

\sloka
Kdo chce zhebnout třeba zejtra,

komu je to všechno fuk,

kdo je sám, jó, jako já,

kdo má srdce v správným místě,

kdo je prostě príma kluk,

ať se na hudsonský šífy najmout dá,

jo-ho-ho.

\refren


\end{centerjustified}
\setcounter{Slokočet}{0}
\end{song}
