%%%%%%%%%%%%%%%%%%%%%%%%%%%%%%%%%%%%%%%%%%%%%%%%%%%%%
%			ŠABLONA PÍSNIČEK v. 18.09               %
%%%%%%%%%%%%%%%%%%%%%%%%%%%%%%%%%%%%%%%%%%%%%%%%%%%%%
% Tento soubor slouží jako (naučná) šablona, pomocí 
% které lze vytvářet zdrojové soubory k jednotlivým 
% písním.
%%%%%%%%%%%%%%%%%%%%%%%%%%%%%%%%%%%%%%%%%%%%%%%%%%%%%
%			Jak psát soubory songů?                 %
%%%%%%%%%%%%%%%%%%%%%%%%%%%%%%%%%%%%%%%%%%%%%%%%%%%%%
%	1. Text písně se začíná psát na místě START 
%	   a končí na místě END. Zbylý text ignorujte.
%	2. Jak bude vypadat pdf písně zjistíte po tom, 
%	   co soubor zkompilujete pomocí souboru   
%      ../Generator/generator. 
%	3. Při psaní dodržujte následující TeX pravidla:
%	 a) Nový řádek napíšete pomocí dvou odsazení 
%	    tedy dvou enterů.
%	 b) Nová sloka se píší pomocí \sloka a odsazení.
%		Refrén se píše jako \refren, v případě více 
%		refrénů \refren[č. refrénu].
%	 c) Akordy se píšou tak, že napíšete před slovo,
%	    kde chcete mít akord (bez mezery):
%		^{AKORD1\,AKORD2...}.
%	4. Pokud chcete ušetřit tvůrcům práci, tak 
%	   si přečtěte další poučný soubor o typografii 
%	   ../../Typo_pravidla.txt.
%	5. Akordy stačí psát jen do první sloky, když 
%	   se nezmění -- kytaristé to zvládnou
%	7. Název písně pište na místo [NÁZEV] a autora 
%	   pište na místo [AUTOR] 
%	7. Jak psát věci na české klávesnici:
%	   \ = alt gr + q; [/] = alt gr f/g; 
%      {/} = alt gr + b/n; ^ = alt gr + 3 , cokoliv
%%%%%%%%%%%%%%%%%%%%%%%%%%%%%%%%%%%%%%%%%%%%%%%%%%%%%
%			Jak kompilovat jednotlivé písně?        %
%%%%%%%%%%%%%%%%%%%%%%%%%%%%%%%%%%%%%%%%%%%%%%%%%%%%%
%	1. Více návodu je k tomuto napsáno v souboru 
%      ../Generator/generator. 
%%%%%%%%%%%%%%%%%%%%%%%%%%%%%%%%%%%%%%%%%%%%%%%%%%%%%
%			Jak kompilovat celý zpěvník?			%
%%%%%%%%%%%%%%%%%%%%%%%%%%%%%%%%%%%%%%%%%%%%%%%%%%%%%
%	1. Více návodu je k tomuto napsáno v souboru
%	   ../Cely_zpevnik/zpevnik.tex.
%%%%%%%%%%%%%%%%%%%%%%%%%%%%%%%%%%%%%%%%%%%%%%%%%%%%%
\begin{song}{title=\predtitle \centering Hudsonský Šífy \\\large Wabi Daněk }  %% sem se napíše jméno songu a autor

\vspace*{.5cm}

\begin{centerjustified}
\vetsi
\sloka  %1
Ten, kdo ^{Ami \z}nezná hukot vody lopat^{C \z}kama \\vířený

Jako ^{G}já, jó, ^{\z Ami}jako~já,

Kdo hud^{Ami \z}sonský slapy nezná sírou ^{G \z}pekla \\sířený,

Ať se ^{Ami}na hudsonský ^{G \z}šífy najmout ^{Ami}dá, \\^{G \z \z Ami}jo-ho-ho.

\sloka Ten, kdo nepřekládál uhlí, šíf když \\na mělčinu vjel,

Málo zná, málo zná,

Ten, kdo neměl tělo ztuhlý, až se \\nočním chladem chvěl,

Ať se na hudsonský šífy najmout dá, \\jo-ho-ho.

\refren A^{F \z}hoj, páru tam ^{Ami \z}hoď,

Ať ^{G}do pekla se dříve dohra^{Ami \z}bem,

^{G \z \z Ami}Jo-ho-ho, ^{G \z \z Ami}jo-ho-ho.

\sloka Ten, kdo nezná noční zpěvy \\zarostenejch lodníků

Jako já, jó, jako já,

Ten, kdo cejtí se bejt chlapem, \\umí dělat rotyku,

Ať se na hudsonský šífy najmout dá, \\jo-ho-ho.

\end{varwidth}\mezisloupci\begin{varwidth}[t]{0.5\textwidth}\setlength{\parindent}{0.45cm}  % V případě varianty č.2 jde odsud text do pravé části
\vspace*{0.375cm}

\sloka Ten, kdo má na bradě mlíko, kdo \\se rumem neopil,

Málo zná, málo zná,

Kdo necejtil hrůzu z vody, kde se \\málem utopil,

Ať se na hudsonský šífy najmout dá, \\jo-ho-ho.

\refren Ahoj, páru tam hoď...

\sloka Kdo má roztrhaný boty, kdo má \\pořád jenom hlad

Jako já, jó, jako já,

Kdo chce celý noci čuchat \\pekelnýho vohně smrad,

Ať se na hudsonský šífy najmout dá, \\jo-ho-ho.

\sloka Kdo chce zhebnout třeba zejtra, \\komu je to všechno fuk,

Kdo je sám, jó, jako já,

Kdo má srdce v správným místě, \\kdo je prostě príma kluk,

Ať se na hudsonský šífy najmout dá, \\jo-ho-ho.

\refren Ahoj, páru tam hoď,
Ať do pekla \\se dříve dohrabem,
\\Jo-ho-ho, jo-ho-ho, Jo-ho-ho ...


\end{centerjustified}
\setcounter{Slokočet}{0}
\end{song}
