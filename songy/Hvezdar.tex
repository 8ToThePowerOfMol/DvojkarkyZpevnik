%%%%%%%%%%%%%%%%%%%%%%%%%%%%%%%%%%%%%%%%%%%%%%%%%%%%%
%			ŠABLONA PÍSNIČEK v. 18.09               %
%%%%%%%%%%%%%%%%%%%%%%%%%%%%%%%%%%%%%%%%%%%%%%%%%%%%%
% Tento soubor slouží jako (naučná) šablona, pomocí 
% které lze vytvářet zdrojové soubory k jednotlivým 
% písním.
%%%%%%%%%%%%%%%%%%%%%%%%%%%%%%%%%%%%%%%%%%%%%%%%%%%%%
%			Jak psát soubory songů?                 %
%%%%%%%%%%%%%%%%%%%%%%%%%%%%%%%%%%%%%%%%%%%%%%%%%%%%%
%	1. Text písně se začíná psát na místě START 
%	   a končí na místě END. Zbylý text ignorujte.
%	2. Jak bude vypadat pdf písně zjistíte po tom, 
%	   co soubor zkompilujete pomocí souboru   
%      ../Generator/generator. 
%	3. Při psaní dodržujte následující TeX pravidla:
%	 a) Nový řádek napíšete pomocí dvou odsazení 
%	    tedy dvou enterů.
%	 b) Nová sloka se píší pomocí \sloka a odsazení.
%		Refrén se píše jako \refren, v případě více 
%		refrénů \refren[č. refrénu].
%	 c) Akordy se píšou tak, že napíšete před slovo,
%	    kde chcete mít akord (bez mezery):
%		^{AKORD1\,AKORD2...}.
%	4. Pokud chcete ušetřit tvůrcům práci, tak 
%	   si přečtěte další poučný soubor o typografii 
%	   ../../Typo_pravidla.txt.
%	5. Akordy stačí psát jen do první sloky, když 
%	   se nezmění -- kytaristé to zvládnou
%	7. Název písně pište na místo [NÁZEV] a autora 
%	   pište na místo [AUTOR] 
%	7. Jak psát věci na české klávesnici:
%	   \ = alt gr + q; [/] = alt gr f/g; 
%      {/} = alt gr + b/n; ^ = alt gr + 3 , cokoliv
%%%%%%%%%%%%%%%%%%%%%%%%%%%%%%%%%%%%%%%%%%%%%%%%%%%%%
%			Jak kompilovat jednotlivé písně?        %
%%%%%%%%%%%%%%%%%%%%%%%%%%%%%%%%%%%%%%%%%%%%%%%%%%%%%
%	1. Více návodu je k tomuto napsáno v souboru 
%      ../Generator/generator. 
%%%%%%%%%%%%%%%%%%%%%%%%%%%%%%%%%%%%%%%%%%%%%%%%%%%%%
%			Jak kompilovat celý zpěvník?			%
%%%%%%%%%%%%%%%%%%%%%%%%%%%%%%%%%%%%%%%%%%%%%%%%%%%%%
%	1. Více návodu je k tomuto napsáno v souboru
%	   ../Cely_zpevnik/zpevnik.tex.
%%%%%%%%%%%%%%%%%%%%%%%%%%%%%%%%%%%%%%%%%%%%%%%%%%%%%
\begin{song}{title=\predtitle \centering Hvězdář \\\large UDG }  %% sem se napíše jméno songu a autor

\vspace*{.5cm}

\begin{centerjustified}
\vetsi
\sloka
Ztrácíš se ^{D\z}před očima, rosteš jen ^{A}ve vlastním stínu.

Každá ^{Emi\z}další vina, odkrývá ^{Cmaj7}mojí vinu.

Ztrácíš se před očima, rosteš jen ve vlastním stínu.

Každá další vina odkrývá mojí vinu.

\sloka
/: Ve vínu dávno nic nehledám, (dávno nic) nehledám. :/

\refren
Jak luna mizíš s nocí v bělostných šatech pro nemocné,

prosit je zvláštní pocit, jen ať je den noc ne.

Jak luna mizíš s nocí v bělostných šatech pro nemocné,

prosit je zvláštní pocit, jen ať je den noc ne.

\sloka
/: Od proseb dávno nic nečekám (dávno nic) nečekám. :/

\ssloka{Mezihra:}
\textbf{D A Emi Cmaj7 D A Emi Cmaj7}

\sloka
Na chodbách v bludných kruzích zářivka vyhasíná,

já ti do infuzí chci přilít trochu vína.

Na nebi jiných sluncí, jak se tam asi cítíš,

s nebeskou interpunkcí, jiným tulákům svítíš.

\sloka
/: Ve vínu dávno nic nehledám (dávno nic) nehledám. :/

\refren

\sloka
Obzor neklesne níž, je ráno a ty spíš.

Od vlků odraná hvězdáře Giordána\dots

Obzor neklesne níž, je ráno a ty spíš.

Od vlků odraná hvězdáře Giordána\dots

Obzor neklesne níž, je ráno a ty spíš.

Od vlků odraná hvězdáře Giordána\elipsa.\elipsa.\elipsa.~opouštíš.

\end{centerjustified}

\setcounter{Slokočet}{0}
\end{song}


