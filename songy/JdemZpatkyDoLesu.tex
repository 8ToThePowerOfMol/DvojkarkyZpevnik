%%%%%%%%%%%%%%%%%%%%%%%%%%%%%%%%%%%%%%%%%%%%%%%%%%%%%
%			ŠABLONA PÍSNIČEK v. 18.09               %
%%%%%%%%%%%%%%%%%%%%%%%%%%%%%%%%%%%%%%%%%%%%%%%%%%%%%
% Tento soubor slouží jako (naučná) šablona, pomocí 
% které lze vytvářet zdrojové soubory k jednotlivým 
% písním.
%%%%%%%%%%%%%%%%%%%%%%%%%%%%%%%%%%%%%%%%%%%%%%%%%%%%%
%			Jak psát soubory songů?                 %
%%%%%%%%%%%%%%%%%%%%%%%%%%%%%%%%%%%%%%%%%%%%%%%%%%%%%
%	1. Text písně se začíná psát na místě START 
%	   a končí na místě END. Zbylý text ignorujte.
%	2. Jak bude vypadat pdf písně zjistíte po tom, 
%	   co soubor zkompilujete pomocí souboru   
%      ../Generator/generator. 
%	3. Při psaní dodržujte následující TeX pravidla:
%	 a) Nový řádek napíšete pomocí dvou odsazení 
%	    tedy dvou enterů.
%	 b) Nová sloka se píší pomocí \sloka a odsazení.
%		Refrén se píše jako \refren, v případě více 
%		refrénů \refren[č. refrénu].
%	 c) Akordy se píšou tak, že napíšete před slovo,
%	    kde chcete mít akord (bez mezery):
%		^{AKORD1\,AKORD2...}.
%	4. Pokud chcete ušetřit tvůrcům práci, tak 
%	   si přečtěte další poučný soubor o typografii 
%	   ../../Typo_pravidla.txt.
%	5. Akordy stačí psát jen do první sloky, když 
%	   se nezmění -- kytaristé to zvládnou
%	7. Název písně pište na místo [NÁZEV] a autora 
%	   pište na místo [AUTOR] 
%	7. Jak psát věci na české klávesnici:
%	   \ = alt gr + q; [/] = alt gr f/g; 
%      {/} = alt gr + b/n; ^ = alt gr + 3 , cokoliv
%%%%%%%%%%%%%%%%%%%%%%%%%%%%%%%%%%%%%%%%%%%%%%%%%%%%%
%			Jak kompilovat jednotlivé písně?        %
%%%%%%%%%%%%%%%%%%%%%%%%%%%%%%%%%%%%%%%%%%%%%%%%%%%%%
%	1. Více návodu je k tomuto napsáno v souboru 
%      ../Generator/generator. 
%%%%%%%%%%%%%%%%%%%%%%%%%%%%%%%%%%%%%%%%%%%%%%%%%%%%%
%			Jak kompilovat celý zpěvník?			%
%%%%%%%%%%%%%%%%%%%%%%%%%%%%%%%%%%%%%%%%%%%%%%%%%%%%%
%	1. Více návodu je k tomuto napsáno v souboru
%	   ../Cely_zpevnik/zpevnik.tex.
%%%%%%%%%%%%%%%%%%%%%%%%%%%%%%%%%%%%%%%%%%%%%%%%%%%%%
\begin{song}{title=\predtitle \centering Jdem zpátky do lesů \\\large Žalman }  %% sem se napíše jméno songu a autor

\vspace*{.5cm}

\begin{centerjustified}
\vetsi
\sloka
^{Ami7}Sedím na kolejích, ^{D}které nikam ^{\z G C G}nevedou,~~~~~~

koukám na kopretinu, jak miluje se s lebedou,

^{Ami7}mraky vzaly slunce ^*{D}za se pod svou ^{\z G Emi}ochranu,~~~~~~~

^{Ami7}jen~ty nejdeš, holka zlatá, ^{D}kdypak já tě ^{\z G D}dostanu?~~~

\refren
Z ^{G}ráje, my vyhnaní z ^{Emi}ráje,

kde není už ^{Ami7}místa, ^*{\z C7}prej\: něco se ^{\z G\phantom{G} D}chystá,~~

z ^{G}ráje nablýskaných ^{Emi}plesů,

jdem zpátky do ^*{Ami C7}lesů~za~něj aký ^{G}čas.

\sloka
Vlak nám včera ujel ze stanice do nebe,

málo jsi se snažil, málo šel jsi do sebe,

šel jsi vlastní cestou, a to se zrovna nenosí,

i pes, kterej chce přízeň, napřed svýho pána poprosí.

\refren

\sloka
Už tě vidím z dálky, jak máváš na mě korunou,

a jestli nám to bude stačit, zatleskáme na druhou,

zabalíme všechny, co si dávaj' rande za branou,

v ráji není místa, možná v pekle se nás zastanou.

\refren

\end{centerjustified}
\setcounter{Slokočet}{0}
\end{song}
