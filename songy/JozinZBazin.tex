%%%%%%%%%%%%%%%%%%%%%%%%%%%%%%%%%%%%%%%%%%%%%%%%%%%%%
%			ŠABLONA PÍSNIČEK v. 18.09               %
%%%%%%%%%%%%%%%%%%%%%%%%%%%%%%%%%%%%%%%%%%%%%%%%%%%%%
% Tento soubor slouží jako (naučná) šablona, pomocí 
% které lze vytvářet zdrojové soubory k jednotlivým 
% písním.
%%%%%%%%%%%%%%%%%%%%%%%%%%%%%%%%%%%%%%%%%%%%%%%%%%%%%
%			Jak psát soubory songů?                 %
%%%%%%%%%%%%%%%%%%%%%%%%%%%%%%%%%%%%%%%%%%%%%%%%%%%%%
%	1. Text písně se začíná psát na místě START 
%	   a končí na místě END. Zbylý text ignorujte.
%	2. Jak bude vypadat pdf písně zjistíte po tom, 
%	   co soubor zkompilujete pomocí souboru   
%      ../Generator/generator. 
%	3. Při psaní dodržujte následující TeX pravidla:
%	 a) Nový řádek napíšete pomocí dvou odsazení 
%	    tedy dvou enterů.
%	 b) Nová sloka se píší pomocí \sloka a odsazení.
%		Refrén se píše jako \refren, v případě více 
%		refrénů \refren[č. refrénu].
%	 c) Akordy se píšou tak, že napíšete před slovo,
%	    kde chcete mít akord (bez mezery):
%		^{AKORD1\,AKORD2...}.
%	4. Pokud chcete ušetřit tvůrcům práci, tak 
%	   si přečtěte další poučný soubor o typografii 
%	   ../../Typo_pravidla.txt.
%	5. Akordy stačí psát jen do první sloky, když 
%	   se nezmění -- kytaristé to zvládnou
%	7. Název písně pište na místo [NÁZEV] a autora 
%	   pište na místo [AUTOR] 
%	7. Jak psát věci na české klávesnici:
%	   \ = alt gr + q; [/] = alt gr f/g; 
%      {/} = alt gr + b/n; ^ = alt gr + 3 , cokoliv
%%%%%%%%%%%%%%%%%%%%%%%%%%%%%%%%%%%%%%%%%%%%%%%%%%%%%
%			Jak kompilovat jednotlivé písně?        %
%%%%%%%%%%%%%%%%%%%%%%%%%%%%%%%%%%%%%%%%%%%%%%%%%%%%%
%	1. Více návodu je k tomuto napsáno v souboru 
%      ../Generator/generator. 
%%%%%%%%%%%%%%%%%%%%%%%%%%%%%%%%%%%%%%%%%%%%%%%%%%%%%
%			Jak kompilovat celý zpěvník?			%
%%%%%%%%%%%%%%%%%%%%%%%%%%%%%%%%%%%%%%%%%%%%%%%%%%%%%
%	1. Více návodu je k tomuto napsáno v souboru
%	   ../Cely_zpevnik/zpevnik.tex.
%%%%%%%%%%%%%%%%%%%%%%%%%%%%%%%%%%%%%%%%%%%%%%%%%%%%%
\begin{song}{title=\predtitle \centering Jožin z bažin \\\large Ivan Mládek }  %% sem se napíše jméno songu a autor

\vspace*{.5cm}
\moveright .5cm \vbox{
\begin{centerjustified}
\vetsi
\sloka
^{Emi \z}Jedu takhle tábořit Škodou 100 na ^{Emi\z }Oravu,

spěchám proto, riskuji, ^{H7\z}projíždím přes ^{Emi \z}Moravu.

^{D7}Řádí tam to ^{G\z}strašidlo, ^{D7\z}vystupuje z ^{G H7\z}bažin,

^{Emi\z}žere hlavně Pražáky, ^{H7\z}jmenuje se ^{Emi D7}Jožin.~~

\refren
^{G\z}Jožin z bažin močálem se ^{D7\z}plíží,

Jožin z bažin k vesnici se ^{G\z}blíží,

Jožin z bažin už si zuby ^{D7\z}brousí,

Jožin z bažin kouše, saje, ^{G\z}rdousí.

^{C}Na Jožina z ^{G\z}bažin, ^{D\z}koho by to ^{\z G}napadlo,

^{C\z}platí jen a ^{G\z}pouze ^{D\z}práškovací ^{G H7 Emi}letadlo.~~~~~~~~~~~~~~~~~

\sloka
Projížděl jsem dědinou cestou na Vizovice,

přivítal mě předseda, řek' mi u slivovice:

\uv{Živého či mrtvého Jožina kdo přivede,

tomu já dám za ženu dceru a půl JZD!}

\refren

\sloka
Říkám:\uv{Dej mi, předsedo, letadlo a prášek,

Jožina ti přivedu, nevidím v tom háček.}

Předseda mi vyhověl, ráno jsem se vznesl,

na Jožina z letadla prášek pěkně klesl.

\refren
Jožin z bažin už je celý bílý,

Jožin z bažin z močálu ven pílí,

Jožin z bažin dostal se na kámen,

Jožin z bažin -- tady je s ním amen!

Jožina jsem dohnal, už ho držím, johoho,

dobré každé lóve, prodám já ho do ZOO.

\end{centerjustified}
}
\setcounter{Slokočet}{0}
\end{song}
