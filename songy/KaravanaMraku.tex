%\documentclass[../main.tex]{subfiles}

\begin{song}{title=\centering Karavana Mraků \\\normalsize Karel Kryl  \vspace*{-0.3cm}}  %% sem se napíše jméno songu a autor
\fontsize{12pt}{13pt}\selectfont
\moveright 4.5cm \vbox{      %Varianta č. 1  ---> Jeden sloupec zarovnaný na střed

\sloka
^{D}Slunce je zlatou skobou ^{Hmi}na vobloze přibitý, 

^{G}pod sluncem ^{A}sedlo kožený, ^{D A7} 

^{D}pod sedlem kůň, pod koněm ^{Hmi}moje boty rozbitý 

^{G}a starý ^{A}ruce sedřený. ^{D} 

\refren
^{D7}Dopředu ^{G}jít s tou ^{A}karavanou ^{Hmi}mraků, 

schovat svou ^{G}pleš pod ^{A}stetson ^{Hmi}děravý,

/: jen kousek ^{Emi}jít, jen ^{A7}chvíli, ^{Hmi}do ^{Emi}soumraku, 

až tam, kde ^{Hmi}svítí město, ^{F\#}město ^{Hmi}bělavý. :/^{A7}

\sloka
^{D}Vítr si tiše hvízdá po ^{Hmi}silnici spálený, 

v ^{G}tom městě ^{A}nikdo ^{D}nezdraví, ^{A7} 

^{D}šerif i soudce - ^{Hmi}gangsteři, voba řádně zvolení

a ^{G}lidi ^{A}strachem nezdraví. ^{D}

\sloka
Sto cizejch zabíječů s pistolema skotačí 

a zákon džungle panuje, 

provazník plete smyčky, hrobař kopat nestačí 

a truhlář rakve hobluje. 

\refren
^{D7}V městě ^{G}je řád a ^{A}pro každého ^{Hmi}práce, 

 buď ještě ^{G}rád, když ^{A}huba voněmí,^{Hmi}  
 
/: může tě ^{Emi}hřát, že ^{A7}nejsi ^{Hmi}na voprátce ^{Emi}  

nebo že ^{Hmi}neležíš pár ^{F\#}inchů pod ^{Hmi}zemí. :/^{A7}  

\sloka = 1.

\refren
Pryč odtud jít s tou karavanou mraků, 

kde tichej dům a pušky rezavý,  

/: orat a sít od rána do soumraku  

a nechat zapomenout srdce bolavý. :/  

}
\setcounter{Slokočet}{0}
\end{song}

