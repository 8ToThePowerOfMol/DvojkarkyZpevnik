\begin{song}{title=\centering Král a klaun \\\normalsize Karel Kryl  \vspace*{-0.3cm}}  %% sem se napíše jméno songu a autor
\moveright 4cm \vbox{      %Varianta č. 1  ---> Jeden sloupec zarovnaný na střed	

\predehra
\textbf{G C G C G C}

\sloka 
	^{D}Král ^{C}do boje ^{C\,G}táh', ^{G}do ^{C}veliké ^{G}dálky, ^{C\,G}

	a s ^{C}ním do té ^{G}války ^{D7}jel na mezku ^{G}klaun,

	^{D}než ^{C}hledí si ^{G\,C}stáh', ^{G}tak z ^{C}výrazu ^{G}tváře ^{C\,G}

	^{C}bys nepoznal ^{G}lháře, ^{D7}co zakrývá ^{G}strach.

	^{D7}Tiše šeptal při té hrůze: ,,^{G}Inter arma silent musae,''

	^{A}místo zvonku cinkal ^{D7}brněním. ^{C#7\,D7}

	Král ^{C}do boje ^{G\,C}táh', ^{G}do ^{C}veliké ^{G}dálky,  ^{C\,G}

	^{C}a s ním do té ^{G}války ^{D7}jel na mezku ^{G}klaun.  ^{C\,G\,A7}

\sloka
	Král do boje táh', a sotva se vzdálil,
	
	tak vesnice pálil a dobýval měst,

	klaun v očích měl hněv, když sledoval žháře,

	jak smývali v páře prach z rukou a krev.
	
	Tiše šeptal při té hrůze: ,,Inter arma silent musae,''
	
	místo loutny držel v ruce meč.

	Král do boje táh', a sotva se vzdálil,
   
	tak vesnice pálil a dobýval měst.

\sloka
	Král do boje táh', s tou vraždící lůzou
	
	klaun třásl se hrůzou a odvetu kul,

	když v noci byl klid, tak oklamal stráže
	
	a, nemaje páže, sám burcoval lid.
	
	Všude křičel do té hrůzy, ve válce že mlčí Múzy,
	
	muži by však mlčet neměli.

	Král do boje táh', s tou vraždící lůzou

	klaun třásl se hrůzou a odvetu kul.

\sloka
	Král do boje táh', a v červáncích vlídných

	zřel, na čele bídných, jak vstříc jde mu klaun,

	když západ pak vzplál, tok potoků temněl,

	klaun tušení neměl, jak zahynul král:

	kdekdo křičel při té hrůze: ,,Inter arma silent musae,''

	krále z toho strachu trefil šlak.

	Klaun tiše se smál a zem žila dále
	
	a neměla krále, klaun na loutnu hrál,

	^{D7}klaun na loutnu ^{G}hrál, ^{D7}klaun na loutnu ^{G}hrál \ldots




}
\setcounter{Slokočet}{0}
\end{song}
