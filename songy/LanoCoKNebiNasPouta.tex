%\documentclass[../main.tex]{subfiles}

\begin{song}{title=\centering Lano co k nebi nás poutá \\\normalsize Traband  \vspace*{-0.3cm}}  %% sem se napíše jméno songu a autor
\moveright 3.2cm \vbox{      %Varianta č. 1  ---> Jeden sloupec zarovnaný na střed


\sloka
^{F}Já sedával v přístavu, ^{Gmi}popíjel kořalu, ^{F}s holkama ^{C7}laškoval.

A ^{F}bylo mi fuk, co je, ^{Gmi}hlavně když fajfka ^{F}mi ^{C7}doutná.

Co ^{F}bylo už není, ^{Gmi}všechno mý jmění jsem ^{A7}dávno ^{Dmi}rozfofroval.

Jsme ^{F}silný jak silný je ^{Gmi}lano, co k nebi nás ^{F C7 F}poutá.

\sloka
Ale najednou zmatek, když vešel ten chlápek, na mou duši! 

Objedná si drink a sedne si vedle do kouta.

Pak se nakloní ke mně a povídá jemně: \uv{Matouši!} 

Jsme silný jak silný je lano, co k nebi nás poutá.

\sloka
Já povídám: \uv{Pane, odkud se známe? Esli se nemýlíte?

A co je vám do mě, starýho mrchožrouta?}

On na to: \uv{Pojď, dej se na moji loď, má jméno Eternité.} 

Jsme silný jak silný je lano, co k nebi nás poutá.

\sloka
^{Dmi, D7}
^{G}Ty jeho slova se ^{Ami}zařízly do mě ^{G}jako bys břitvou ^{D7}šmik,

^{G}jako když po ránu ^{Ami}vzbudí tě křik ^{G D7}kohouta.

^{G}Tak povídám: ,,Jdem!“ A ^{Ami}ještě ten den stal se ^{H7}ze mě ^{Emi}námořník. 

^{G}Jsme silný jak silný je ^{Ami}lano, co k nebi nás ^{G D7 G}poutá.

\sloka
Tak zvedněme kotvy a napněme plachty, vítr začíná vát! 

Černý myšlenky vymeťme někam do kouta.

Hudba ať hraje o dobytí ráje, teď není čeho se bát.

/: Jsme silný jak silný je lano, co k nebi nás poutá. :/

^{Emi}
^{G, D7, G, D7, G, D7, G, D7, G}

}
\setcounter{Slokočet}{0}
\end{song}
