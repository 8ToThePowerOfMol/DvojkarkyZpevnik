\begin{song}{title=\centering Lilie \\\normalsize Karel Kryl  \vspace*{-0.3cm}}  %% sem se napíše jméno songu a autor
\moveright 4cm \vbox{      %Varianta č. 1  ---> Jeden sloupec zarovnaný na střed	

\sloka 
	^{Cmi}Než zavřel bránu, oděl se do oceli ^{G#}, ^{G}a zhasil ^{Cmi}svíci, ^{G} 

	^{Cmi}bylo už k ránu, 
	
	políbil na ^{G#}posteli, ^{G}svou ženu ^{Cmi}spící, 

	/: ^{D#}spala jak víla, ^{A#}jen vlasy halily ji, 
	
	^{Cmi}jak zlatá žíla, ^{G#}jak jitra v ^{G}Kastilii, 
	
	^{Cmi}něžná a bílá jak rosa na lilii, 
	
	^{G#} ^{G}jak luna ^{Cmi}bdící. :/ 

\phantom{tom}

(pískání) \textbf{Cmi, G#, G, Cmi, G# G Cmi}

\sloka
	Jen mraky šedé a ohně na pahorcích -- svědkové němí, 
	
	lilie bledé svítily na praporcích, když táhli zemí, 

	/: polnice břeskné vojácká melodie, 

	potoky teskné - to koně zkalili je, 
	
	a krev se leskne, když padla na lilie kapkami třemi. :/ 

\sloka
	Dozrály trnky, zvon zvoní na neděli a čas se vleče, 

	rezavé skvrnky zůstaly na čepeli u jílce meče, 
	
	/: s rukama v týle jdou vdovy alejemi, 

	za dlouhé chvíle zdobí se liliemi, 
	
	lilie bílé s rudými krůpějemi trhají vkleče. :/  



}
\setcounter{Slokočet}{0}
\end{song}
