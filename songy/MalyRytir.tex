%%%%%%%%%%%%%%%%%%%%%%%%%%%%%%%%%%%%%%%%%%%%%%%%%%%%%
%			ŠABLONA PÍSNIČEK v. 18.09               %
%%%%%%%%%%%%%%%%%%%%%%%%%%%%%%%%%%%%%%%%%%%%%%%%%%%%%
% Tento soubor slouží jako (naučná) šablona, pomocí 
% které lze vytvářet zdrojové soubory k jednotlivým 
% písním.
%%%%%%%%%%%%%%%%%%%%%%%%%%%%%%%%%%%%%%%%%%%%%%%%%%%%%
%			Jak psát soubory songů?                 %
%%%%%%%%%%%%%%%%%%%%%%%%%%%%%%%%%%%%%%%%%%%%%%%%%%%%%
%	1. Text písně se začíná psát na místě START 
%	   a končí na místě END. Zbylý text ignorujte.
%	2. Jak bude vypadat pdf písně zjistíte po tom, 
%	   co soubor zkompilujete pomocí souboru   
%      ../Generator/generator. 
%	3. Při psaní dodržujte následující TeX pravidla:
%	 a) Nový řádek napíšete pomocí dvou odsazení 
%	    tedy dvou enterů.
%	 b) Nová sloka se píší pomocí \sloka a odsazení.
%		Refrén se píše jako \refren, v případě více 
%		refrénů \refren[č. refrénu].
%	 c) Akordy se píšou tak, že napíšete před slovo,
%	    kde chcete mít akord (bez mezery):
%		^{AKORD1\,AKORD2...}.
%	4. Pokud chcete ušetřit tvůrcům práci, tak 
%	   si přečtěte další poučný soubor o typografii 
%	   ../../Typo_pravidla.txt.
%	5. Akordy stačí psát jen do první sloky, když 
%	   se nezmění -- kytaristé to zvládnou
%	7. Název písně pište na místo [NÁZEV] a autora 
%	   pište na místo [AUTOR] 
%	7. Jak psát věci na české klávesnici:
%	   \ = alt gr + q; [/] = alt gr f/g; 
%      {/} = alt gr + b/n; ^ = alt gr + 3 , cokoliv
%%%%%%%%%%%%%%%%%%%%%%%%%%%%%%%%%%%%%%%%%%%%%%%%%%%%%
%			Jak kompilovat jednotlivé písně?        %
%%%%%%%%%%%%%%%%%%%%%%%%%%%%%%%%%%%%%%%%%%%%%%%%%%%%%
%	1. Více návodu je k tomuto napsáno v souboru 
%      ../Generator/generator. 
%%%%%%%%%%%%%%%%%%%%%%%%%%%%%%%%%%%%%%%%%%%%%%%%%%%%%
%			Jak kompilovat celý zpěvník?			%
%%%%%%%%%%%%%%%%%%%%%%%%%%%%%%%%%%%%%%%%%%%%%%%%%%%%%
%	1. Více návodu je k tomuto napsáno v souboru
%	   ../Cely_zpevnik/zpevnik.tex.
%%%%%%%%%%%%%%%%%%%%%%%%%%%%%%%%%%%%%%%%%%%%%%%%%%%%%
\begin{song}{title=\predtitle \centering Malý rytíř \\\large Klíč }  %% sem se napíše jméno songu a autor

\vspace*{-.5cm}

\begin{centerjustified}
\vetsi
\refren[1]
^{G}Má pět let a ^{D}jméno ^*{G}Par sifal,

^*{Emi}pod~st olem jak v ^{D}hradu ^{Emi}bydlí,

^{G}na plotně mi ^{D}střeží ^{G}svatý Grál,

^*{Emi}na~tur naj mi ^{D}jezdí ^{Emi}židlí,

^{C}pod přílbou z ^{D}novin ^*{G}skálo pevný ^{D}zrak,

^{G}čelo jako ^{D\z}anděl, ^{G}sílu jako ^{D}drak,

^*{G}Nor many ^{D}lžičkou ^{Emi}mydlí.

\sloka
^*{Emi}Po~bit vách ^{\z D}vždy se ^{Emi}schoulí, ^{G}klášter si ^{D \phantom{D} G}vyhledá,

^{Emi}rytíř s ^*{D}mod rou ^{Emi}boulí ^*{G}pof oukat ^{D}se ^{G}dá,

^{C}pán hradu s ^{D}pláčem ^*{G}záp olí a máma s ^*{D}pán em ^{G}zas,

^{C}jen co ta ^*{D}bou le ^*{G}přebo lí,

chápe se ^{D}dřevce, ^{G}být doma ^*{D}nec hce, ^*{G}táh ne do ^{\z D}polí, je ^{\z Emi}čas.

\refren[1]

\refren[2]
 Má pět let a jméno Lohengrin,

 místo meče koště svírá,

 zahrabal si poklad do peřin,

 tak silná je jeho víra,

 pod přílbou z novin skálopevný zrak,

 čelo jako anděl, sílu jako drak,

 v kalhotách zeje díra.

\sloka
Ze spaní skály láme, kraluje, jak se dá,

s jedním uchem máme trůn, na němž zase dá,

starost mám, které z princezen svůj prsten jednou dá,

kde má tu svoji Svatou zem,

proč starost, mámo, teď je teprv ráno, on vyhledá ji sám.

\refren[2]

\end{centerjustified}
\setcounter{Slokočet}{0}
\end{song}
