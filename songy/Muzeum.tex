\begin{song}{title=\centering Muzeum \\\normalsize Jaromír Nohavica  \vspace*{-0.3cm}}  %% sem se napíše jméno songu a autor
\moveright 3cm \vbox{      %Varianta č. 1  ---> Jeden sloupec zarovnaný na střed	

\sloka
	^{D}Ve Slezském muzeu, ^{A}v oddělení ^{Hmi}třetihor
	
	je bílý ^{G}krokodýl a ^{Dmi}edvěd a liška a ^{A7}kamenní ^{D}trilobiti, ^{A7}
	
	^{D}chodí se tam jen tak, co ^{A}noha nohu ^{Hmi}mine, 
	
	abys viděl, jak ten ^{G}život plyne, ^{D}jaké je to všechno ^{A7}pomíjivé ^{D}živobytí,
	
	^{G}pak vyjdeš do parku a ^{D4sus}celou noc se ^{G}touláš noční Opavou

	a ^{C}opájíš se ^{G}představou, jaké to bude ^{D}v ^{G}ráji.
	

\refren
	V ^{D}pět třicet pět jednou z ^{A}pravidelných ^{Hmi}linek,
	
	sedm zastávek do ^{G}Kateřinek, ^{D}ukončete nástup, ^{A7}dveře se ^{D}zavírají.


\sloka
	Budeš-li poslouchat a nebudeš-li odmlouvat,
	
	složíš-li svoje maturity, vychováš pár dětí a vyděláš dost peněz,
	
	můžeš se za odměnu svézt na velkém kolotoči,
	
	dostaneš krásnou knihu s věnováním zaručeně,
	
	a ty bys chtěl plout na hřbetě krokodýla po řece Nil
	
	a volat: \uv{Tutanchámon, vivat, vivat!} po egyptském kraji.
	
\refren

\sloka
	Pionýrský šátek uvážeš si kolem krku,
	
	ve Valtické poručíš si čtyři deci rumu a utopence k tomu,
	
	na politém stole na ubruse píšeš svou rýmovanou Odysseu,
	
	nežli přijde někdo, abys šel už domů,
	
	ale není žádné doma jako není žádné venku, není žádné venku,
	
	to jsou jenom slova, která obrátit se dle libosti dají.

\refren

\sloka
	Možná si k tobě někdo přisedne,
	
	a možná to bude zrovna muž, který osobně znal Egypťana Sinuheta,
	
	dřevěnou nohou bude vyťukávat do podlahy rytmus metronomu,
	
	který tady klepe od počátku světa,
	
	nebyli jsme, nebudem a nebyli jsme, nebudem a co budem, až nebudem,
	
	jen navezená mrva v boží stáji.
	
	
\refren

}
\setcounter{Slokočet}{0}
\end{song}

