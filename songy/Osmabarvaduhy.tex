\begin{song}{title=\centering Osmá barva duhy \\\normalsize Jaromír Nohavica  \vspace*{-0.3cm}}  %% sem se napíše jméno songu a autor
\moveright 1cm \vbox{      %Varianta č. 1  ---> Jeden sloupec zarovnaný na střed	
\begin{minipage}[t]{0.48\textwidth}\setlength{\parindent}{0.45cm}  %Varianta č. 2 --> Dva sloupce
\sloka
	^{Ami}Chladná jsou ^{Dmi}dubnová ^{Ami}rána ^{Dmi},
	
	^{Ami}ze slunce je ^{Dmi}vidět jenom ^{C\,E}kousek.
	
	^{Ami}Ve flašce ^{Dmi}od ^{Ami\,Dmi}činzána
	
	^{Ami}úhoři, ^{Dmi}úhoři ^{G}třou ^{Ami}se.
	
\refren
	^{Dmi}Všichni moji ^{Ami}známí
	
	^{F}spí, ^{Dmi}spí,	^{E}spí doma s manželkami.
	
	^{Dmi}Zůstali jsme ^{Ami}sami –- ^{E}já a ^{Ami}já.
	
	^{Ami}Města jsou jedno jako druhý,
	
	černá je osmá barva duhy.
	
	^{Dmi}Černá je barva, kterou ^{E}mám teď ^{Ami}nejraděj.
	
	Jó je to bída, je to bída,
	
	hledal jsem ostrov jménem Atlantýda
	
	^{Dmi}a našel vody ^{E}vody \dots vody ^{Ami}habaděj.
	
\sloka
	Kdyby měl někdo z vás zájem,
	
	uděláme velikánský mejdan.
	
	Pojedme tam a zpět rájem
	
	a svatej Petr bude náš strejda.

\refren

\sloka
	Pod okny řve někdo: \uv{Kémo,
	
	každý správný folkáč nosí vousy}.
	
	A já umím písničky jen v E moll
	
	a prsty jsem si až do masa zbrousil.
	
	Protože:
	
\end{minipage}\begin{minipage}[t]{0.48\textwidth}\setlength{\parindent}{0.45cm}\vspace*{0.55cm}  % V případě varianty č.2 jde odsud text do pravé části

\refren

\sloka
	Všichni moji známí
	
	teď spí, spí, spí doma s manželkami.

	Zůstali jsme sami -- já a vy.
	
	Města jsou jedno jako druhý,
	
	černá je osmá barva duhy.
	
	Černá je barva, kterou mám teď nejraděj.
	
	Jó, je to bída, je to bída,
	
	hledáme ostrov jménem Atlantida
	
	A nacházíme vody, vody, vody, vody,
	
	vody, vody, vody, vody, vody, vody,
	
	vody, vody, vody, vody, vody, vody,

	vody, vody habaděj. 

\end{minipage}
}
\setcounter{Slokočet}{0}
\end{song}
