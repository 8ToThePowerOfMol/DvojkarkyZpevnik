\begin{song}{title=\centering Pijánovka \\\normalsize Tři sestry  \vspace*{-0.3cm}}  %% sem se napíše jméno songu a autor
\moveright 4.0cm \vbox{      %Varianta č. 1  ---> Jeden sloupec zarovnaný na střed	

\sloka
^{D}Na highway číslo jedna ^{A{\color{white}\_\_\_\_\_}}vyjetejma kolejema

v ^{G{\color{white}\_\_\_\_}}prachu na pět svíček brachu ^{A{\color{white}\_\_\_}}jezděj traky beze strachu.

^{D{\color{white}\_\_\_}}Možná potkáš cestou Láďu, ^{A\,\,}jak žene svůj dravej vůz,

^{G{\color{white}\_\_}}svojí starou rychlou káru, ^{A{\color{white}\_\_\_}}který říká autobus.

^{D}V Průhonicích vezme plnou, ^{A}na padesátým spláchne kolou

^{G{\color{white}\_}}rybu, kterou snědl celou ^{A{\color{white}\_\_\_\_\_}}smaženou jen trochu leklou.

\refren
/: ^{D{\color{white}\_\_\_\_}}Highway číslo ^{Hmi{\color{white}\_\_}}jedna a po ní jede ^{A\,\,}king.

^{D{\color{white}\_\_}}Láďa jede ^{Hmi{\color{white}\_\_\_\_}}autobusem ^{G}je fakt highway ^{A\,\,}king:/

\sloka
U Humpolce v příkrym kopci často

bouraj politici, Láďa vždycky zapálí si

camelku jak dobrodruzi, u Křížů si

koupí žvejku, velkej stejk a pěknou holku.

Na cestě je chlapům smutno, zvlášť když

nemaj rychlou ruku. Každej řidič má svůj

příběh, někdo zdrhá před osudem,

někdo zdrhá před svou starou

nebo ztratil prachy s vírou.


\refren


\sloka
Láďa nosí ve svém srdci příběh černej

jako krtci, plnej zrady, krve, pěstí,

závisti a nenávisti. Jeho příběh je tak

krutej, že je ze všech nejkrutější,

že ho Láďa nosí v sobě a zemře s ním

ve svym hrobě. A tak končí tahle píseň

o Láďovi s autobusem. Jeho příběh mi

neznáme, a tak o něm nezpíváme.


\refren


}
\setcounter{Slokočet}{0}
\end{song}
