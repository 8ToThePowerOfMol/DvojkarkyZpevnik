%%%%%%%%%%%%%%%%%%%%%%%%%%%%%%%%%%%%%%%%%%%%%%%%%%%%%
%			ŠABLONA PÍSNIČEK v. 18.09               %
%%%%%%%%%%%%%%%%%%%%%%%%%%%%%%%%%%%%%%%%%%%%%%%%%%%%%
% Tento soubor slouží jako (naučná) šablona, pomocí 
% které lze vytvářet zdrojové soubory k jednotlivým 
% písním.
%%%%%%%%%%%%%%%%%%%%%%%%%%%%%%%%%%%%%%%%%%%%%%%%%%%%%
%			Jak psát soubory songů?                 %
%%%%%%%%%%%%%%%%%%%%%%%%%%%%%%%%%%%%%%%%%%%%%%%%%%%%%
%	1. Text písně se začíná psát na místě START 
%	   a končí na místě END. Zbylý text ignorujte.
%	2. Jak bude vypadat pdf písně zjistíte po tom, 
%	   co soubor zkompilujete pomocí souboru   
%      ../Generator/generator. 
%	3. Při psaní dodržujte následující TeX pravidla:
%	 a) Nový řádek napíšete pomocí dvou odsazení 
%	    tedy dvou enterů.
%	 b) Nová sloka se píší pomocí \sloka a odsazení.
%		Refrén se píše jako \refren, v případě více 
%		refrénů \refren[č. refrénu].
%	 c) Akordy se píšou tak, že napíšete před slovo,
%	    kde chcete mít akord (bez mezery):
%		^{AKORD1\,AKORD2...}.
%	4. Pokud chcete ušetřit tvůrcům práci, tak 
%	   si přečtěte další poučný soubor o typografii 
%	   ../../Typo_pravidla.txt.
%	5. Akordy stačí psát jen do první sloky, když 
%	   se nezmění -- kytaristé to zvládnou
%	7. Název písně pište na místo [NÁZEV] a autora 
%	   pište na místo [AUTOR] 
%	7. Jak psát věci na české klávesnici:
%	   \ = alt gr + q; [/] = alt gr f/g; 
%      {/} = alt gr + b/n; ^ = alt gr + 3 , cokoliv
%%%%%%%%%%%%%%%%%%%%%%%%%%%%%%%%%%%%%%%%%%%%%%%%%%%%%
%			Jak kompilovat jednotlivé písně?        %
%%%%%%%%%%%%%%%%%%%%%%%%%%%%%%%%%%%%%%%%%%%%%%%%%%%%%
%	1. Více návodu je k tomuto napsáno v souboru 
%      ../Generator/generator. 
%%%%%%%%%%%%%%%%%%%%%%%%%%%%%%%%%%%%%%%%%%%%%%%%%%%%%
%			Jak kompilovat celý zpěvník?			%
%%%%%%%%%%%%%%%%%%%%%%%%%%%%%%%%%%%%%%%%%%%%%%%%%%%%%
%	1. Více návodu je k tomuto napsáno v souboru
%	   ../Cely_zpevnik/zpevnik.tex.
%%%%%%%%%%%%%%%%%%%%%%%%%%%%%%%%%%%%%%%%%%%%%%%%%%%%%
\begin{song}{title=\predtitle \centering Proklatej vůz \\\large Greenhorns }  %% sem se napíše jméno songu a autor

\vspace*{.5cm}

\begin{centerjustified}
\vetsi
\sloka
^{C}Čtyři (jen tři) (jen dvě) ^{G\z}bytelný ^{Ami}kola ^{F}má náš ^*{\z Emi}prok latej ^{G}vůz,

tak ^{C\z}ještě ^{F\z}pár ^*{C}dlo uhejch ^{G}mil ^*{C}zbe jvá ^{F\z}dál, ^{C\z}tam je ^{G}cíl,

a tak ^*{C}zpí vej o ^{F G}Santa ^{C F C}Cruz.~~~ ^{C7}

\refren
^{F\z}Polykej whisky a ^*{C}zví řenej prach,

^{G\z}nesmí nás ^*{C}por azit strach,

až ^*{F}pře jedem támhleten ^{C\z}pískovej plát,

pak ^{D7 \z}nemusíš ^{G}se už rudochů ^{C\z}bát.


\ssloka{Rec.:}
\textsc{Matka}: {Georgi, už jsem celá roztřesená, zastav!}

\textsc{Otec}: {Zalez zpátky do vozu, ženo!}


\sloka
= 1.


\refren

\ssloka{Rec.:}
\textsc{Syn:} {Tatínku, tatínku, už mám plnej nočníček!}

\textsc{Otec:} {Probůh, to není nočníček, to je soudek s prachem!}

\sloka
= 1.

\refren

\ssloka{Rec:}
\textsc{Děd:} {Synu, vždyť jedeme jak s hnojem!}

\textsc{Otec:} {Jó, koho jsem si naložil, toho vezu!}

\end{centerjustified}
\newpage
\begin{centerjustified}

\sloka
Už jen jediný kolo má náš proklatej vůz,

tak ještě pár dlouhejch mil zbejvá dál, tam je cíl,

a tak zpívej o Santa Cruz.

\refren

\ssloka{Rec:}
\textsc{Matka:} {Georgi, zastav, já se strašně bojím Indiánů!}

\textsc{Otec:} {Zatáhni za sebou plachtu, ženo, a mlč!}

\sloka
Už ani jediný kolo nemá náš proklatej vůz,

tak ještě pár dlouhejch mil zbejvá dál, tam je cíl,

a tak zpívej o Santa Cruz.

\refren

\ssloka{Rec:}
\textsc{Všichni:} Indiáni!!!

\end{centerjustified}
\setcounter{Slokočet}{0}
\end{song}
