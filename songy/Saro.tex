\begin{song}{title=\centering Sáro! \\\normalsize Traband  \vspace*{-0.3cm}}  %% sem se napíše jméno songu a autor
\moveright 1cm \vbox{      %Varianta č. 1  ---> Jeden sloupec zarovnaný na střed	
\begin{minipage}[t]{0.48\textwidth}\setlength{\parindent}{0.45cm}  %Varianta č. 2 --> Dva sloupce
\sloka 
	^{Ami}Sáro, ^{Emi}Sáro, ^{F}v noci se mi ^{C}zdálo

	že ^{F}tři poslové ^{C}Boží k nám ^{F}přišli na ^{G}oběd.

	^{Ami}Sáro, ^{Emi}Sáro, jak ^{F}moc a nebo ^{C}málo

	^{F}mi chybí abych ^{C}tvojí ^{F}duši mohl ^{G}rozumět?

\sloka
	Sbor kajícných mnichů jde krajinou v tichu
	
	a pro všechnu lidskou pýchu
	
	má jen přezíravý smích.

	A z prohraných válek se vojska domů vrací,
	
	však zbraně stále burácí
	
	a bitva zuří v nich.

\sloka
	Vévoda v zámku čeká na balkóně,
	
	až přivedou mu koně
	
	a pak mává na pozdrav.
	
	A srdcová dáma má v každé ruce růže.
	
	Tak snadno poplést může
	
	sto urozených hlav.

\sloka
	Královnin šašek s pusou od povidel
	
	sbírá zbytky jídel
	
	a myslí na útěk.
	
	A v podzemí skrytí slepí alchymisté
	
	už objevili jistě

	proti povinnosti lék.

	\end{minipage}\begin{minipage}[t]{0.48\textwidth}\setlength{\parindent}{0.45cm}\vspace*{0.55cm}  % V případě varianty č.2 jde odsud text do pravé části

\sloka
	Páv pod tvým oknem zpívá sotva procit
	
	o tajemstvích noci
	
	ve tvých zahradách.
	
	A já -- potulný kejklíř, co svázali mu ruce,

	teď hraju o tvé srdce
	
	a chci mít tě na dosah.

\sloka
	^{Ami}Sáro, ^{Emi}Sáro, ^{F}pomalu a ^{C}líně

	
	^{F}s hlavou na tvém ^{C}klíně ^{F}chci se ^{G}probouzet.
	
	^{F}Sáro, Sáro, ^{C}Sáro, ^{F}rosa padá ^{C}ráno
	
	a ^{F}v poledne už ^{C}možná ^{F}bude jiný ^{G}svět.

	^{F}Sáro, ^{C}Sáro, ^{F}vstávej, milá ^{C}Sáro!

	^{F}Andělé k nám ^{Dmi}přišli na ^{Cmaj}oběd.


\end{minipage}
}
\setcounter{Slokočet}{0}
\end{song}

