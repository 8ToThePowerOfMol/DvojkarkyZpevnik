%%%%%%%%%%%%%%%%%%%%%%%%%%%%%%%%%%%%%%%%%%%%%%%%%%%%%
%			ŠABLONA PÍSNIČEK v. 18.09               %
%%%%%%%%%%%%%%%%%%%%%%%%%%%%%%%%%%%%%%%%%%%%%%%%%%%%%
% Tento soubor slouží jako (naučná) šablona, pomocí 
% které lze vytvářet zdrojové soubory k jednotlivým 
% písním.
%%%%%%%%%%%%%%%%%%%%%%%%%%%%%%%%%%%%%%%%%%%%%%%%%%%%%
%			Jak psát soubory songů?                 %
%%%%%%%%%%%%%%%%%%%%%%%%%%%%%%%%%%%%%%%%%%%%%%%%%%%%%
%	1. Text písně se začíná psát na místě START 
%	   a končí na místě END. Zbylý text ignorujte.
%	2. Jak bude vypadat pdf písně zjistíte po tom, 
%	   co soubor zkompilujete pomocí souboru   
%      ../Generator/generator. 
%	3. Při psaní dodržujte následující TeX pravidla:
%	 a) Nový řádek napíšete pomocí dvou odsazení 
%	    tedy dvou enterů.
%	 b) Nová sloka se píší pomocí \sloka a odsazení.
%		Refrén se píše jako \refren, v případě více 
%		refrénů \refren[č. refrénu].
%	 c) Akordy se píšou tak, že napíšete před slovo,
%	    kde chcete mít akord (bez mezery):
%		^{AKORD1\,AKORD2...}.
%	4. Pokud chcete ušetřit tvůrcům práci, tak 
%	   si přečtěte další poučný soubor o typografii 
%	   ../../Typo_pravidla.txt.
%	5. Akordy stačí psát jen do první sloky, když 
%	   se nezmění -- kytaristé to zvládnou
%	7. Název písně pište na místo [NÁZEV] a autora 
%	   pište na místo [AUTOR] 
%	7. Jak psát věci na české klávesnici:
%	   \ = alt gr + q; [/] = alt gr f/g; 
%      {/} = alt gr + b/n; ^ = alt gr + 3 , cokoliv
%%%%%%%%%%%%%%%%%%%%%%%%%%%%%%%%%%%%%%%%%%%%%%%%%%%%%
%			Jak kompilovat jednotlivé písně?        %
%%%%%%%%%%%%%%%%%%%%%%%%%%%%%%%%%%%%%%%%%%%%%%%%%%%%%
%	1. Více návodu je k tomuto napsáno v souboru 
%      ../Generator/generator. 
%%%%%%%%%%%%%%%%%%%%%%%%%%%%%%%%%%%%%%%%%%%%%%%%%%%%%
%			Jak kompilovat celý zpěvník?			%
%%%%%%%%%%%%%%%%%%%%%%%%%%%%%%%%%%%%%%%%%%%%%%%%%%%%%
%	1. Více návodu je k tomuto napsáno v souboru
%	   ../Cely_zpevnik/zpevnik.tex.
%%%%%%%%%%%%%%%%%%%%%%%%%%%%%%%%%%%%%%%%%%%%%%%%%%%%%
\begin{song}{title=\predtitle \centering Staré dobré časy \\\large Jaromír Nohavica }  %% sem se napíše jméno songu a autor

\vetsi

\moveright -1.5cm \vbox{

%\vspace*{-.3cm}

\begin{centerjustified}

\begin{varwidth}[t]{0.58\textwidth}\setlength{\parindent}{\pindent}  %Varianta č. 2 --> Dva sloupce

\sloka
^{D \z}Řekla mi včera dívka ^{Hmi \z}Tereza,

^{D}že patřím do starého ^{Hmi \z}železa,

^{D}že už mi šroubky, nýty ^{Hmi}reziví,

^{D}já se té dívce vlastně ^{Hmi \z}nedivím,

^{G}jsem starý vysloužilý ^*{Emi}party zán,

^{D}puding, co ^{D/A}byl už z misky ^{Hmi \z}vylízán,

příslušník dávno ^*{\z A}vyhynu lé rasy,

^{D}jó, kde jsou ty časy.

\sloka
^{E \z}Člověk měl prázdný břich, však ^{C#mi \z}víry~byl pln,

uši měl otlačené od krátkých vln,

do práce chodil jenom na vyspání

a všichni mukli byli pruhovaní,

^*{A}mel ouny stály sedm a ^{F#mi}třešně tři,

^*{E}Slo váci ^{E/D}byli~naši rodní ^{C#mi}bratři,

kdo tě chtěl urazit, řek': ^{H},,ty jseš Vasil``,

^{E}jó, kde jsou ty časy.

\sloka
Hráli jsme nebezpečnou muziku

a dívky byly svolné ke styku

a ten, kdo nechyt' jednu pendrekem,

ten nebyl právoplatným člověkem,

pod heslem \uv{Spějeme ke šťastným zítřkům!}

jsme rozkládali rodnou zemi zvnitřku

my -- příslušníci disidentskej klasy,

jó, kde jsou ty časy.

%\end{centerjustified}
%\newpage
%\begin{centerjustified}

\end{varwidth}\mezisloupci\begin{varwidth}[t]{0.48\textwidth}\setlength{\parindent}{\pindent}
\vspace*{0.405cm}  % V případě varianty č.2 jde odsud text do pravé části

\sloka
Vaškovi do vězení v Heřmanicích

jsme tajně posílali smaženici

a mezi vajíčkama byly skryty

pilníky, ba i libri prohibiti,

jézeďák staral se jen o dobytek,

jistý byl sociální výdobytek,

po lukách zněly šťastné dětské hlasy,

jó, kde jsou ty časy.

\sloka
Písně se pašovaly pod košilí,

náměstím velel soudruh Džugašvili,

ruskému medvědovi línala srst,

jestřábům z Ameriky podbíral prst,

dostat se za železnou střeženou zeď,

to bylo dobrodružství, ne jako teď,

teď mají všichni vůkol svoje pasy,

jó, kde jsou ty časy.

\sloka
Svetr se po vyprání správně srážel,

prezident na Hradě moc nepřekážel,

na polích moh' jsi potkat antikrista

a hrdě znělo slovo \uv{komunista},

noviny byly troje, strana jedna

a policie krásně zodpovědná,

cikánům hrůzou ježily se vlasy,

jó, kde jsou ty časy.

\sloka
Armáda střežila nám klidné spaní

a žáci byli disciplinovaní,

čtyřicet korun stálo maso hovězí

a vůkol znělo to, že pravda vítězí,

dnes už to není, co to bývalo,

všecko se jaksi blbě splantalo,

jediné, co mi dneska zbývá asi,

je zpívat: jó, kde jsou ty časy,

jó, kde jsou ty časy, staré dobré časy,

hoho.

\end{varwidth}

\end{centerjustified}
}
\setcounter{Slokočet}{0}
\end{song}
