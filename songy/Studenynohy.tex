\begin{song}{title=\centering Studený nohy \\\normalsize Radůza\vspace*{-0.3cm}}  %% sem se napíše jméno songu a autor
\moveright \stred \vbox{      %Varianta č. 1  ---> Jeden sloupec zarovnaný na střed	

\predehra /:\textbf{A\,D E\,H\,A\,D\,E}:/

\sloka 
	^{Ami}Prší, ^{Emi\,Dmi}choulím se do ^{Emi}svrchníku,

	^{Ami}než se ^{Emi}otočím ^{F}na ^{G}podpatku
	
	zalesknou se světla na chodníku,
   	
   	jak pětka na věčnou oplátku.

\sloka
	Slyším kroky zakletejch panen,
   	
   	to je vínem, to je ten pozdní sběr,
   
   	každá kosa najde svůj kámen,
   	
   	to je vínem, ber mě, ber.

\refren
	Studený ^{F}nohy ^{E}schovám doma ^{Ami}pod peřinou

	a ráno ^{F}kafe dám si ^{E}hustý jako ^{Ami}tér,

	přežiju ^{F}tuhle ^{E}neděli tak jako ^{Ami}každou jinou,

	na koho ^{F}slovo padne, ^{E}ten je ^{D}solitér. 

\sloka
	Broukám si píseň o klokočí,
   	
   	prší a dlažba leskne se,
   	
   	je chladno a hlava, ta se točí,
   	
   	jak světla na plese.

\refren

\sloka
	Tak mám a nebo nemám kliku,
   
   	zakletá panna směje se
   	
   	a moje oči, lesknou se na chodníku,
   	
   	jak světla na plese.

\refren



}
\setcounter{Slokočet}{0}
\end{song}
