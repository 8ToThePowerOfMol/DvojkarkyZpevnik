\begin{song}{title=\centering Trezor \\\normalsize Karel Gott  \vspace*{-0.3cm}}  %% sem se napíše jméno songu a autor
\moveright 3cm \vbox{      %Varianta č. 1  ---> Jeden sloupec zarovnaný na střed	

\sloka 
	^{D}Ze zdi na mě tupě zírá ^{G}po trezoru temná díra,

	^{E7{\color{white}\_\_\_}}poznám tedy bez nesnází, ^{A7}že tam nepochybně něco schází.

	^{D}Ve zdi byl totiž po dědovi ^{G{\color{white}\_}}velký trezor ocelový.

	^{A7{\color{white}\_}}Mám tedy ztrátu zdánlivě ^{{\color{white}\_\_\_}D}minimální.
      
\sloka
	^{D}Na to že ^{A7{\color{white}\_\_\_}}náhodně v krámě vášnivé dámě ^{D{\color{white}\_}}padl jsem za ^{Hmi}trofej,

	^{E7{\color{white}\_\_}}tvrdila pevně: ,,Přijdu tě levně, ^{A7{\color{white}\_\_\_}}nezoufej!`` Ó jé, jé, jé, jé.

	^{D{\color{white}\_\_}}Jenže potom v naší vile ^{G{\color{white}\_\_\_}}chovala se zhůvěřile,

	^{E7}aby měla správné klima, ^{A7}dal jsem ji do trezoru, ať v klidu dřímá.
      
\refren
	^{D{\color{white}\_\_\_}}Spánku se bráním už noc pátou, ^{G}ne ale žalem nad tou ztrátou,

	^{A7}jen hynu bázní, že kasař úlovek ^{G D}vrátí.
      
\sloka = 2.
      
\refren


}
\setcounter{Slokočet}{0}
\end{song}

