\begin{song}{title=\centering Už to nenapravím \\\normalsize Jaroslav Samson Lenk  \vspace*{-0.3cm}}  %% sem se napíše jméno songu a autor
\moveright 2cm \vbox{      %Varianta č. 1  ---> Jeden sloupec zarovnaný na střed	

\refren /: ^{Ami}Vap tada dap\elipsa\dots ^{D\,\,F\,\,E} :/

\sloka
	V ^{Ami}devět hodin dvacet pět mě ^{D{\color{white}\_\_\_}}opustilo štěstí, 
	
	ten ^{F{\color{white}\_}}vlak, co jsem jím měl jet, na koleji ^{E{\color{white}\_\_}}dávno ^{E7{\color{white}\_}}nestál.
	
	V ^{Ami}devět hodin dvacet pět ^{D{\color{white}\_}}jako bych dostal pěstí,
	
	já ^{F}za hodinu na náměstí měl jsem ^{E{\color{white}\_}}stát, ale ^{E7}v jiným městě.
	
	
	Tvá ^{Ami}zpráva zněla prostě a byla tak krátká,
		
	že ^{Dmi{\color{white}\_}}stavíš se jen na skok, že nechalas mi vrátka ^{G{\color{white}\_\_}}zadní otevřená, ^{E{\color{white}\_\_}}zadní otevře^{E7}ná.
	
	Já ^{Ami{\color{white}\_}}naposled tě viděl když ti bylo dvacet 
	
	a ^{Dmi}to si tenkrát řekla, že už se nechceš vracet, ^{G}že si unavená, ^{E}ze mě unave^{E7}ná.
	


\refren

\sloka
	Já čekala jsem, hlavu jako střep a zdálo se, že dlouho, 
	
	snad může za to vinný sklep, že člověk často sleví.
	
	Já čekala jsem, hlavu jako střep, podvědomou touhou, 
	
	já čekala jsem dobu dlouhou víc než dost, kolik přesně nevím.


	Pak jedenáctá bila a už to bylo passé, 

	já dřív jsem měla vědět, že vidět tě chci zase, že láska nerezaví, láska nerezaví.
	
	Ten dopis, co jsem psala byl dozajista hloupý,
	
	byl odměřený moc, na vlídný slovo skoupý, už to nenapravím, už to nenapravím.
	


\refren



}
\setcounter{Slokočet}{0}
\end{song}
