\begin{song}{title=\predtitle \centering V Jejích Šlépějích \\\large Traband  \vspace*{-0.3cm}}  %% sem se napíše jméno songu a autor


\vetsi
\begin{centerjustified}

\sloka
^{D \z}Časně zrána ^{A}za ^{\z D  A}rozbřesku

^{D}lovec ^{\z A}vstal a ^{\z D}nabil ^{\z A}zbraň.

^{Hmi}Pán ^{\z F#mi}mluvícího ^{Hmi \z}blesku ^{F#mi}

^{Hmi}kráčel tam, kde ^{F#mi}tušil ^{Hmi}laň. ^{A}

^{G\z}A~za ním skrytá ^{C \z}někde v ^{G \z}houští, ^{C}

^{G\z}v~listí, v trávě, ^{C}ve ^{G}větvích. ^{D}

^{Emi\z}Ta,~po ^{\z Hmi}které on ^{Emi \z}toužil ^{Hmi}

šla ^{Emi\z}za~ním v jeho ^{Hmi  Emi \z}šlépějích. ^{A}

\sloka
Cítil, jak je blízko celá.

Cítil ji až do kostí.

A všechny póry jeho těla

se chvěly divnou žádostí.

Krok za krokem ji četl v kraji,

jak dítě slabikuje z knih.

Nezná ji, a přesto zná ji,

jde za ní v jejích šlépějích.

\end{varwidth}\mezisloupci\begin{varwidth}[t]{0.48\textwidth}\setlength{\parindent}{0.45cm}

\sloka
^{}Náhle přiložil zbraň k tváři,

znehybněl a zamířil.

A z hlavně v oslnivé záři

vyšla střela hledat cíl.

A kdesi z hloubky, možná z hrobu

se ozval něčí tichý smích.

Ta, která celou tu dobu

šla za ním v jeho šlépějích.


^{Dmi Ami A7}

\sloka
A od té doby bloudí v lesích,

podél potoků a kolem skal.

Ten, jehož budiž jméno,

trefil, ale nezískal.

To šílenství ho stále vrací

tam, kde krví zrudl sníh.

A i když se mu ona ztrácí,

jde za ní v jejích šlépějích.

Jde za ní v jejích šlépějích.

\end{centerjustified}


\vspace*{0.5cm}
\textit{Pozn.: od 2. sloky dvouhlasný kánon.}

\setcounter{Slokočet}{0}
\end{song}
