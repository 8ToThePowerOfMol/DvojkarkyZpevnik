\begin{song}{title=\centering Vědel pan Cicero \\\normalsize \vspace*{-0.3cm}}  %% sem se napíše jméno songu a autor
\moveright \stred \vbox{      %Varianta č. 1  ---> Jeden sloupec zarovnaný na střed	

\sloka
^{C{\color{white}\_\_\_}}Vědel pan Cicero,

věděl toho ^{G{\color{white}\_\_\_\_}}vícero,

\phantom{.}

neuměl pro nás však

vymyslet ^{C{\color{white}\_\_\_\_\_}}táborák.

\refren
^{F}Na sever, západ, ^{C{\color{white}\_\_\_\_}}východ, jih

neseme do lesů ^{G{\color{white}\_\_\_\_}}zdravý smích,

\phantom{.}

každý musí chápat, 

že, ^{C}kdo tábory ^{C7{\color{white}\_\_\_\_}}vymyslel,

^{F}že měl ^{G{\color{white}\_\_\_\_}}dobrej ^{\,\,C}nápad.


\sloka
Věděl pan Kakadu,

věděl toho hromadu,

neuměl pro nás však

vymyslet almaru.


\refren
Na sever, západ, východ, jih

almarou trestáme každý hřích,

každý musí chápat, 

že, kdo almary vymyslel,

že měl špatnej nápad.


\sloka
Každý, kdo nás uslyší,

v náš tábor pospíší,

honem jde pakovat,

s námi se radovat.

\refren (první)



}
\setcounter{Slokočet}{0}
\end{song}
