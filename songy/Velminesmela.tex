\begin{song}{title=\centering Velmi nesmělá \\\normalsize Jablkoň  \vspace*{-0.3cm}}  %% sem se napíše jméno songu a autor
\moveright 2cm \vbox{      %Varianta č. 1  ---> Jeden sloupec zarovnaný na střed	
\begin{minipage}[t]{0.48\textwidth}\setlength{\parindent}{0.45cm}  %Varianta č. 2 --> Dva sloupce
\sloka
	^{Ami}Potkali se v pondělí v ^{Emi\,Ami}pondělí.
	
	^{Ami}Byli velmi nesmělí ^{G\,E7}nesmělí
	
	^{C}A tak oba dělali ^{Dmi7\,Esmi7\,Emi7}dělali
	
	^{Ami}Jakoby se neznali ^{Emi\,Ami}neznali
	
	
\sloka
	V úterý sebral odvahu odvahu.
	
	Odhodlal se k pozdravu pozdravu.
	
	A pak v citové panice panice
	
	prchali oba k mamince mamince.

\refren
	^{C\,E}Semafor popásá ^{F}chodce
	
	motorky ^{C}auta ^{G\,C}tramvaje.
	
	A všechny ^{E}cesty dneska ^{F}vedou
	
	/: ^{C}do pekla i ^{G}do ^{C\,Ami}ráje. :/
	
\sloka
	Ve středu spolu postáli postáli,
	
	dívali se do dáli do dáli.
	
	A do dáli se dívali dívali
	
	i když už spolu nestáli nestáli.
	

\sloka
	Ve čtvrtek přišel první zvrat první zvrat,
	
	prohlásil že má ji rád má ji rád.
	
	A ona špitla do ticha do ticha,
	
	že na ni moc pospíchá pospíchá

\refren

\sloka
	V pátek to vzal útokem útokem,
	
	jak tak šli krok za krokem za krokem.
	
	Přesně v šestnáct dvacet pět dvacet pět
	
	zavadil loktem o loket o loket.

\sloka
	V sobotu ji chyt za ruku za ruku.
	
	Hlavou jí kmitlo je to tu je to tu.
	
	A jak hodiny běžely běžely,
	
	drželi se drželi drželi.

\end{minipage}\begin{minipage}[t]{0.48\textwidth}\setlength{\parindent}{0.45cm}\vspace*{0.55cm}  % V případě varianty č.2 jde odsud text do pravé části

\refren
	
\sloka
	V neděli už věděli věděli,

	že jsou možná dospělí dospělí.
	
	A tak při sedmém pokusu pokusu,
	
	dal jí pusu na pusu na pusu.


\sloka
	Když zas přišlo pondělí pondělí,
	
	příšerně se styděli styděli.
	
	A tak oba dělali dělali,
	
	jakoby se neznali neznali.

\end{minipage}
}
\setcounter{Slokočet}{0}
\end{song}
