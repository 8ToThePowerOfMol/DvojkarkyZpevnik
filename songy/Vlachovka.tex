\begin{song}{title=\centering Vlachovka \\\normalsize Tři sestry  \vspace*{-0.3cm}}  %% sem se napíše jméno songu a autor
\moveright 5.5cm \vbox{      %Varianta č. 1  ---> Jeden sloupec zarovnaný na střed	


\sloka
Až budu ^{F{\color{white}\_\_\_\_\_\_}}rockovej ^{{\color{white}\_\_\_\_}G}důchodce,

budou hrát ^{C\,\,}Tři Sestry na ^{{\color{white}\_\_\_}Ami}Vlachovce,

Každej rok ^{F{\color{white}\_\_\_\_}}budeme ve ^{G}Varech

a já na ^{C\,\,}smrt ^{\,\,G}čekat v ^{{\color{white}\_}C}Jevanech.

Zřídka budou nejspíš produkce,

mládež nás bude mít za tupce.

Budem hrát na pohřbech fan clubu,

inženýr učí se na tubu.\elipsa.\elipsa.\elipsa.


\sloka
Trpíme srdeční slabostí,

každý si stěžuje na kosti,

do kalhot čůrat se nestydím

už ani s brýlemi nevidím.\elipsa.\elipsa.\elipsa.

Holky nás pouštějí v tramvaji

a z Olšan hrobníci mávají,

pijeme už jen rum za rumem,

snad brzo budeme pod drnem.


\refren
Fanánek má infarkt na baru,

kapelník umírá v Subaru,

zpívat chci, však sláb jsem na plíce,

houslista sáhl po devítce.

Jsme strašně starý a nemocný,

bubenec hraje jak ponocný,

Supici dostala skleróza,

fanoušky zabila cirhóza.


\sloka
A že nás nelákaj kavárny,

můžem jít o holi z Kovárny

a klidně svalit se v podchodu,

stáří má velikou výhodu,

už na nás nevolaj záchytky,

vědí že míříme pod kytky.

Důchodce má sotva na nájem

a budí v okolí nezájem.


\refren


}
\setcounter{Slokočet}{0}
\end{song}
